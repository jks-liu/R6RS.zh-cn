\documentclass[twoside,twocolumn]{algol60}
%\documentclass[twoside]{algol60}


\pagestyle{headings}
\showboxdepth=0
\makeindex
% Macros for R^nRS.

% tex2page.sty mucks with in some manner
\let\centerlinesaved=\centerline
\usepackage{tex2page}
\let\centerline=\centerlinesaved

\usepackage{xr-hyper}

\usepackage{makeidx}
\usepackage{hyperref}

% \let\htmlonly=\iffalse
% \let\endhtmlonly=\fi
% \let\texonly=\iftrue
% \let\endtexonly=\fi

\makeatletter

\texonly
\newcommand{\topnewpage}{\@topnewpage}
\endtexonly

\htmlonly
\newcommand{\topnewpage}[0][]{#1}
\endhtmlonly

\newcommand{\authorsc}[1]{{\scriptsize\scshape #1}}

% Chapters, sections, etc.

\newcommand{\extrapart}[1]{
 % \chapter{#1}
  \chapter*{#1}
  \markboth{#1}{#1}
  \vskip 1ex
  \addcontentsline{toc}{chapter}{#1}}

\newcommand{\clearchaptergroupstar}[1]{
  \texonly
  \clearpage
  \addcontentsline{toc}{chaptergroup}{#1}
  \topnewpage[
    \centerline{\large\bf\uppercase{#1}}
    \bigskip]
    \endtexonly
  }

\newcommand{\clearchapterstar}[1]{
  \clearpage
  \topnewpage[
    \centerline{\large\bf\uppercase{#1}}
    \bigskip]}

\newcommand{\clearextrapart}[1]{
  \clearchapterstar{#1}
  \markboth{#1}{#1}
  \addcontentsline{toc}{chapter}{#1}}

\newcommand{\vest}{}
\newcommand{\dotsfoo}{$\ldots\,$}

\newcommand{\sharpfoo}[1]{{\tt\##1}}
\newcommand{\schfalse}{\sharpfoo{f}}
\newcommand{\schtrue}{\sharpfoo{t}}

\newcommand{\ampfoo}[1]{{\tt\&#1}}

\newcommand{\libfoo}[1]{{\tt(#1)}}

\newcommand{\singlequote}{{\tt'}}  %\char19
\newcommand{\doublequote}{{\tt"}}
\newcommand{\backquote}{{\tt\char18}}
\newcommand{\backwhack}{{\tt\char`\\}}
\newcommand{\comma}{{\tt\char`\@}}
\newcommand{\atsign}{{\tt\char`\@}}
\newcommand{\bang}{{\tt\char`\!}}
\newcommand{\sharpsign}{{\tt\#}}
\newcommand{\verticalbar}{{\tt|}}
\newcommand{\openbracket}{{\tt\char`\[}}
\newcommand{\closedbracket}{{\tt\char`\]}}
\newcommand{\ampersand}{{\tt\char`\&}}

\newcommand{\coerce}{\discretionary{->}{}{->}}

% Knuth's \in sucks big boulders
\def\elem{\hbox{\raise.13ex\hbox{$\scriptstyle\in$}}}

\newcommand{\meta}[1]{{\noindent\hbox{\rm$\langle$#1$\rangle$}}}
\let\hyper=\meta
\newcommand{\hyperi}[1]{\hyper{#1$_1$}}
\newcommand{\hyperii}[1]{\hyper{#1$_2$}}
\newcommand{\hyperiii}[1]{\hyper{#1$_3$}}
\newcommand{\hyperiv}[1]{\hyper{#1$_4$}}
\newcommand{\hyperj}[1]{\hyper{#1$_i$}}
\newcommand{\hypern}[1]{\hyper{#1$_n$}}
\texonly
\newcommand{\var}[1]{\noindent\hbox{\textnormal{\textit{#1}}}}
\endtexonly
\htmlonly
\newcommand{\var}[1]{\textnormal{\textit{#1}}}
\endhtmlonly
\newcommand{\vari}[1]{\var{#1$_1$}}
\newcommand{\varii}[1]{\var{#1$_2$}}
\newcommand{\variii}[1]{\var{#1$_3$}}
\newcommand{\variv}[1]{\var{#1$_4$}}
\newcommand{\varj}[1]{\var{#1$_j$}}
\newcommand{\vark}[1]{\var{#1$_k$}}
\newcommand{\varn}[1]{\var{#1$_n$}}

\newcommand{\vr}[1]{{\noindent\hbox{$#1$\/}}}  % Careful, is \/ always the right thing?
\newcommand{\vri}[1]{\vr{#1_1}}
\newcommand{\vrii}[1]{\vr{#1_2}}
\newcommand{\vriii}[1]{\vr{#1_3}}
\newcommand{\vriv}[1]{\vr{#1_4}}
\newcommand{\vrv}[1]{\vr{#1_5}}
\newcommand{\vrj}[1]{\vr{#1_j}}
\newcommand{\vrn}[1]{\vr{#1_n}}

%%R4%% The excessive use of the code font in the numbers section was
% confusing, somewhat obnoxious, and inconsistent with the rest
% of the report and with parts of the section itself.  I added
% a \tupe no-op, and changed most old uses of \type to \tupe,
% to make it easier to change the fonts back if people object
% to the change.

\newcommand{\type}[1]{{\it#1}}
\newcommand{\tupe}[1]{{#1}}

\newcommand{\defining}[1]{\mainindex{#1}{\em #1}}
\newcommand{\ide}[1]{{\schindex{#1}\frenchspacing\tt{#1}}}

\newcommand{\lambdaexp}{{\cf lambda} expression}

\newcommand{\callcc}{{\tt call-with-current-continuation}}

\newcommand{\mainschindex}[1]{\label{#1}\index{#1@\texttt{#1}}}
\newcommand{\mainindex}[1]{\index{#1}}
\newcommand{\schindex}[1]{\index{#1@\texttt{#1}}}
\newcommand{\sharpindex}[1]{\index{#1@\texttt{\#{}#1}}}
\newcommand{\sharpbangindex}[1]{\index{#1@\texttt{\#!#1}}}
\newcommand{\ampindex}[1]{\index{#1@\texttt{\&{}#1}}}
\newcommand{\libindex}[1]{\index{#1@\texttt{(#1)}}}

\texonly
\newcommand{\extref}[2]{\ref{#1}}
\endtexonly
\htmlonly
\newcommand{\extref}[2]{on ``#2''}
\endhtmlonly

\renewenvironment{theindex}
{\texonly\clearpage\endtexonly
\topnewpage[
    \begin{center}
      \large\bf\MakeUppercase{\indexheading}
    \end{center}
    \vskip 1ex \bigskip]
    \markboth{Index}{Index}
    \addcontentsline{toc}{chapter}{\indexheading}
    \parindent\z@
    \texonly\parskip\z@ plus .1pt\endtexonly\relax\let\item\@idxitem
    \indexintro\par\bigskip}
               {\texonly\clearpage\endtexonly}


\newcommand{\domain}[1]{#1}
\newcommand{\nodomain}[1]{}
%\newcommand{\todo}[1]{{\rm$[\![$!!~#1$]\!]$}}
\newcommand{\todo}[1]{}

% \frobq will make quote and backquote look nicer.
\def\frobqcats{%\catcode`\'=13
\catcode`\`=13{}}
{\frobqcats
\gdef\frobqdefs{%\def'{\singlequote}
\def`{\backquote}}}
\def\frobq{\frobqcats\frobqdefs}

% \cf = code font
% Unfortunately, \cf \cf won't work at all, so don't even attempt to
% next constructions which use them...
\newcommand{\cf}{\frenchspacing\frobq\tt}

\texonly
% Same as \obeycr, but doesn't do a \@gobblecr.
{\catcode`\^^M=13 \gdef\myobeycr{\catcode`\^^M=13 \def^^M{\\}}%
\gdef\restorecr{\catcode`\^^M=5 }}
\endtexonly

{\obeyspaces\gdef {\hbox{\hskip0.5em}}}

\gdef\gobblecr{\@gobblecr}

\def\setupcode{\@makeother\^}

% Scheme example environment
% At 11 points, one column, these are about 56 characters wide.
% That's 32 characters to the left of the => and about 20 to the right.

\newcommand{\exception}[1]{{\cf#1} \textnormal{\textit{exception}}}
\newenvironment{schemenoindent}{
  % Commands for scheme examples
  \newcommand{\ev}{\>\>\evalsto}
  \newcommand{\xev}{\>\>\hspace*{-1em}\evalsto}
  \newcommand{\lev}{\\\>\evalsto}
  \newcommand{\unspecified}{{\em{}unspecified}}
  \newcommand{\theunspecified}{{\em{}unspecified}}
  \setupcode
  \small \cf \obeyspaces \myobeycr
  \begin{tabbing}%
\qquad\=\hspace*{5em}\=\hspace*{9em}\=\evalsto~\=\kill%   was 16em
\gobblecr}{\unskip\end{tabbing}}

%\newenvironment{scheme}{\begin{schemenoindent}\+\kill}{\end{schemenoindent}}
\newenvironment{scheme}{
  % Commands for scheme examples
  \newcommand{\ev}{\>\>\evalsto}
  \newcommand{\xev}{\>\>\hspace*{-1em}\evalsto}
  \newcommand{\lev}{\\\>\evalsto}
  \renewcommand{\em}{\rmfamily\itshape}
  \newcommand{\unspecified}{{\em{}unspecified}}
  \newcommand{\theunspecified}{{\em{}unspecified}}
  \setupcode
  \small \cf \obeyspaces \myobeycr
  \begin{tabbing}%
\qquad\=\hspace*{5em}\=\hspace*{9em}\=\evalsto~\=\+\kill%   was 16em
\gobblecr}{\unskip\end{tabbing}}

\texonly
\newcommand{\evalsto}{$\Longrightarrow$}
\endtexonly
 \htmlonly
\newcommand{\evalsto}{$\Rightarrow$}
\endhtmlonly

% Rationale

\newenvironment{rationale}{%
\bgroup\small\noindent{\em Rationale:}\space}{%
\egroup}

% Notes

\newenvironment{note}{%
\bgroup\small\noindent{\em Note:}\space}{%
\egroup}

% Names of library modules

\newcommand{\library}[1]{{\tt (#1)}}
\newcommand{\deflibrary}[1]{\library{#1}\libindex{#1}}

\newcommand{\rsixlibrary}[1]{\library{rnrs #1 (6)}}
\newcommand{\defrsixlibrary}[1]{\deflibrary{rnrs #1 (6)}}

\newcommand{\thersixlibrary}{\library{rnrs (6)}}
\newcommand{\defthersixlibrary}{\deflibrary{rnrs (6)}}

% Manual entries

\newenvironment{entry}[1]{
  \vspace{3.1ex plus .5ex minus .3ex}\noindent#1%
\unpenalty\nopagebreak}{\vspace{0ex plus 1ex minus 1ex}}

\newcommand{\exprtype}{syntax}

\newcommand{\unspecifiedreturn}{unspecified values}
\newcommand{\isunspecified}{is unspecified}
\newcommand{\areunspecified}{are unspecified}

% Primitive prototype
\newcommand{\pproto}[2]{\unskip%
\hbox{\cf\spaceskip=0.5em#1}\hfill\penalty 0%
\hbox{ }\nobreak\hfill\hbox{\rm #2}\break}

% Parenthesized prototype
\newcommand{\proto}[3]{\pproto{(\mainschindex{#1}\hbox{#1}{\it#2\/})}{#3}}

% Variable prototype
\newcommand{\vproto}[2]{\mainschindex{#1}\pproto{#1}{#2}}

% Condition-type prototype
 \newcommand{\ctproto}[1]{\ampindex{#1}\pproto{\ampfoo{#1}}{condition type}}

% Prototype for literal syntax, no index
\newcommand{\litprotonoindex}[1]{\pproto{#1}{auxiliary syntax}}

% Prototype for literal syntax
\newcommand{\litproto}[1]{\mainschindex{#1}\litprotonoindex{#1}}

% Prototype for literal syntax at level 1, no index
\newcommand{\litprotoexpandnoindex}[1]{\pproto{#1}{auxiliary syntax ({\tt expand)}}}

% Prototype for literal syntax at level 1
\newcommand{\litprotoexpand}[1]{\mainschindex{#1}\litprotoexpandnoindex{#1}}

% Extending an existing definition (\proto without the index entry)
\newcommand{\rproto}[3]{\pproto{(\hbox{#1}{\it#2\/})}{#3}}

% Extending an existing definition, with index entry
\newcommand{\irproto}[3]{\schindex{#1}\rproto{#1}{#2}{#3}}

% Variable prototype
\newcommand{\rvproto}[2]{\pproto{#1}{#2}}

% Grammar environment

\newenvironment{grammar}{
  \def\:{\goesto{}}
  \def\|{$\vert$}
  \cf \myobeycr
  \begin{tabbing}
    %\qquad\quad \= 
    \qquad \= $\vert$ \= \kill
  }{\unskip\end{tabbing}}

%\newcommand{\unsection}{\unskip}
\newcommand{\unsection}{{\vskip -2ex}}

% Allow line break after hyphen
\newcommand{\hp}{\linebreak[0]}

\texonly
\newcommand{\itspace}{\hspace{1pt}}
\endtexonly
\htmlonly
\newcommand{\itspace}{}
\endhtmlonly

% Commands for grammars
\newcommand{\arbno}[1]{#1\hbox{\rm*}}  
\newcommand{\atleastone}[1]{#1\hbox{$^+$}}

\texonly
\newcommand{\goesto}{$\longrightarrow$}
\endtexonly
 \htmlonly
\newcommand{\goesto}{$\rightarrow$}
\endhtmlonly

\newcommand{\syntax}{{\em Syntax: }}
\newcommand{\semantics}{{\em Semantics: }}
\newcommand{\implresp}{{\em Implementation responsibilities: }}

\newcommand{\rrs}[1]{\textit{Revised$^#1$ Report on the Algorithmic Language Scheme}}

\newcommand{\libindexentry}[1]{#1 (library)}

\makeatother

\def\rnrsrevision{6}
\def\rnrsrevisiondate{26 September 2007}


%!TEX root = r6rs.tex

\usepackage{latexsym}
\usepackage{mathrsfs}
\usepackage{stmaryrd}

\newcounter{subfig}
\newcommand{\subfigurestart}{\renewcommand{\thefigure}{A.\arabic{figure}\alph{subfig}}\setcounter{subfig}{1}}


% needed for the second thru the nth figure
\newcommand{\subfigureadjust}{\addtocounter{figure}{-1}\addtocounter{subfig}{1}}

\newcommand{\subfigurestop}{\renewcommand{\thefigure}{A.\arabic{figure}}}


\newcommand{\semanticsindex}[2]{\index{#1@{\texttt{#1} (formal semantics)}}}

\newcommand{\pltreducks}{PLT Redex}
\newcommand{\rnrs}{Report}
\newcommand{\rnrslongspace}{\mbox{Revised\ensuremath{\,^{\mbox{\textrm{\scriptsize 5}}}} Report on Scheme}}
\newcommand{\rnrslong}{\mbox{Revised\ensuremath{^{\mbox{\textrm{\scriptsize 5}}}} Report on Scheme}}
\newcommand{\largernrslong}{\mbox{Revised\ensuremath{\,^{\mbox{\textrm{\large 5}}}} Report on Scheme}}

%\newenvironment*{proof}
%{\noindent\textbf{Proof} }
%{$\Box$ \\}

%\newcommand{\either}{*\!{}\!{}\!\!\circ}
\newcommand{\either}{*\!\circ}

\newcommand{\hole}{[~]}
\newcommand{\holes}{\ensuremath{\hole_{\star}}}
\newcommand{\holeone}{\ensuremath{\hole_\circ}}
\newcommand{\holeany}{\ensuremath{\hole_{\either}}}

%% multi-letter nonterminals (one-letter can be done with $_$)
\newcommand{\nt}[1]{\textnormal{\textit{#1}}}

%\newcommand{\sy}[1]{\textnormal{\textbf{#1}}}
%\newcommand{\va}[1]{\textnormal{\textsf{#1}}}

\newcommand{\sy}[1]{{\cf #1}}
\newcommand{\va}[1]{{\cf #1}}


\newcommand{\beginF}{\ensuremath{\textbf{begin}^{\mbox{\textrm{\textbf{\scriptsize F}}}}}}
\newcommand{\Eo}{\ensuremath{E^{\circ}}}
\newcommand{\Estar}{\ensuremath{E^{\star}}}
\newcommand{\Fo}{\ensuremath{F^{\circ}}}
\newcommand{\Fstar}{\ensuremath{F^{\star}}}
\newcommand{\Io}{\ensuremath{I^{\circ}}}
\newcommand{\Istar}{\ensuremath{I^{\star}}}

\imgdef\calP{\ensuremath{\mathcal{P}}}
\imgdef\calS{\ensuremath{\mathcal{S}}}
\imgdef\calR{\ensuremath{\mathcal{R}}}
\imgdef\calRv{\ensuremath{\mathcal{R}_v}}
\imgdef\calA{\ensuremath{\mathcal{A}}}
\imgdef\scrO{\ensuremath{\mathscr{O}}}

\newcommand{\semfalse}{\texttt{\#f}}
\newcommand{\semtrue}{\texttt{\#t}}

\newcommand{\aline}{\noindent\hrulefill\par}

%\def\beginfig{\begin{figure*}[t]{\noindent\hrulefill\par}\small}
%\def\endfig{{\noindent\hrulefill\par}\end{figure*}}

\def\beginfig{\begin{figure*}[tb!]{\noindent\par}\small}
\def\endfig{{\noindent\hrulefill\par}\end{figure*}}

\newcommand{\dom}{\textit{dom}}

\newcommand{\gopen}{{^{\scriptscriptstyle\lceil}\!\!}}
\newcommand{\gclose}{\!\!{}^{\scriptscriptstyle\rceil}}

\newcommand{\mrk}{\diamond}
\newcommand{\umrk}{^\mrk}

\newcommand{\rulename}[1]{\textsf{[#1]}}

\newcommand{\extraspterm}{\\[6pt]}

\newcommand{\twolinerule}[3]{\twolineruleA{#1}{#2}{\rulename{#3}}{\rightarrow}}
\newcommand{\twolinescrule}[4]{\twolinescruleA{#1}{#2}{\rulename{#3}}{#4}{\rightarrow}}
\newcommand{\onelinerule}[3]{\onelineruleA{#1}{#2}{\rulename{#3}}{\rightarrow}}
\newcommand{\onelinescrule}[4]{\onelinescruleA{#1}{#2}{\rulename{#3}}{#4}{\rightarrow}}

\newcommand{\twolineruleA}[4]{
\multicolumn{3}{l}{{#1} {#4}} & {#3}\\ 
\multicolumn{3}{l}{{#2}} & \extraspterm}

\newcommand{\twolinescruleA}[5]{
\multicolumn{3}{l}{{#1} {#5}} & {#3}\\ 
\multicolumn{4}{l}{{#2 ~ ~ ~ {#4}}} \extraspterm}

\newcommand{\twolinescruleB}[5]{
\multicolumn{3}{l}{{#1} {#5}} & {#3}\\ 
\multicolumn{4}{l}{#2} \\
\multicolumn{4}{l}{~ ~ ~ #4} \extraspterm}

\newcommand{\threelinescruleA}[5]{
\multicolumn{3}{l}{{#1} {#5}} & {#4}\\ 
\multicolumn{4}{l}{#2} \\
\multicolumn{4}{l}{#3} \extraspterm}

\newcommand{\threelinescruleB}[6]{
\multicolumn{3}{l}{{#1} {#6}} & {#4}\\ 
\multicolumn{4}{l}{#2} \\
\multicolumn{4}{l}{#3} \\
\multicolumn{4}{l}{~ ~ ~ #5} \extraspterm}


\newcommand{\fourlinescruleB}[7]{
\multicolumn{3}{l}{{#1} {#7}} & {#5}\\ 
\multicolumn{4}{l}{#2} \\
\multicolumn{4}{l}{#3} \\
\multicolumn{4}{l}{#4} \\
\multicolumn{4}{l}{~ ~ ~ #6} \extraspterm}


\newcommand{\onelineruleA}[4]{
\multicolumn{1}{l}{#1} & {#4} ~ & {#2} & {#3} \extraspterm}

\newcommand{\onelinescruleA}[5]{
\multicolumn{1}{l}{#1} & {#5} ~ & {#2} & {#3} \\
& & {#4} \extraspterm}


\texonly
\externaldocument[lib:]{r6rs-lib}
\endtexonly

\usepackage[utf8]{inputenc}
\usepackage{CJKutf8}

\def\headertitle{Scheme第$^{\rnrsrevision}$次修订简体中文翻译}
\def\TZPtitle{算法语言Scheme修订^\rnrsrevision{}报告}

\begin{document}
\begin{CJK*}{UTF8}{gbsn}

\thispagestyle{empty}

\topnewpage[{
\begin{center}   {\huge\bf
    算法语言Scheme修订
    {\Huge$^{\mathbf{\htmlonly\tiny\endhtmlonly{}\rnrsrevision}}$}报告}

\vskip 1ex
$$
\begin{tabular}{l@{\extracolsep{.5in}}lll}
\multicolumn{4}{c}{M\authorsc{ICHAEL} S\authorsc{PERBER}}
\\
\multicolumn{4}{c}{R.\ K\authorsc{ENT} D\authorsc{YBVIG},
  M\authorsc{ATTHEW} F\authorsc{LATT},
  A\authorsc{NTON} \authorsc{VAN} S\authorsc{TRAATEN}}
\\
\multicolumn{4}{c}{(\textit{编辑})} \\
\multicolumn{4}{c}{
  R\authorsc{ICHARD} K\authorsc{ELSEY}, W\authorsc{ILLIAM} C\authorsc{LINGER},
  J\authorsc{ONATHAN} R\authorsc{EES}} \\
\multicolumn{4}{c}{(\textit{编辑,算法语言Scheme修订\itspace{}$^5$报告})} \\
\multicolumn{4}{c}{
  R\authorsc{OBERT} B\authorsc{RUCE} F\authorsc{INDLER}, J\authorsc{ACOB} M\authorsc{ATTHEWS}} \\
\multicolumn{4}{c}{(\textit{作者,形式语义})} \\[1ex]
\multicolumn{4}{c}{\bf \rnrsrevisiondate}
\end{tabular}
$$



\end{center}

\chapter*{摘要}
\medskip

{\parskip 1ex

报告给出了程序设计语言Scheme的定义性描述。Scheme是由Guy Lewis Steele Jr.和Gerald Jay Sussman设计的具有静态作用域和严格尾递归特性的Lisp程序设计语言的方言。它的设计目的是以异常清晰,语义简明和较少表达方式的方法来组合表达式。包括函数(functional)式,命令(imperative)式和消息传递(message passing)式风格在内的绝大多数程序设计模式都可以用Scheme方便地表述。

和本报告一起的还有一个描述标准库的报告~\cite{R6RS-libraries};用描述符“库的第多少小节(library section)”或“库的第多少章(library chapter)”来识别此文档的引用。和它一起的还有一个包含非规范性附录的报告~\cite{R6RS-appendices}。第四次报告在语言和库的许多方面阐述了历史背景和基本原理~\cite{R6RS-rationale}。

\medskip

上面列到的人不是这篇报告文字的唯一作者。多年来,下面这些人也参与到Scheme语言设计的讨论中,我们也将他们列为之前报告的作者:

Hal Abelson,Norman Adams,David Bartley,Gary Brooks,William Clinger,R. Kent Dybvig,Daniel Friedman,Robert Halstead,Chris Hanson,Christopher Haynes,Eugene Kohlbecker,Don Oxley,Kent Pitman,Jonathan Rees,Guillermo Rozas,Guy L. Steele Jr.,Gerald Jay Sussman和Mitchell Wand。

为了突出最近的贡献,他们没有被列为本篇报告的作者。然而,他们的贡献和服务应被确认。

\medskip

我们认为这篇报告属于整个Scheme社区,并且我们授权任何人复制它的全部或部分。我们尤其鼓励Scheme的实现者使用本报告作为手册或其它文档的起点,必要时也可以对它进行修改。
}

\bigskip

\input{status}
}]

\texonly\clearpage\endtexonly

\chapter*{目录}
\addvspace{3.5pt}                  % don't shrink this gap
\renewcommand{\tocshrink}{-4.0pt}  % value determined experimentally
{
\tableofcontents
}

\vfill
\eject


\clearextrapart{Introduction}

\label{historysection}

Programming languages should be designed not by piling feature on top of
feature, but by removing the weaknesses and restrictions that make additional
features appear necessary.  Scheme demonstrates that a very small number
of rules for forming expressions, with no restrictions on how they are
composed, suffice to form a practical and efficient programming language
that is flexible enough to support most of the major programming
paradigms in use today.

Scheme
was one of the first programming languages to incorporate first-class
procedures as in the lambda calculus, thereby proving the usefulness of
static scope rules and block structure in a dynamically typed language.
Scheme was the first major dialect of Lisp to distinguish procedures
from lambda expressions and symbols, to use a single lexical
environment for all variables, and to evaluate the operator position
of a procedure call in the same way as an operand position.  By relying
entirely on procedure calls to express iteration, Scheme emphasized the
fact that tail-recursive procedure calls are essentially gotos that
pass arguments.  Scheme was the first widely used programming language to
embrace first-class escape procedures, from which all previously known
sequential control structures can be synthesized.  A subsequent
version of Scheme introduced the concept of exact and inexact number objects,
an extension of Common Lisp's generic arithmetic.
More recently, Scheme became the first programming language to support
hygienic macros, which permit the syntax of a block-structured language
to be extended in a consistent and reliable manner.

\subsection*{Guiding principles}

To help guide the standardization effort, the editors have adopted a
set of principles, presented below.
Like the Scheme language defined in \rrs{5}~\cite{R5RS}, the language described
in this report is intended to:

\begin{itemize}
\item allow programmers to read each other's code, and allow
  development of portable programs that can be executed in any
  conforming implementation of Scheme;

\item derive its power from simplicity, a small number of generally
  useful core syntactic forms and procedures, and no unnecessary
  restrictions on how they are composed;
  
\item allow programs to define new procedures and new hygienic
  syntactic forms;
  
\item support the representation of program source code as data;
  
\item make procedure calls powerful enough to express any form of
  sequential control, and allow programs to perform non-local control
  operations without the use of global program transformations;
  
\item allow interesting, purely functional programs to run indefinitely
  without terminating or running out of memory on finite-memory
  machines;
  
\item allow educators to use the language to teach programming
  effectively, at various levels and with a variety of pedagogical
  approaches; and

\item allow researchers to use the language to explore the design,
  implementation, and semantics of programming languages.
\end{itemize}

In addition, this report is intended to:

\begin{itemize}
\item allow programmers to create and distribute substantial programs
  and libraries, e.g., implementations of Scheme Requests for
  Implementation, that run without
  modification in a variety of Scheme implementations;
  
\item support procedural, syntactic, and data abstraction more fully
  by allowing programs to define hygiene-bending and hygiene-breaking
  syntactic abstractions and new unique datatypes along with
  procedures and hygienic macros in any scope;
  
\item allow programmers to rely on a level of automatic run-time type
  and bounds checking sufficient to ensure type safety; and

\item allow implementations to generate efficient code, without
  requiring programmers to use implementation-specific operators or
  declarations.
\end{itemize}

While it was possible to write portable programs in Scheme as
described in \rrs{5}, and indeed portable Scheme programs were written
prior to this report, many Scheme programs were not, primarily because
of the lack of substantial standardized libraries and the
proliferation of implementation-specific language additions.

In general, Scheme should include building blocks that allow a wide
variety of libraries to be written, include commonly used user-level
features to enhance portability and readability of library and
application code, and exclude features that are less commonly used and
easily implemented in separate libraries.

The language described in this report is intended to also be backward
compatible with programs written in Scheme as described in \rrs{5} to
the extent possible without compromising the above principles and
future viability of the language.  With respect to future viability,
the editors have operated under the assumption that many more Scheme
programs will be written in the future than exist in the present, so
the future programs are those with which we should be most concerned.

\subsection*{Acknowledgements}

Many people contributed significant help to this revision of the
report.  Specifically, we thank Aziz Ghuloum and Andr\'e van Tonder for
contributing reference implementations of the library system.  We
thank Alan Bawden, John Cowan, Sebastian Egner, Aubrey Jaffer, Shiro
Kawai, Bradley Lucier, and Andr\'e van Tonder for contributing insights on
language design.  Marc Feeley, Martin Gasbichler, Aubrey Jaffer, Lars T Hansen,
Richard Kelsey, Olin Shivers, and Andr\'e van Tonder wrote SRFIs that
served as direct input to the report.  Marcus Crestani, David Frese, 
Aziz Ghuloum, Arthur A.\ Gleckler, Eric Knauel, Jonathan Rees, and Andr\'e
van Tonder thoroughly proofread early versions of the report.

We would also like to thank the following people for their
help in creating this report: Lauri Alanko,
Eli Barzilay, Alan Bawden, Brian C.\ Barnes, Per Bothner, Trent Buck,
Thomas Bushnell, Taylor Campbell, Ludovic Court\`es, Pascal Costanza,
John Cowan, Ray Dillinger, Jed Davis, J.A.\ ``Biep'' Durieux, Carl Eastlund,
Sebastian Egner, Tom Emerson, Marc Feeley, Matthias Felleisen, Andy
Freeman, Ken Friedenbach, Martin Gasbichler, Arthur A.\ Gleckler, Aziz
Ghuloum, Dave Gurnell, Lars T Hansen, Ben Harris, Sven Hartrumpf, Dave
Herman, Nils M.\ Holm, Stanislav Ievlev, James Jackson, Aubrey Jaffer,
Shiro Kawai, Alexander Kjeldaas, Eric Knauel, Michael Lenaghan, Felix Klock,
Donovan Kolbly, Marcin Kowalczyk, Thomas Lord, Bradley Lucier, Paulo
J.\ Matos, Dan Muresan, Ryan Newton, Jason Orendorff, Erich Rast, Jeff
Read, Jonathan Rees, Jorgen Sch\"afer, Paul Schlie, Manuel Serrano,
Olin Shivers, Jonathan Shapiro, Jens Axel S\o{}gaard, Jay Sulzberger,
Pinku Surana, Mikael Tillenius, Sam Tobin-Hochstadt, David Van Horn,
Andr\'e van Tonder, Reinder Verlinde, Alan Watson, Andrew Wilcox, Jon
Wilson, Lynn Winebarger, Keith Wright, and Chongkai Zhu.

We would like to thank the following people for their help in creating
the previous revisions of this report: Alan Bawden, Michael
Blair, George Carrette, Andy Cromarty, Pavel Curtis, Jeff Dalton, Olivier Danvy,
Ken Dickey, Bruce Duba, Marc Feeley,
Andy Freeman, Richard Gabriel, Yekta G\"ursel, Ken Haase, Robert
Hieb, Paul Hudak, Morry Katz, Chris Lindblad, Mark Meyer, Jim Miller, Jim Philbin,
John Ramsdell, Mike Shaff, Jonathan Shapiro, Julie Sussman,
Perry Wagle, Daniel Weise, Henry Wu, and Ozan Yigit.

We thank Carol Fessenden, Daniel
Friedman, and Christopher Haynes for permission to use text from the Scheme 311
version 4 reference manual.  We thank Texas Instruments, Inc.~for permission to
use text from the {\em TI Scheme Language Reference Manual}~\cite{TImanual85}.
We gladly acknowledge the influence of manuals for MIT Scheme~\cite{MITScheme},
T~\cite{Rees84}, Scheme 84~\cite{Scheme84}, Common Lisp~\cite{CLtL},
Chez Scheme~\cite{csug7}, PLT~Scheme~\cite{mzscheme352},
and Algol 60~\cite{Naur63}.

\vest We also thank Betty Dexter for the extreme effort she put into
setting this report in \TeX, and Donald Knuth for designing the program
that caused her troubles.

\vest The Artificial Intelligence Laboratory of the
Massachusetts Institute of Technology, the Computer Science
Department of Indiana University, the Computer and Information
Sciences Department of the University of Oregon, and the NEC Research
Institute supported the preparation of this report.  Support for the MIT
work was provided in part by
the Advanced Research Projects Agency of the Department of Defense under Office
of Naval Research contract N00014-80-C-0505.  Support for the Indiana
University work was provided by NSF grants NCS 83-04567 and NCS
83-03325.


%%% Local Variables: 
%%% mode: latex
%%% TeX-master: "r6rs"
%%% End: 
   \par
\vskip 2ex
\clearchaptergroupstar{语言描述} %\unskip\vskip -2ex
\chapter{Scheme概论}
\label{semanticchapter}

本章概述了Scheme的语义。此概述的目的在于充分解释语言的基本概念,以帮助理解本报告的后面的章节,这些章节以参考手册的形式被组织起来。因此,本概述不是本语言的完整介绍,在某些方面也不够精确和规范。

\vest 像Algol语言一样,Scheme是一种静态作用域的程序设计语言。对变量的每一次使用都对应于该变量在词法上的一个明显的绑定。

\vest Scheme中采用的是隐式类型而非显式类型\cite{WaiteGoos}。类型与对象(object)
\mainindex{object 对象}(也称值)相关联,而非与变量相关联。(一些作者将隐式类型的语言称为弱类型或动态类型的语言。)其它采用隐式类型的语言有Python, Ruby, Smalltalk和Lisp的其他方言。采用显式类型的语言(有时被称为强类型或静态类型的语言)包括Algol 60, C, C\#, Java, Haskell和ML。

\vest 在Scheme计算过程中创建的所有对象,包括过程和继续(continuation),都拥有无限的生存期(extent)。Scheme对象从不被销毁。Scheme的实现(通常!)不会耗尽空间的原因是,如果它们能证明某个对象无论如何都不会与未来的任何计算发生关联,它们就可以回收该对象占据的空间。其他允许多数对象拥有无限生存期的语言包括C\#, Java, Haskell, 大部分Lisp方言, ML, Python, Ruby和Smalltalk。

Scheme的实现必须支持严格尾递归。这一机制允许迭代计算在常量空间内执行,即便该迭代计算在语法上是用递归过程描述的。借助严格尾递归的实现,迭代就可以用普通的过程调用机制来表示,这样一来,专用的迭代结构就只剩下语法糖衣的用途了。


\vest Scheme是最早支持把过程在本质上当作对象的语言之一。过程可以动态创建,可以存储于数据结构中,可以作为过程的结果返回,等等。其他拥有这些特性的语言包括Common Lisp, Haskell, ML, Ruby和Smalltalk。

\vest 在大多数其他语言中只在幕后起作用的继续,在Scheme中也拥有“第一级”状态,这是Scheme的一个独树一帜的特征。第一级继续可用于实现大量不同的高级控制结构,如非局部退出(non-local exits)、回溯(backtracking)和协作程序(coroutine)等。

在Scheme中,过程的参数表达式会在过程获得控制权之前被求值,无论这个过程需不需要这个值。C, C\#, Common Lisp, Python, Ruby和Smalltalk是另外几个总在调用过程之前对参数表达式进行求值的语言。这和Haskell惰性求值(lazy-evaluation)的语义以及Algol 60按名求值(call-by-name)的语义不同,在这些语义中,参数表达式只有在过程需要这个值的时候才会被求值。

Scheme的算术模型被设计为尽量独立于计算机内数值的特定表示方式。此外,他
还区分\textit{精确}数和\textit{非精确}数对象:本质上,一个精确数对象精确地等价于一个数,一个非精确数对象是一个涉及到近似或其它误差的计算结果。

\section{基本类型}

Scheme程序操作\textit{对象(object)},有时也被成为\textit{值(value)}。Scheme对象被组织为叫做\textit{类型(type)}的值的集合。本节将给你Scheme语言十分重要的类型的一个概述。更多的类型将在后面章节进行描述。

\begin{note}
  由于Scheme采用隐式类型,所以本报告中术语\textit{类型}的使用与其它语言文本中本术语的使用不同,尤其是与那些显式语言中的不同。
\end{note}

\paragraph{布尔(Booleans)}

\mainindex{boolean 布尔}布尔类型是一个真值,可以是真或假。在Scheme中,对象“假”被写作`\schfalse{}。对象“真”被写作\schtrue{}。然而,在大部分需要一个真值的情况下,凡是不同于\schfalse{}的对象被看作真。

\paragraph{数值(Numbers)}

\mainindex{number 数值}Scheme广泛支持各类数值数据结构,包括任意精度的整数,有理数,复数和各种类型的非精确数。第~\ref{numbertypeschapter}章给了Scheme数值塔结构的概述。

\paragraph{字符(Characters)}

\mainindex{character 字符}Scheme字符多半等价于一个文本字符。更精确地说,它们同构与Unicode字符编码标准的\textit{标量值(scalar values)}。

\paragraph{字符串(Strings)}

\mainindex{string 字符串}字符串是确定长度字符的有限序列,因此它代表任意的Unicode文本。

\paragraph{符号(Symbols)}

\mainindex{symbol 符号}符号是一个表示为字符串的对象,即它的\textit{名字}。不同于字符串,两个名字拼写一样的符号永远无法区分。符号在许多应用中十分有用;比如:它们可以像其它语言中枚举值那样使用。

\paragraph{点对(Pairs)和表(lists)}

\mainindex{pair 点对}\mainindex{list 表}一个点对是两个元素的数据结构。点对最常用的用法是(逐一串联地)表示表,在这种表结构中,第一个元素(即“car”)表示表的第一个元素,第二个元素(即“cdr”)表示表的剩余部分。Scheme还有一个著名的空表,它是表中一串点对的最后一个cdr。

\paragraph{向量(Vectors)}

\mainindex{vector 向量}像表一样,向量是任意对象有限序列的线性数据结构。然而,表的元素循环地通过链式的点对进行访问,向量的元素通过整数索引进行标识。所以,比起表,向量更适合做元素的随机访问。

\paragraph{过程(Procedures)}

\mainindex{procedure 过程}在Scheme中过程是值。

\section{表达式(Expressions)}

Scheme代码中最重要的元素是\textit{表达式}。表达式可以被\textit{计算(evaluated)},产生一个\textit{值(value)}。(实际上,是任意数量的值—参见~\ref{multiplereturnvaluessection}节。)最基本的表达式就是字面的表达式:

\begin{scheme}
\schtrue{} \ev \schtrue
23 \ev 23%
\end{scheme}

这表明表达式\schtrue{}的结果就是\schtrue,也就是“真”的值,同时,表达式
{\cf 23}计算得出一个表示数字23的对象。

复合表达式通过在子表达式周围放置小括号来组合。第一个子表达式表示一个运算;剩余的子表达式是运算的操作数:
%
\begin{scheme}
(+ 23 42) \ev 65
(+ 14 (* 23 42)) \ev 980%
\end{scheme}
%

上面例子中的第一个例子,{\cf +}是内置的表示加法的运算的名字,{\cf 23}
和{\cf 42}是操作数。表达式{\cf (+ 23 42)}读作“23和42的和”。复合表达式可以被嵌套—第二个例子读作“14和23和42的积的和”。

正如这些例子所展示的,Scheme中的复合表达式使用相同的前缀表示法(prefix notation)\mainindex{prefix notation 前缀表示法}书写。所以,括号在表达这种结构的时候是必要的。所以,在数学表示和大部分语言中允许的“多余的”括号在Scheme中是不被允许的。

正如其它许多语言,空白符(包括换行符)在表达式中分隔子表达式的时候不是
很重要,它也可以用来表示结构。

\section{变量(variable)和绑定(binding)}

\mainindex{variable 变量}\mainindex{binding 绑定}\mainindex{identifier
标识符}Scheme允许标识符(identifier)代表一个包含值得位置。这些标识符叫做变量。在许多情况下,尤其是这个位置的值在创建之后再也不被修改的时候,认为变量直接代表这个值是非常有用的。

\begin{scheme}
(let ((x 23)
      (y 42))
  (+ x y)) \ev 65%
\end{scheme}

在这种情况下,以{\cf let}开头的表达式是一种绑定结构。跟在{\cf let}之后
的括号结构列出了和表达式一起的变量:和23一起的变量{\cf x},以及和42一
起的变量{\cf y}。{\cf let}表达式将{\cf x}绑定到{\cf 23},将{\cf y}绑定
到{\cf 42}。这些绑定在{\cf let}表达式的\textit{内部(body)}是有效的,如上面的{\cf (+ x y)},也仅仅在那是有效的。

\section{定义(definition)}

\index{definition,定义}{\cf let}表达式绑定的变量是\textit{局部的(local)},因为绑定只在{\cf let}的内部可见。Scheme同样允许创建标识符的顶层绑定,方法如下:

\begin{scheme}
(define x 23)
(define y 42)
(+ x y) \ev 65%
\end{scheme}

(这些实际上是在顶层程序或库的内部的“顶层”;参见下面的~\ref{librariesintrosection}节。)

开始的两个括号结构是\textit{定义},它们创造了顶层的绑定,绑定{\cf x}到23,绑定{\cf y}到42。定义不是表达式,它不能出现在所有需要一个表达式出现的地方。而且,定义没有值。

绑定遵从程序的词法结构:当存在相同名字的绑定的时候,一个变量参考离它最近的绑定,从它出现的地方开始,并且以从内而外的方式进行,如果在沿途没有发现局部绑定的话就使用顶层绑定:

\begin{scheme}
(define x 23)
(define y 42)
(let ((y 43))
  (+ x y)) \ev 66

(let ((y 43))
  (let ((y 44))
    (+ x y))) \ev 67%
\end{scheme}

\section{形式(Forms)}
尽管定义不是表达式,但是复合表达式和定义有相同的语法形式:

%
\begin{scheme}
(define x 23)
(* x 2)%
\end{scheme}
%
尽管第一行包含一个定义,第二行包含一个表达式,但它们的不同依赖于{\cf define}和{\cf *}的绑定。在纯粹的语法层次上,两者都是\textit{形式(forms)}\index{form,形式},\textit{形式}是Scheme程序一个语法部分的名字。特别地,23是形式{\cf (define x 23)}的一个\textit{子形式(subform)}\index{subform,子形式} 。

\section{过程(Procedures)}
\label{proceduressection}

\index{procedure,过程}定义也可以被用作定义一个过程:

\begin{scheme}
(define (f x)
  (+ x 42))

(f 23) \ev 65%
\end{scheme}

简单地说,过程是一个表达式对象的抽象。在上面这个例子中,第一个定义定义了一个叫{\cf f}的过程。(注意{\cf f x}周围的括号,这表示这是一个过程的定义。)表达式{\cf (f 23)}是一个\index{procedure call,过程调用}过程调用(procedure call),大概意思是“将{\cf x}绑定到23计算{\cf (+ x 42)}(过程的内部)的值”。


由于过程是一个对象,所以它们可以被传递给其它过程:
%
\begin{scheme}
(define (f x)
  (+ x 42))

(define (g p x)
  (p x))

(g f 23) \ev 65%
\end{scheme}

在上面这个例子中,{\cf g}的内部被视为{\cf p}绑定到{\cf f},{\cf x}绑定到23,这等价于{\cf (f 23)},最终结果是65。

实际上,Scheme许多预定义的操作不是通过语法提供的,而是值是过程的变量。比如,在其它语言中受到特殊对待的{\cf +}操作,在Scheme中只是一个普通的标识符,这个标识符绑定到一过程,这个过程将数字对象加起来。{\cf *}和许多其它的操作也是一样的:

\begin{scheme}
(define (h op x y)
  (op x y))

(h + 23 42) \ev 65
(h * 23 42) \ev 966%
\end{scheme}

过程定义不是定义过程的唯一方法。一个{\cf lambda}表达式可以在不指定一个名字的情况下以对象的形式生成一个过程:

\begin{scheme}
((lambda (x) (+ x 42)) 23) \ev 65%
\end{scheme}

这个例子中的整个表达式是一个过程调用;{\cf (lambda (x) (+ x 42))},相当于一个输入一个数字并加上42的过程。

\section{过程调用和语法关键词(syntactic keywords)}

尽管{\cf (+ 23 42)}, {\cf (f 23)}和{\cf ((lambda (x) (+ x 42)) 23)}都是过程调用的例子,但{\cf lambda}和{\cf let}表达式不是。这时因为尽管{\cf let}是一个标识符,但它不是一个变量,而是一个*语法关键词*\textit{语法关键词(syntactic keyword)}\index{syntactic keyword,语法关键词}。以一个语法关键词作为第一个子表达式的形式遵从关键词决定的特殊规则。{\cf define}标识符是一个定义,也是一个语法关键词。因此,定义也不是一个过程调用。

{\cf lambda}关键词的规则规定第一个子形式是一个参数的表,其余的子形式是过程的内部。在{\cf let}表达式中,第一个子形式是一个绑定规格的表,其余子形式构成表达式的内部。

通过在形式的第一个位置查找语法关键词的方法,过程调用通常可以和这些\textit{特殊形式(special forms)}\mainindex{special form,特殊形式} 区别开来:如果第一个位置不包含一个语法关键词,则这个表达式是一个过程调用。(所谓的\textit{标识符的宏(identifier macros)}允许创建另外的特殊形式,但相当不常见。)Scheme语法关键词的集合是非常小的,这通常使这项任务变得相当简单。可是,创建新的语法关键词的绑定也是有可能的,见下面的\ref{macrosintrosection}节。

\section{赋值(Assignment)}

Scheme变量通过定义或{\cf let}或{\cf lambda}表达式进行的绑定不是直接绑定在各自绑定时指定的对象上,而是包含这些对象的位置上。这些位置的内容随后可通过\textit{赋值(assignment)}\index{assignment,赋值}破坏性地改写:
%
\begin{scheme}
(let ((x 23))
  (set! x 42)
  x) \ev 42%
\end{scheme}

在这种情况下,{\cf let}的内部包括两个表达式,这两个表达式被顺序地求值,最后一个表达式的值会成为整个{\cf let}表达式的值。表达式{\cf (set! x 42)}是一个赋值,表示“用42代替{\cf x}所指向位置的对象”。因此,{\cf x}以前的值,23,被42代替了。

\section{衍生形式(Derived forms)和宏(macros)}
\label{macrosintrosection}

在本报告中,许多特殊的形式可被转换成更基本的特殊形式。比如,一个{\cf let}表达式可以被转换成一个过程调用和一个{\cf lambda}表达式。下面两个表达式是等价的:
%
\begin{scheme}
(let ((x 23)
      (y 42))
  (+ x y)) \ev 65

((lambda (x y) (+ x y)) 23 42) \lev 65%
\end{scheme}

像{\cf let}表达式这样的特殊形式叫做\textit{衍生形式(derived forms)}\index{derived form,衍生形式},因为它们的语义可以通过语法转换衍生自其它类型的形式。一些过程定义也是衍生形式。下面的两个定义是等价的:

\begin{scheme}
(define (f x)
  (+ x 42))

(define f
  (lambda (x)
    (+ x 42)))%
\end{scheme}

在Scheme中,一个程序通过绑定语法关键词到宏\index{macro,宏}以定义它自己的衍生形式是可能的:

\begin{scheme}
(define-syntax def
  (syntax-rules ()
    ((def f (p ...) body)
     (define (f p ...)
       body))))

(def f (x)
  (+ x 42))%
\end{scheme}

{\cf define-syntax}结构指定一个符合{\cf (def f (p ...) body)}模式的括号结构,其中{\cf f},{\cf p}和{\cf body}是模式变量,被转换成{\cf (define (f p ...) body)}。因此,这个例子中的{\cf def}形式将被转换成下面的语句:

\begin{scheme}
(define (f x)
  (+ x 42))%
\end{scheme}

创建新的语法关键词的能力使得Scheme异常灵活和负有表现力,允许许多内建在其它语言的特性在Scheme中以衍生形式出现。

\section{语法数据(Syntactic data)和数据值(datum values)}

Scheme对象的一个子集叫做\textit{数据值}\index{datum value,数据值}。这些包括布尔,数据对象,字符,符号,和字符串还有表和向量,它们的元素是数据。每一个数据值都可以以\textit{语法数据}\index{syntactic datum,语法数据}的形式被表现成字面形式,语法数据可以在不丢失信息的情况下导出和读入。一个数据值可以被表示成多个不同的语法数据。而且,每一个数据值可以被平凡地转换成一个一个字面表达式,这种转换通过在程序中加上一个{\cf\singlequote}在其对应的语法数据前:

\begin{scheme}
'23 \ev 23
'\schtrue{} \ev \schtrue{}
'foo \ev foo
'(1 2 3) \ev (1 2 3)
'\#(1 2 3) \ev \#(1 2 3)%
\end{scheme}

在数字对象和布尔中,前一个例子中的{\cf\singlequote}是不必要的。语法数据{\cf foo}表示一个名字是“foo”的符号,且{\cf 'foo}是一个符号作为其值对的字面表达式。{\cf (1 2 3)}是一个元素是1,2和3的表的语法数据,且{\cf '(1 2 3)}是一个值是表的字面表达式。同样,{\cf \#(1 2 3)}是一个元素是1,2和3的向量,且{\cf '\#(1 2 3)}是对应的字面量。

语法数据是Scheme形式的超集。因此,数据可以被用作以数据对象的形式表示Scheme形式。特别地,符号可以被用作表示标识符。

\begin{scheme}
'(+ 23 42) \ev (+ 23 42)
'(define (f x) (+ x 42)) \lev (define (f x) (+ x 42))%
\end{scheme}

这方便了操作Scheme源代码程序的编写,尤其是解释和程序转换。

\section{继续(Continuations)}

在Scheme表达式被求值的时候,总有一个\textit{继续(continuation)}\index{continuation,继续}在等待这个结果。继续表示计算的一整个(默认的)未来。比如,不正式地说,表达式
%
\begin{scheme}
(+ 1 3)%
\end{scheme}
%
中{\cf 3}的继续给它加上了1。一般地,这些普遍存在的继续是隐藏在背后的,程序员不需要对它们考虑太多。可是,偶尔,一个程序员需要显示地处理继续。过程{\cf call-with-current-continuation}(见\ref{call-with-current-continuation}节)允许Scheme程序员通过创建一个接管当前继续的过程来处理这些。过程{\cf call-with-current-continuation}传入一个过程,并以一个\textit{逃逸过程(escape procedure)}\index{escape procedure,逃逸过程}作为参数,立刻调用这个过程。随后,这个逃逸过程可以用一个参数调用,这个参数会成为{\cf call-with-current-continuation}的结果。也就是说,这个逃逸过程放弃了它自己的继续,恢复了{\cf call-with-current-continuation}调用的继续。

在下面这个例子中,一个代表加1到其参数的逃逸过程被绑定到{\cf escape}上,且然后以参数3被调用。{\cf escape}调用的继续被放弃,取而代之的是3被传递给加1这个继续:
%
\begin{scheme}
(+ 1 (call-with-current-continuation
       (lambda (escape)
         (+ 2 (escape 3))))) \lev 4%
\end{scheme}
%
一个逃逸过程有无限的生存期:它可以在它捕获的继续被调用之后被调用,且它可以被调用多次。这使得{\cf call-with-current-continuation}相对于特定的非本地跳转,如其它语言的异常,显得异常强大。

\section{库(Libraries)}
\label{librariesintrosection}

Scheme代码可以被组织在叫做\textit{库(libraries)}\index{library,库}的组件中。每个库包含定义和表达式。它可以从其它的库中导入定义,也可以向其它库中导出定义。

下面叫做{\cf (hello)}的库导出一个叫做{\cf hello-world}的定义,且导入基础库(base library)(见第\ref{baselibrarychapter}章)和简单I/O库(simple I/O library)(见库的第\extref{lib:simpleiosection}{Simple I/O}小节)。{\cf hello-world}导出的是一个在单独的行上显示{\cf Hello World}的过程。
%
\begin{scheme}
(library (hello)
  (export hello-world)
  (import (rnrs base)
          (rnrs io simple))
  (define (hello-world)
    (display "Hello World")
    (newline)))%
\end{scheme}

\section{顶层程序(Top-level programs)}

一个Scheme程序被一个\textit{顶层程序}\index{top-level program}调用。像库一样,一个顶层程序包括导入,定义和表达式,且指定一个执行的入口。因此,一个顶层程序通过它导入的库的传递闭包定义了一个Scheme程序。

下面的程序通过来自\rsixlibrary{programs}库(见库的第\extref{lib:programlibchapter}{Command-line access and exit values}章)的{\cf command-line}过程从命令行获得第一个参数。它然后用{\cf open-file-input-port}(见库的第\extref{lib:portsiosection}小节)打开一个文件,产生一个\textit{端口(port)},也就是作为一个数据源的文件的一个连接,并调用{\cf get-bytes-all}过程以获得文件的二进制数据。程序然后使用{\cf put-bytes}输出文件的内容到标准输出:

(译注:以下的示例程序已根据勘误表进行了修正。)
%
\begin{scheme}
\#!r6rs
(import (rnrs base)
        (rnrs io ports)
        (rnrs programs))
(put-bytes (standard-output-port)
           (call-with-port
               (open-file-input-port
                 (cadr (command-line)))
             get-bytes-all))%
\end{scheme}

%%% Local Variables:
%%% mode: latex
%%% TeX-master: "r6rs"
%%% End:
  \par
\chapter{Requirement levels} 
\label{requirementchapter}

The key words ``must'', ``must not'', ``should'',
``should not'', ``recommended'', ``may'', and ``optional'' in this
report are to be interpreted as described in RFC~2119~\cite{mustard}.
Specifically:

\begin{description}
\item[must]\mainindex{must} This word means that a statement is an absolute
  requirement of the specification.
\item[must not]\mainindex{must not} This phrase means that a statement is an absolute
  prohibition of the specification.
\item[should]\mainindex{should} This word, or the adjective ``recommended'', means that
  valid reasons may exist in particular circumstances to ignore a
  statement, but that the implications must be understood and weighed
  before choosing a different course.
\item[should not]\mainindex{should not} This phrase, or the phrase ``not recommended'', means
  that valid reasons may exist in particular circumstances when the
  behavior of a statement is acceptable, but that the implications
  should be understood and weighed before choosing the course described
  by the statement.
\item[may]\mainindex{may} This word, or the adjective ``optional'', means that an item
  is truly optional.
\end{description}

In particular, this report occasionally uses ``should'' to designate
circumstances that are outside the specification of this report, but
cannot be practically detected by an implementation; see
section~\ref{argumentcheckingsection}.  In such circumstances, a
particular implementation may allow the programmer to ignore the
recommendation of the report and even exhibit reasonable behavior.
However, as the report does not specify the behavior,
these programs may be unportable, that is, their execution might
produce different results on different implementations.

Moreover, this report occasionally uses the phrase ``not required'' to note the
absence of an absolute requirement.

%%% Local Variables: 
%%% mode: latex
%%% TeX-master: "r6rs"
%%% End: 
 \par
\chapter{Numbers}
\label{numbertypeschapter}
\mainindex{number}

This chapter describes Scheme's model for numbers.  It is important to
distinguish between the mathematical numbers, the Scheme objects that
attempt to model them, the machine representations used to implement
the numbers, and notations used to write numbers.  In this report, the
term \textit{number} refers to a mathematical number, and the term
\textit{number object} refers to a Scheme object representing a
number.  This report uses the types \type{complex}, \type{real},
\type{rational}, and \type{integer} to refer to both mathematical
numbers and number objects.  The \type{fixnum} and \type{flonum} types
refer to special subsets of the number objects, as determined by
common machine representations, as explained below.

\section{Numerical tower}
\label{numericaltypes}
\index{numerical types}

Numbers may be arranged into a tower of subsets in which each level
is a subset of the level above it:
\begin{tabbing}
\ \ \ \ \ \ \ \ \ \=\tupe{number} \\
\> \tupe{complex} \\
\> \tupe{real} \\
\> \tupe{rational} \\
\> \tupe{integer} 
\end{tabbing}

For example, 5 is an integer.  Therefore 5 is also a rational,
a real, and a complex.  The same is true of the number objects
that model 5.  

Number objects are organized as a corresponding tower of subtypes
defined by the predicates {\cf number?}, {\cf complex?}, {\cf real?},
{\cf rational?}, and {\cf integer?}; see section~\ref{number?}.
Integer number objects are also called \textit{integer
  objects}\mainindex{integer object}.

There is no simple relationship between the subset that contains a
number and its representation inside a computer.  For example, the
integer 5 may have several representations.  Scheme's numerical
operations treat number objects as abstract data, as independent of
their representation as possible.  Although an implementation of
Scheme may use many different representations for numbers, this should
not be apparent to a casual programmer writing simple programs.

\section{Exactness}
\label{exactly}

\mainindex{exactness}It is useful to distinguish between number objects
that are known to correspond to a number exactly, and those number
objects whose computation involved rounding or other errors.  For
example, index operations into data structures may need to know the index
exactly, as may some operations on polynomial coefficients in a symbolic algebra
system.  On the other hand, the results of measurements are inherently
inexact, and irrational numbers may be approximated by rational and
therefore inexact approximations.  In order to catch uses of numbers
known only inexactly where exact numbers are required, Scheme
explicitly distinguishes \defining{exact} from \defining{inexact} number objects.  This
distinction is orthogonal to the dimension of type.

A
number object is exact if it is the value of an exact numerical
literal or was derived from exact number objects using only exact
operations.  Exact number objects correspond to mathematical numbers
in the obvious way.

Conversely, a number object is inexact if it is the value of an
inexact numerical literal, or was derived from inexact number objects,
or was derived using inexact operations.  Thus inexactness is
contagious.

Exact arithmetic is reliable in the following sense: If exact number
objects are passed to any of the arithmetic procedures described in
section~\ref{propagationsection}, and an exact number object is
returned, then the result is mathematically correct.  This is
generally not true of computations involving inexact number objects
because approximate methods such as floating-point arithmetic may be
used, but it is the duty of each implementation to make the result as
close as practical to the mathematically ideal result.

\section{Fixnums and flonums}

A \defining{fixnum} is an exact integer object that lies
within a certain implementation-dependent subrange of the
exact integer objects. (Library section \extref{lib:fixnumssection}{Fixnums} describes a
library for computing with fixnums.)
Likewise, every implementation must
designate a subset of its inexact real number objects as \defining{flonum}s, and
to convert certain external representations into flonums.  
(Library section \extref{lib:flonumssection}{Flonums} describes a library for
computing with flonums.)  Note that
this does not imply that an implementation must use
floating-point representations.

\section{Implementation requirements}

\index{implementation restriction}\label{restrictions}

Implementations of Scheme must support number objects for
the entire tower of subtypes given in section~\ref{numericaltypes}.
Moreover, implementations must support exact integer 
objects and exact rational number objects of practically unlimited
size and precision, and to implement certain procedures (listed in
\ref{propagationsection}) so they always return exact results when
given exact arguments.  (``Practically unlimited'' means that the size
and precision of these numbers should only be limited by the size of
the available memory.)

Implementations may support only a limited range of inexact number
objects of any type, subject to the requirements of this section.  For
example, an implementation may limit the range of the inexact real
number objects (and therefore the range of inexact integer and
rational number objects) to the dynamic range of the flonum format.
Furthermore the gaps between the inexact integer objects and
rationals are likely to be very large in such an implementation as the
limits of this range are approached.

An implementation may use floating point and other approximate 
representation strategies for \tupe{inexact} numbers.
This report recommends, but does not require, that the IEEE 
floating-point standards be followed by implementations that use
floating-point representations, and that implementations using
other representations should match or exceed the precision achievable
using these floating-point standards~\cite{IEEE}.

In particular, implementations that use floating-point representations
must follow these rules: A floating-point result must be represented
with at least as much precision as is used to express any of the
inexact arguments to that operation.
Potentially inexact operations such as {\cf sqrt}, when
applied to exact arguments, should produce exact answers whenever possible
(for example the square root of an exact 4 ought to be an exact 2).
However, this is not required.
If, on the other hand, an exact number object is operated upon so as to produce an
inexact result (as by {\cf sqrt}), and if the result is represented in
floating point, then the most precise floating-point format available
must be used; but if the result is represented in some other way then
the representation must have at least as much precision as the most
precise floating-point format available.

It is the programmer's responsibility to avoid using inexact number
objects with magnitude or significand too large to be represented in
the implementation.

\section{Infinities and NaNs}

Some Scheme implementations, specifically those that follow the IEEE
floating-point standards, distinguish special number objects called
\mainindex{infinity}\defining{positive infinity}, \defining{negative
  infinity}, and \defining{NaN}.

Positive infinity is regarded as an inexact real (but not rational) number
object that represents an indeterminate number greater than the
numbers represented by all rational number objects.  Negative infinity
is regarded as an inexact real (but not rational) number object that represents
an indeterminate number less than the numbers represented by all
rational numbers.

A NaN is regarded as an inexact real (but not rational) number object so
indeterminate that it might represent any real number, including
positive or negative infinity, and might even be greater than positive
infinity or less than negative infinity.

\section{Distinguished -0.0}

\index{-0.0}
Some Scheme implementations, specifically those that follow the IEEE
floating-point standards, distinguish between number objects for $0.0$
and $-0.0$, i.e., positive and negative inexact zero.  This report
will sometimes specify the behavior of certain arithmetic operations
on these number objects.  These specifications are marked with ``if
$-0.0$ is distinguished'' or ``implementations that distinguish
$-0.0$''.

%%% Local Variables: 
%%% mode: latex
%%% TeX-master: "r6rs"
%%% End: 
 \par
% Lexical structure
\hyphenation{white-space}
%%\vfill\eject
\chapter{词汇语法(Lexical syntax)和数据语法(datum syntax)}
\label{readsyntaxchapter}

Scheme的语法被组织进三个层次:
%
\begin{enumerate}
\item \textit{词汇语法}描述一个程序文本怎样被分成语义(lexemes)的序列。
\item \textit{数据语法},按词汇语法制定,将\textit{句法数据(syntactic data)\mainindex{datum,数据}\mainindex{syntactic datum,句法数据}}的序列组织为语义的序列,其中一个句法数据递归地组织一个实体。
\item \textit{程序语法(program syntax)}按数据语法制定,实施进一步的句法数据含义的组织和分配。(本句已根据勘误表修改。)
\end{enumerate}
%
句法数据(也叫做\textit{外部表示(external representations)\index{external representation}})兼作为对象的一个符号,且Scheme的\rsixlibrary{io ports}库(见库的\extref{lib:portsiosection}{Port I/O}小节)提供了{\cf get-datum}和{\cf put-datum}过程用作句法数据的读写,在它们的文本表示和对应的对象之间进行转换。每一个句法数据表示一个对应的\defining{数据值}。一个句法数据可以使用{\cf quote}在一个程序中获得对应的数据值(见第\ref{quote}节)。

Scheme源程序包含句法数据和(不重要的)注释。Scheme源程序中的句法数据叫做\textit{形式(forms)}\mainindex{form,形式}。(嵌套在另一个形式中的形式叫做\defining{子形式(subform)}。)因此,Scheme的语法有如下性质,字符的任何序列是一个形式也是一个代表一些对象的句法数据。这可能导致混淆,因为在脱离上下文的情况下,一个给定的字符序列是用于对象的表示还是一个程序的文本可能不是很明显。它可能也是使Scheme强大的原因,因为它使得编写解释器或编译器这类把程序当作对象(或相反)的程序变得简单。

一个数据值可能有多个不同的外部表示。比如“{\tt \#e28.000}”和“{\tt\#x1c}”都是表示精确整数对象28的句法数据,句法数据“{\tt(8 13)}”, “{\tt( 08 13 )}”和“{\tt(8 .\ (13 .\ ()))}”都表示包含精确整数对象8和13的表。表示相同对象(从{\cf equal?}的意义来说;见\ref{equal?}小节)的句法数据作为程序的形式来说总是等价的。

因为句法数据和数据值之间密切的对应关系,所以当根据上下文精确含义显而易见时,本报告有时使用术语\defining{数据(datum)}来表示句法数据或数据值。

一个实现不允许以任何方式扩展词汇或数据语法,仅有一个例外:对任何不是{\cf r6rs}内建的\meta{identifier}(见\ref{identifiersection}小节),实现不需要把{\cf \sharpsign{}!\meta{identifier}}当作一个语法错误,且实现可以使用特定的{\cf \sharpsign{}!}前缀的标识符作为指示接下来的输入包含标准词汇和数据语法的扩展的标记。语法{\cf \sharpsign{}!r6rs}可被用作预示接下来的输入是用本报告中描述的词汇语法和数据语法写成的。或者,{\cf \sharpsign{}!r6rs}被当作注释对待,见\ref{whitespaceandcomments}小节。

\section{符号(Notation)}
\label{BNF}

Scheme的形式语法(formal syntax)被用扩展的BNF写成。非终结符(Non-terminals)使用尖角括号进行书写。非终结符名字的大小写是无关紧要的。

语法中的空格都是为了便于阅读而存在的。\meta{Empty}代表空字符串。

对BNF的以下扩展可以使描述更加简洁:\arbno{\meta{thing}}代表零个或更多的\meta{thing};\atleastone{\meta{thing}}代表至少一个\meta{thing}。

一些非终结符的名字表示相同名字的Unicode标量值:\meta{character tabulation} (U+0009),\meta{linefeed} (U+000A),\meta{line tabulation} (U+000B),\meta{form feed} (U+000C),\meta{carriage return} (U+000D),
\meta{space} (U+0020),\meta{next line} (U+0085),\meta{line
  separator} (U+2028)和\meta{paragraph separator} (U+2029)。

\section{词汇语法(Lexical syntax)}
\label{lexicalsyntaxsection}

词汇语法决定了怎样将字符的序列分隔成语义(lexemes)\index{lexeme,语义}的序列,省略不重要的部分如注释和空白。字符序列被假定是Unicode标准\cite{Unicode}的文本。词汇语法的一些语义,比如标识符,数字对象的表示,字符串等等,是数据语法中的句法数据,且因此代表对象。除了语法的形式解释(formal account),本节还描述了这些句法数据表示什么数据值。

注释中描述的词汇语法包含\meta{datum}的一个向前引用,这是数据语法的一部分。然而,作为注释,这些\meta{datum}在语法中并不起什么重要的作用。

除了布尔,数字对象以及用16进制表示的Unicode标量是不区分大小写的,其它情况下大小写是敏感的。比如,{\cf \#x1A}和{\cf \#X1a}是一样的。然而,标识符{\cf Foo}和标识符{\CF FOO}是有区别的。

\subsection{形式解释(Formal account)}
\label{lexicalgrammarsection}

\meta{Interlexeme space}可以出现在任意词位(lexeme)的两侧,但不允许出现在一个词位的中间。

\hyper{Identifier}, {\cf .}, \hyper{number}, \hyper{character}和\hyper{boolean}必须被一个\meta{delimiter}或输入的结尾终结。

下面的两个字符保留用作未来的语言扩展:{\tt \verb"{" \verb"}"}

(以下规则已根据勘误表修改。)

\begin{grammar}%
\meta{lexeme} \: \meta{identifier} \| \meta{boolean} \| \meta{number}\index{identifier}
\>  \| \meta{character} \| \meta{string}
\>  \| ( \| ) \| \openbracket{} \| \closedbracket{} \| \sharpsign( \| \sharpsign{}vu8( | \singlequote{} \| \backquote{} \| , \| ,@ \| {\bf.}
\>  \| \sharpsign\singlequote{} \| \sharpsign\backquote{} \| \sharpsign, \| \sharpsign,@
\meta{delimiter} \: ( \| ) \| \openbracket{} \| \closedbracket{} \| " \| ; \| \sharpsign{}
\>  \| \meta{whitespace}
\meta{whitespace} \: \meta{character tabulation}
\> \| \meta{linefeed} \| \meta{line tabulation} \| \meta{form feed}
\> \| \meta{carriage return} \| \meta{next line}
\> \| \meta{any character whose category is Zs, Zl, or Zp}
\meta{line ending} \: \meta{linefeed} \| \meta{carriage return}
\> \| \meta{carriage return} \meta{linefeed} \| \meta{next line}
\> \| \meta{carriage return} \meta{next line} \| \meta{line separator}
\meta{comment} \: ; \= $\langle$\rm all subsequent characters up to a
                    \>\ \rm \meta{line ending} or \meta{paragraph separator}$\rangle$\index{comment}
\qquad \= \| \meta{nested comment}
\> \| \#; \meta{interlexeme space} \meta{datum}
\> \| \#!r6rs
\meta{nested comment} \: \#| \= \meta{comment text}
\> \arbno{\meta{comment cont}} |\#
\meta{comment text} \: \= $\langle$\rm character sequence not containing
\>\ \rm {\tt \#|} or {\tt |\#}$\rangle$
\meta{comment cont} \: \meta{nested comment} \meta{comment text}
\meta{atmosphere} \: \meta{whitespace} \| \meta{comment}
\meta{interlexeme space} \: \arbno{\meta{atmosphere}}%
\end{grammar}

\label{extendedalphas}
\label{identifiersyntax}

% This is a kludge, but \multicolumn doesn't work in tabbing environments.
\setbox0\hbox{\cf\meta{variable} \goesto{} $\langle$}

\begin{grammar}%
\meta{identifier} \: \meta{initial} \arbno{\meta{subsequent}}
 \>  \| \meta{peculiar identifier}
\meta{initial} \: \meta{constituent} \| \meta{special initial}
 \> \| \meta{inline hex escape}
\meta{letter} \:  a \| b \| c \| ... \| z
\> \| A \| B \| C \| ... \| Z
\meta{constituent} \: \meta{letter}
 \> \| $\langle${\rm any character whose Unicode scalar value is greater than}
 \> \quad {\rm 127, and whose category is Lu, Ll, Lt, Lm, Lo, Mn,}
 \> \quad {\rm Nl, No, Pd, Pc, Po, Sc, Sm, Sk, So, or Co}$\rangle$
\meta{special initial} \: ! \| \$ \| \% \| \verb"&" \| * \| / \| : \| < \| =
 \>  \| > \| ? \| \verb"^" \| \verb"_" \| \verb"~"
\meta{subsequent} \: \meta{initial} \| \meta{digit}
 \>  \| \meta{any character whose category is Nd, Mc, or Me}
 \>  \| \meta{special subsequent}
\meta{digit} \: 0 \| 1 \| 2 \| 3 \| 4 \| 5 \| 6 \| 7 \| 8 \| 9
\meta{hex digit} \: \meta{digit}
 \> \| a \| A \| b \| B \| c \| C \| d \| D \| e \| E \| f \| F
\meta{special subsequent} \: + \| - \| .\ \| @
\meta{inline hex escape} \: \backwhack{}x\meta{hex scalar value};
\meta{hex scalar value} \: \atleastone{\meta{hex digit}}
\meta{peculiar identifier} \: + \| - \| ... \| -> \arbno{\meta{subsequent}}
\meta{boolean} \: \schtrue{} \| \#T \| \schfalse{} \| \#F
\meta{character} \: \#\backwhack{}\meta{any character}
 \>  \| \#\backwhack{}\meta{character name}
 \>  \| \#\backwhack{}x\meta{hex scalar value}
\meta{character name} \: nul \| alarm \| backspace \| tab
\> \| linefeed \| newline \| vtab \| page \| return
\> \| esc \| space \| delete
\meta{string} \: " \arbno{\meta{string element}} "
\meta{string element} \: \meta{any character other than \doublequote{} or \backwhack}
 \> \| \backwhack{}a \| \backwhack{}b \| \backwhack{}t \| \backwhack{}n \| \backwhack{}v \| \backwhack{}f \| \backwhack{}r
 \>  \| \backwhack\doublequote{} \| \backwhack\backwhack
 \>  \| \backwhack\arbno{\meta{intraline whitespace}}\meta{line ending}
 \>  \hspace*{4em}\arbno{\meta{intraline whitespace}}
 \>  \| \meta{inline hex escape}
\meta{intraline whitespace} \: \meta{character tabulation}
\> \| \meta{any character whose category is Zs}%
\end{grammar}

\meta{Hex scalar value}表示0到\sharpsign{}x10FFFF的Unicode标量值,这个范围要排除$\left[\sharpsign{}x\textrm{D800}, \sharpsign{}x\textrm{DFFF}\right]$。

% \linebreak:强制换行,与\newline的区别为\linebreak的当前行分散对齐。
\label{numbersyntax}%
让\hbox{$R = 2$,$8$,$10$}和$16$,然后重复下面的规则\meta{num $R$},\linebreak{}\meta{complex $R$},\meta{real $R$},\meta{ureal $R$},\meta{uinteger $R$}和\linebreak{}\meta{prefix $R$}。没有\meta{decimal $2$},\meta{decimal $8$}和\meta{decimal $16$}的规则,这意味着包含小数点和指数的数字表示必须使用十进制基数。

下面的规则中,大小写是不重要的。(本句根据勘误表添加。)

(以下规则已根据勘误表修改。)


\begin{grammar}%
\meta{number} \: \meta{num $2$} \| \meta{num $8$}
   \>  \| \meta{num $10$} \| \meta{num $16$}
\meta{num $R$} \: \meta{prefix $R$} \meta{complex $R$}
\meta{complex $R$} \: %
         \meta{real $R$} %
      \| \meta{real $R$} @ \meta{real $R$}
   \> \| \meta{real $R$} + \meta{ureal $R$} i %
      \| \meta{real $R$} - \meta{ureal $R$} i
   \> \| \meta{real $R$} + \meta{naninf} i %
      \| \meta{real $R$} - \meta{naninf} i
   \> \| \meta{real $R$} + i %
      \| \meta{real $R$} - i
   \> \| + \meta{ureal $R$} i %
      \| - \meta{ureal $R$} i
   \> \| + \meta{naninf} i %
      \| - \meta{naninf} i
   \> \| + i %
      \| - i
\meta{real $R$} \: \meta{sign} \meta{ureal $R$}
  \> \| + \meta{naninf} \| - \meta{naninf}
\meta{naninf} \: nan.0 \| inf.0
\meta{ureal $R$} \: %
         \meta{uinteger $R$}
   \> \| \meta{uinteger $R$} / \meta{uinteger $R$}
   \> \| \meta{decimal $R$} \meta{mantissa width}
\meta{decimal $10$} \: %
         \meta{uinteger $10$} \meta{suffix}
   \> \| . \atleastone{\meta{digit $10$}} \meta{suffix}
   \> \| \atleastone{\meta{digit $10$}} . \arbno{\meta{digit $10$}} \meta{suffix}
\meta{uinteger $R$} \: \atleastone{\meta{digit $R$}}
\meta{prefix $R$} \: %
         \meta{radix $R$} \meta{exactness}
   \> \| \meta{exactness} \meta{radix $R$}
\end{grammar}

\begin{grammar}%
\meta{suffix} \: \meta{empty}
   \> \| \meta{exponent marker} \meta{sign} \atleastone{\meta{digit $10$}}
\meta{exponent marker} \: e \| E \| s \| S \| f \| F
   \> \| d \| D \| l \| L
\meta{mantissa width} \: \meta{empty}
   \> \| | \atleastone{\meta{digit 10}}
\meta{sign} \: \meta{empty}  \| + \|  -
\meta{exactness} \: \meta{empty}
   \> \| \#i\sharpindex{i} \| \#I \| \#e\sharpindex{e} \| \#E
\meta{radix 2} \: \#b\sharpindex{b} \| \#B
\meta{radix 8} \: \#o\sharpindex{o} \| \#O
\meta{radix 10} \: \meta{empty} \| \#d \| \#D
\meta{radix 16} \: \#x\sharpindex{x} \| \#X
\meta{digit 2} \: 0 \| 1
\meta{digit 8} \: 0 \| 1 \| 2 \| 3 \| 4 \| 5 \| 6 \| 7
\meta{digit 10} \: \meta{digit}
\meta{digit 16} \: \meta{hex digit}
\end{grammar}

\subsection{换行符}
\label{lineendings}

在Scheme单行注释(见\ref{whitespaceandcomments}小节)和字符串字面量中,换行符是重要的。在Scheme源代码中,在\meta{line ending}中的任意换行符都指示一个行的结束。此外,两字符的换行符\meta{carriage return} \meta{linefeed}和\meta{carriage return} \meta{next line}都仅表示一个单独的换行。

在一个字符串字面量中,一个之前没有{\cf\backwhack}的\hyper{line ending}表示一个换行字符(linefeed character),这个字符是Scheme中的标准换行符。

\subsection{空白和注释}
\label{whitespaceandcomments}

\defining{空白(Whitespace)}字符是空格,换行,回车,字符制表符,换页符,行制表符和其它种类是Zs,Zl或Zp的任意其它字符。空白字符用于提高可读性和必要地相互分隔词位。空白可以出现在两个词位中间,但不允许出现在一个词位的中间。空白字符也可以出现在一个字符串中,但此时空白是有意义的。

词汇语法包括一些注释形式。在所有的情况下,注释对Scheme是不可见的,除非它们作为分隔符,所以,比如,一个注释不能出现在标识符或一个数字对象的表示的中间。

一个分号({\tt;})指示一个行注释的开始。\mainindex{comment,注释}\mainschindex{;}注释一直延续到分号出现的那一行的结尾。

另一个指示注释的方法是在\hyper{datum}(参见\ref{datumsyntax}节)前加一个前缀 {\tt \#;},可选的在\hyper{datum}前加上\meta{interlexeme space}。注释包括注释前缀{\tt \#;}和\hyper{datum}。这个符号用作“注释掉”代码段。

块注释可用正确嵌套的{\tt \#|}\index{#"|@\texttt{\sharpsign\verticalbar}}\index{"|#@\texttt{\verticalbar\sharpsign}}和{\tt |\#}指示。

\begin{scheme}
\#|
   FACT过程计算一个非负数的阶乘。
|\#
(define fact
  (lambda (n)
    ;; 基础条件(base case)
    (if (= n 0)
        \#;(= n 1)
        1       ; *的单位元(identity)
        (* n (fact (- n 1))))))%
\end{scheme}

词位{\cf \sharpsign{}!r6rs},表示接下来的输入是用本报告中描述的词汇语法和数据语法写成的,另外,它也被当作注释对待。

\subsection{标识符(Identifiers)}
\label{identifiersection}

其他程序设计语言认可的大多数标识符(identifiers)\mainindex{标识符,identifier}也能被Scheme接受。通常,第一个字符不是任何数值的字母、数字和“扩展字符”序列就是一个标识符。此外,\ide{+},\ide{-}和\ide{...}都是标识符,以两个字符序列\ide{->}开始的字母、数字和“扩展字符”序列也是。这里有一些标识符的例子:

\begin{scheme}
lambda         q                soup
list->vector   {+}                V17a
<=             a34kTMNs         ->-
the-word-recursion-has-many-meanings%
\end{scheme}

扩展字符可以像字母那样用于标识符内。以下是扩展字符:

\begin{scheme}
!\ \$ \% \verb"&" * + - . / :\ < = > ? @ \verb"^" \verb"_" \verb"~" %
\end{scheme}

此外,所有Unicode标量值大于127且类型属于Lu, Ll, Lt, Lm, Lo, Mn, Mc, Me, Nd, Nl, No, Pd, Pc, Po, Sc, Sm, Sk, So或Co的字符都可以被用在标识符中。当通过一个\meta{inline hex escape}表示的时候,任何字符都可以用在标识符中。比如,标识符\verb|H\x65;llo|和标识符\verb|Hello|是一样的,标识符\verb|\x3BB;|和标识符$\lambda$是一样的。

在Scheme程序中,任意标识符可作为一个变量\index{variable,变量}或一个语法关键词(syntactic keyword)\index{syntactic keyword,语法关键词}(见\index{variable}和\ref{macrosection}节)。任何标识符也可以作为一个句法数据,在这种情况下,它表示一个\textit{符号(symbol)}\index{symbol,符号}(见\ref{symbolsection}小节)。

\subsection{布尔(Booleans)}

标准布尔对象真和假有外部表示\schtrue{}和\schfalse.\sharpindex{t}\sharpindex{f}

\subsection{字符(Characters)}

字符可以用符号\sharpsign\backwhack\hyper{character}\index{#\@\texttt{\sharpsign\backwhack}}或\sharpsign\backwhack\hyper{character name}或\linebreak{}\sharpsign\backwhack{}x\meta{hex scalar value}表示。

比如:

\texonly
\newcommand{\extab}{\>}
\begin{tabbing}
{\cf\#\backwhack{}x0000000000}\=\kill
\endtexonly
\htmlonly
\newcommand{\extab}{&}
\begin{tabular}{ll}
\endhtmlonly
{\cf\#\backwhack{}a}          \extab \textrm{小写字母a}\\
{\cf\#\backwhack{}A}          \extab \textrm{大写字母A}\\
{\cf\#\backwhack{}(}          \extab \textrm{左小括号}\\
{\cf\#\backwhack{}}           \extab \textrm{空格}\\
{\cf\#\backwhack{}nul}        \extab \textrm{U+0000}\\
{\cf\#\backwhack{}alarm}      \extab \textrm{U+0007}\\
{\cf\#\backwhack{}backspace}  \extab \textrm{U+0008}\\
{\cf\#\backwhack{}tab}        \extab \textrm{U+0009}\\
{\cf\#\backwhack{}linefeed}   \extab \textrm{U+000A}\\
{\cf\#\backwhack{}newline}   \extab \textrm{U+000A}\\
{\cf\#\backwhack{}vtab}       \extab \textrm{U+000B}\\
{\cf\#\backwhack{}page}       \extab \textrm{U+000C}\\
{\cf\#\backwhack{}return}     \extab \textrm{U+000D}\\
{\cf\#\backwhack{}esc}        \extab \textrm{U+001B}\\
{\cf\#\backwhack{}space}      \extab \textrm{U+0020}\\
 \extab 表示一个空格时优先使用这种方法\\
{\cf\#\backwhack{}delete}     \extab \textrm{U+007F}\\[1ex]
{\cf\#\backwhack{}xFF}        \extab \textrm{U+00FF}\\
{\cf\#\backwhack{}x03BB}      \extab \textrm{U+03BB}\\
{\cf\#\backwhack{}x00006587}  \extab \textrm{U+6587}\\
{\cf\#\backwhack{}\(\lambda\)} \extab \textrm{U+03BB}\\[1ex]
{\cf\#\backwhack{}x0001z}     \extab \exception{\&lexical}\\
{\cf\#\backwhack{}\(\lambda\)x}         \extab \exception{\&lexical}\\
{\cf\#\backwhack{}alarmx}     \extab \exception{\&lexical}\\
{\cf\#\backwhack{}alarm x}    \extab \textrm{U+0007}\\
 \extab 跟着{\cf{}x}\\
{\cf\#\backwhack{}Alarm}      \extab \exception{\&lexical}\\
{\cf\#\backwhack{}alert}      \extab \exception{\&lexical}\\
{\cf\#\backwhack{}xA}         \extab \textrm{U+000A}\\
{\cf\#\backwhack{}xFF}        \extab \textrm{U+00FF}\\
{\cf\#\backwhack{}xff}        \extab \textrm{U+00FF}\\
{\cf\#\backwhack{}x ff}       \extab \textrm{U+0078}\\
 \extab 跟着另外一个数据,{\cf{}ff}\\
{\cf\#\backwhack{}x(ff)}      \extab \textrm{U+0078}\\
 \extab 跟着另外一个数据,\\
 \extab 一个被括号括着的{\cf{}ff}\\
{\cf\#\backwhack{}(x)}        \extab \exception{\&lexical}\\
{\cf\#\backwhack{}(x}         \extab \exception{\&lexical}\\
{\cf\#\backwhack{}((x)}       \extab \textrm{U+0028}\\
 \extab 跟着另外一个数据,\\
 \extab 在括号内的{\cf{}x}\\
{\cf\#\backwhack{}x00110000}  \extab \exception{\&lexical}\\
 \extab 超出范围\\
{\cf\#\backwhack{}x000000001} \extab \textrm{U+0001}  \\
{\cf\#\backwhack{}xD800}      \extab \exception{\&lexical}\\
 \extab 在被排除的范围
\htmlonly
\end{tabular}
\endhtmlonly
\texonly
\end{tabbing}
\endtexonly

(记号\exception{\&lexical}表示有问题的行违反了词汇语法。)

在\sharpsign\backwhack\hyper{character}和\sharpsign\backwhack\hyper{character name}中,大小写是敏感的,但在{\cf\sharpsign\backwhack{}x}\meta{hex scalar value}中\meta{hex scalar value}的大小写是不敏感的。(上句已根据勘误表进行修改。)一个\meta{character}必须跟着一个\meta{delimiter}或输入的结束。此规则解决了关于命名字符的各种歧义,比如,要求字符序列“{\tt\sharpsign\backwhack space}”被解释为一个空白字符而不是字符“{\tt\sharpsign\backwhack s}”和跟着的标识符“{\tt pace}”。

\begin{note}
  由于反向兼容的原因我们保留符号{\cf\sharpsign\backwhack{}newline}。它已经过时了;应使用{\cf\sharpsign\backwhack{}linefeed}代替。
\end{note}

\subsection{字符串(Strings)}

\vest 字符串使用被双引号({\cf "})括起来的字符序列表示。在字符串字面量中,各种转义序列(escape sequences)\mainindex{escape sequence,转义序列}被用来表示字符,而不是字符自己。转义序列总是以一个反斜杠(\backwhack{})开始:

\begin{itemize}
\item{\cf\backwhack{}a} : 响铃(alarm), U+0007
\item{\cf\backwhack{}b} : 退格(backspace), U+0008
\item{\cf\backwhack{}t} : 制表(character tabulation), U+0009
\item{\cf\backwhack{}n} : 换行(linefeed), U+000A
\item{\cf\backwhack{}v} : 行制表(line tabulation), U+000B
\item{\cf\backwhack{}f} : 换页(formfeed), U+000C
\item{\cf\backwhack{}r} : 回车(return), U+000D
\item{\cf\backwhack{}}\verb|"| : 双引号(doublequote), U+0022
\item{\cf\backwhack{}\backwhack{}} : 反斜杠(backslash), U+005C
\item{\cf\backwhack{}}\hyper{intraline whitespace}\hyper{line ending}\\\hspace*{2em}\hyper{intraline whitespace} : 空
\item{\cf\backwhack{}x\meta{hex scalar value};} : 指定字符(注意结尾的分号)。
\end{itemize}

这些转义序列是大小写敏感的,除了\meta{hex scalar value}中的字母数字(alphabetic digits)可以是大写也可以是小写。

字符串中反斜杠后的任意其它字符都是违反语法的。除了行结尾外,任意在转义序列外且不是一个双引号的字符在字符串字面量中代表它自己。比如,但字符字符串字面量{\tt "$\lambda$"}(双引号,一个小写的lambda,双引号)和{\tt "\backwhack{}x03bb;"}表示一样的字符串。一个前面不是反斜杠的行结尾表示一个换行字符。

例如:

\texonly
\begin{tabbing}
{\cf "\backwhack{}x0000000000;"} \=\kill
\endtexonly
\htmlonly
\begin{tabular}{ll}
\endhtmlonly
{\cf "abc"} \extab  \textrm{U+0061, U+0062, U+0063}\\
{\cf "\backwhack{}x41;bc"} \extab  {\cf "Abc"} ; \textrm{U+0041, U+0062, U+0063}\\
{\cf "\backwhack{}x41; bc"} \extab {\cf "A bc"}\\
 \extab U+0041, U+0020, U+0062, U+0063\\
{\cf "\backwhack{}x41bc;"} \extab  \textrm{U+41BC}\\
{\cf "\backwhack{}x41"} \extab \exception{\&lexical}\\
{\cf "\backwhack{}x;"} \extab \exception{\&lexical}\\
{\cf "\backwhack{}x41bx;"} \extab \exception{\&lexical}\\
{\cf "\backwhack{}x00000041;"} \extab  {\cf "A"} ; \textrm{U+0041}\\
{\cf "\backwhack{}x0010FFFF;"} \extab \textrm{U+10FFFF}\\
{\cf "\backwhack{}x00110000;"} \extab  \exception{\&lexical}\\
 \extab 超出范围\\
{\cf "\backwhack{}x000000001;"} \extab \textrm{U+0001}\\
{\cf "\backwhack{}xD800;"} \extab \exception{\&lexical}\\
 \extab 在被排除的范围\\
{\cf "A}\\
{\cf bc"} \extab \textrm{U+0041, U+000A, U+0062, U+0063}\\
 \extab 如果{\cf{}A}后面没有空格
\htmlonly
\end{tabular}
\endhtmlonly
\texonly
\end{tabbing}
\endtexonly

\subsection{数字(Numbers)}
\label{numbernotations}

数字对象外部表示的语法在形式语法的\meta{number}规则中被正式描述。在数字对象的外部表示中,大小写是不重要的。

数字对象的表示可以通过特定的基数前缀被写作二进制,八进制,十进制和十六进制。基数前缀是{\cf \#b}\sharpindex{b}(二进制),{\cf \#o}\sharpindex{o}(八进制),{\cf \#d}\sharpindex{d}(十进制)和{\cf \#x}\sharpindex{x}(十六进制)。在没有基数前缀时,一个数字对象的表示被假设是十进制的。

一个数字对象的表示可通过前缀被指定为精确的货非精确的。前缀{\cf \#e}\sharpindex{e}表示精确的,前缀{\cf \#i}\sharpindex{i}表示非精确的。如果使用基数前缀的话,精确性前缀可使用在其之前或之后。如果一个数字对象的表示没有精确性前缀,则在下列情况是非精确的,包含一个小数点,指数,或一个非空的尾数宽度(mantissa width),否则它是精确的。

在一个非精确数可以有不同精度的系统中,指定一个常数的精度可能是有用的。如果这样的话,数字对象的表示可以用一个指示非精确数预期精度的指数标记写成。字母{\cf s}, {\cf f}, {\cf d}和{\cf l}分别表示使用\var{short},\var{single},\var{double}和\var{long}精度。(当内部非精确表示少于四种时,这四个精度定义被映射到当前可用的定义。例如,只有两种内部表示的实现可以将short和single映射为一种精度,将long和double映射为一种)。另外,指数标记{\cf e}指明了Scheme实现的缺省精度。缺省精度应达到或超过\var{double}的精度,但Scheme实现也许会希望用户可设置此缺省精度。

\begin{scheme}
3.1415926535898F0
       {\rm{}舍入到single, 大概是} 3.141593
0.6L0
       {\rm{}扩展到long, 大概是} .600000000000000%
\end{scheme}

一个有非空尾数宽度数字对象的表示,{\cf \var{x}|\var{p}},代表有\var{p}位有效数字的\var{x}的最好的二进制浮点数近似。比如,{\cf 1.1|53}是使用IEEE double精度的1.1最好的近似的表示。如果\var{x}是一个不包含竖线的非精确实数对象的外部表示,那么它的数值应该被认为有53或更多的尾数宽度。

如果实际可行的话,实数对象使用二进制浮点表示的实现应该用\var{p}位精度表示{\cf \var{x}|\var{p}},或者如果实际不行的话,应该使用大于\var{p}位精度的实现,或者如果以上两个都不行的话,应该使用实现中最大的精度。

\begin{note}
有效数字的精度不应该和用作表示有效数字的位的数量相混淆。在IEEE浮点标准中,比如,有效数字的最高有效位暗示在single和double精度之间但明确使用扩展精度。不管那一位是不是明确指定的都不影响数学精度。使用二进制浮点的实现,默认的精度可通过调用下面的过程计算:

\begin{scheme}
(define (precision)
  (do ((n 0 (+ n 1))
       (x 1.0 (/ x 2.0)))
    ((= 1.0 (+ 1.0 x)) n)))
\end{scheme}
\end{note}

\begin{note}
当优先使用的浮点表示是IEEE double精度时,{\cf |\var{p}}后缀不应该总是被省略:非规范化浮点数在精度上有所松懈,所以它们的外部表示应该加上一个使用有效数字真实宽度的{\cf |\var{p}}后缀。
\end{note}

字面量{\cf +inf.0}和{\cf -inf.0}分别表示正无穷大和负无穷大。字面量{\cf +nan.0}表示非数,它是{\cf (/ 0.0 0.0)}的结果,且同样也可以表示其它非数。字面量{\cf -nan.0}也表示一个非数。(上句根据勘误表添加。)

如果\var{x}是一个非精确实数对象的外部表示且不包括竖线且不包括除了{\cf e}之外的指数标记,这个非精确实数对象表示一个浮点数(见库的第\extref{lib:flonumssection}{Flonums}小节)。其它非精确实数对象外部表示的一些或所有也可以表示浮点数,但这不是本报告要求的。

\section{数据语法(Datum syntax)}
\label{datumsyntaxsection}

数据语法按照词汇语法中定义的\meta{lexeme}的序列描述句法数据(syntactic data)\mainindex{syntactic datum,句法数据}的语法。

句法数据包括本报告前面章节描述的语义数据以及以下用于组织复合数据的结构:
%
\begin{itemize}
\item 点对和表,被\verb|( )|或\verb|[ ]| (见\ref{pairlistsyntax}小节)括起来
\item 向量(见\ref{vectorsyntax}小节)
\item 字节向量(bytevectors)(见\ref{bytevectorsyntax}小节)
\end{itemize}

\subsection{形式解释(Formal account)}
\label{datumsyntax}

下面的语法按照定义在\ref{lexicalsyntaxsection}节的语法的各种语义描述句法数据的语法:

\begin{grammar}%
\meta{datum} \: \meta{lexeme datum}
\>  \| \meta{compound datum}
\meta{lexeme datum} \: \meta{boolean} \| \meta{number}
\>  \| \meta{character} \| \meta{string} \|  \meta{symbol}
\meta{symbol} \: \meta{identifier}
\meta{compound datum} \: \meta{list} \| \meta{vector} \| \meta{bytevector}
\meta{list} \: (\arbno{\meta{datum}}) \| [\arbno{\meta{datum}}]
\>    \| (\atleastone{\meta{datum}} .\ \meta{datum}) \| [\atleastone{\meta{datum}} .\ \meta{datum}]
\>    \| \meta{abbreviation}
\meta{abbreviation} \: \meta{abbrev prefix} \meta{datum}
\meta{abbrev prefix} \: ' \| ` \| , \| ,@
\>    \| \#' | \#` | \#, | \#,@
\meta{vector} \: \#(\arbno{\meta{datum}})
\meta{bytevector} \: \#vu8(\arbno{\meta{u8}})
\meta{u8} \: $\langle${\rm any \meta{number} representing an exact}
 \>\>\quad\quad {\rm integer in $\{0, \ldots, 255\}$}$\rangle$%
\end{grammar}

\subsection{点对(Pairs)和表(lists)}
\label{pairlistsyntax}

表示值对和值的列表(见\ref{listsection}节)的列表和点对数据使用小括号和中括号表示。规则\meta{list}中匹配的中括号对等价于匹配的小括号对。

Scheme点对的句法数据最常用的符号是“点”符号\hbox{\cf (\hyperi{datum} .\ \hyperii{datum})},其中\hyperi{datum}是car区域的值的表示,\hyperii{datum}是cdr区域的值的表示。比如{\cf (4 .\ 5)}是一个car是4,cdr是5的点对。

表可以使用一个改进的记号:表中的元素被简单地括进小括号中并用空格分隔。空表(empty list)\index{empty list,空表}用 {\tt()}表示。比如:

\begin{scheme}
(a b c d e)%
\end{scheme}

和

\begin{scheme}
(a . (b . (c . (d . (e . ())))))%
\end{scheme}

作为符号的表是等价的表示。

一个通用的规则是,如果一个点跟着一个左小括号,则在外部表示中可以省略这个点,左小括号和对应的右小括号。

符号序列“{\cf (4 .\ 5)}”是一个点对的外部表示,而不是一个计算结果为点对的表达式。类似的,符号序列“{\tt(+ 2 6)}”{\em{}不}是整数8的外部表示,虽然它{\em{}是}一个计算结果是整数8的表达式(在\rsixlibrary{base}库的语言中);当然,它是一个句法数据,这个句法数据表示一个三个元素的表,这个表的元素是符号{\tt +},整数2和6。

\subsection{向量(Vectors)}
\label{vectorsyntax}

表示向量对象(见\ref{vectorsection}小节)的向量数据使用{\tt\#(\hyper{datum} \dotsfoo)}表示。比如:一个长度是3,且0号元素位置是数字对象零,1号元素位置是表{\cf (2 2 2 2)},2号元素位置是字符串{\cf "Anna"},的向量可以如下表示:

\begin{scheme}
\#(0 (2 2 2 2) "Anna")%
\end{scheme}

这是一个向量的外部表示,而不是一个计算结果为向量的表达式。

\subsection{字节向量(Bytevectors)}
\label{bytevectorsyntax}

表示字节向量(见库的第\extref{lib:bytevectorschapter}{Bytevectors}章)的向量数据使用符号{\tt\#vu8(\hyper{u8} \dotsfoo)}表示,其中\hyper{u8}表示字节向量的八位字节(octets)。比如:一个长度是3且包含八位字节2,24和123的字节向量可以如下表示:

\begin{scheme}
\#vu8(2 24 123)%
\end{scheme}

这是一个字节向量的外部表示,也是一个计算结果为字节向量的表达式。

\subsection{Abbreviations}\unsection
\label{abbreviationsection}

\begin{entry}{%
\pproto{\singlequote\hyper{datum}}{}
\pproto{\backquote\hyper{datum}}{}
\pproto{,\hyper{datum}}{}
\pproto{,\atsign\hyper{datum}}{}
\pproto{\#'\hyper{datum}}{}
\pproto{\#\backquote\hyper{datum}}{}
\pproto{\#,\hyper{datum}}{}
\pproto{\#,@\hyper{datum}}{}
}

上面的每一个都是一个缩写:
\\\quad\schindex{'}\singlequote\hyper{datum}
是{\cf (quote \hyper{datum})}的缩写,
\\\quad\schindex{`}\backquote\hyper{datum}
是{\cf (quasiquote \hyper{datum})}的缩写,
\\\quad\schindex{,}{\cf,}\hyper{datum}
是{\cf (unquote \hyper{datum})}的缩写,
\\\quad\index{,@\texttt{,\atsign}}{\cf,}\atsign\hyper{datum}
是{\cf (unquote-splicing \hyper{datum})}的缩写,
\\\quad\sharpindex{'}{\cf\#'}\hyper{datum}
是{\cf (syntax \hyper{datum})}的缩写,
\\\quad\sharpindex{`}{\cf\#`}\hyper{datum}
是{\cf (quasisyntax \hyper{datum})}的缩写,
\\\quad\sharpindex{,}{\cf\#,}\hyper{datum}
是{\cf (unsyntax \hyper{datum})}的缩写,且
\\\quad\index{#,@\texttt{\#,\atsign}}{\cf\#,@}\hyper{datum}
是{\cf (unsyntax-splicing \hyper{datum})}的缩写。
\end{entry}

%%% Local Variables:
%%% mode: latex
%%% TeX-master: "r6rs"
%%% End:
     \par
%\vfill\eject
\chapter{Semantic concepts}
\label{basicchapter}

\section{Programs and libraries}

A Scheme program consists of a \textit{top-level program\index{top-level program}}
together with a set of \textit{libraries\index{library}}, each
of which defines a part of the program connected to the others through
explicitly specified exports and imports.  A library consists of a set
of export and import specifications and a body, which consists of
definitions, and expressions.
A top-level program is similar to a library, but
has no export specifications.
Chapters~\ref{librarychapter} and \ref{programchapter}
describe the syntax and semantics of libraries and top-level programs,
respectively. 
Chapter~\ref{baselibrarychapter} describes a base
library that defines many of the constructs traditionally associated with
Scheme.
A separate report~\cite{R6RS-libraries}
describes the various \textit{standard libraries}\index{standard
  library} provided by a Scheme system.

The division between the base library and the other standard libraries is
based on use, not on construction.  In particular, some facilities
that are typically implemented as ``primitives'' by a compiler or the
run-time system rather than in terms of other standard procedures
 or syntactic forms are not part of the base library, but are defined in
separate libraries.  Examples include the fixnums and flonums libraries,
the exceptions and conditions libraries, and the libraries for
records.

\section{Variables, keywords, and regions}
\label{specialformsection}
\label{variablesection}

Within the body of a library or top-level program,
an identifier\index{identifier} may name a kind of syntax, or it may name
a location where a value can be stored.  An identifier that names a kind
of syntax is called a {\em keyword}\mainindex{keyword}, or {\em syntactic keyword}\mainindex{syntactic keyword},
and is said to be {\em bound} to that kind of syntax (or, in the case of a
syntactic abstraction, a {\em transformer} that translates the syntax into more
primitive forms; see section~\ref{macrosection}).  An identifier that names a
location is called a {\em variable}\mainindex{variable} and is said to be
{\em bound} to that location.  
At each point within a top-level program or a library, a specific, fixed set
of identifiers is bound.  The set of these identifiers, the set of \textit{visible
bindings}\mainindex{binding}, is
known as the {\em environment} in effect at that point.

Certain forms are used to create syntactic abstractions
and to bind keywords to transformers for those new syntactic abstractions, while other
forms create new locations and bind variables to those
locations.  Collectively, these forms are called {\em binding
  constructs}.\mainindex{binding construct}
Some binding constructs take the form of
\textit{definitions}\index{definition}, while others are
expressions.
With the exception of exported library bindings, a binding created
by a definition is visible only within the body in which the
definition appears, e.g., the body of a library, top-level program,
or {\cf lambda} expression.
Exported library bindings are also visible within the bodies of
the libraries and top-level programs that import them (see
chapter~\ref{librarychapter}).

Expressions that bind variables include the {\cf lambda},
{\cf let}, {\cf let*}, {\cf letrec}, {\cf letrec*}, {\cf let-values},
and {\cf let*-values} forms from the base library (see
sections~\ref{lambda}, \ref{letrec}).
Of these, {\cf lambda} is the most fundamental.
Variable definitions appearing within the body of 
such an expression, or within the bodies of a library or top-level
program, are treated as a set of
{\cf letrec*} bindings.
In addition, for library bodies, 
the variables exported from the library can be referenced by
importing libraries and top-level programs.

Expressions that bind keywords include the {\cf
  let-syntax} and {\cf letrec-syntax} forms (see
section~\ref{bindsyntax}).  A {\cf define} form (see section~\ref{define}) is a
definition that creates a variable binding (see 
section~\ref{defines}), and a {\cf define-syntax} form is
a definition that creates a keyword binding (see
section~\ref{syntaxdefinitionsection}).

\vest Scheme is a statically scoped language with
block structure.  To each place in a top-level program or library body where an identifier is bound 
there corresponds a \defining{region} of code within which
the binding is visible.  The region is determined by the particular
binding construct that establishes the binding; if the binding is
established by a {\cf lambda} expression, for example, then its region
is the entire {\cf lambda} expression.  Every mention of an identifier
refers to the binding of the identifier that establishes the
innermost of the regions containing the use.  If a use of an
identifier appears in a place where none of the surrounding expressions
contains a binding for the identifier, the use may refer to a
binding established by a definition or import at the top of the
enclosing library or top-level program
(see chapter~\ref{librarychapter}).
If there is no binding for the identifier,
it is said to be \defining{unbound}.\mainindex{bound}

\section{Exceptional situations}
\label{exceptionalsituationsection}

\mainindex{exceptional situation}A variety of exceptional situations
are distinguished in this report, among them violations of syntax,
violations of a procedure's specification, violations of
implementation restrictions, and exceptional situations in the
environment.  When an exceptional situation is detected by the
implementation, an \textit{exception is raised}\mainindex{raise},
which means that a special procedure called the \textit{current
  exception handler} is called.  A program can also raise an
exception, and override the current exception handler; see
library section~\extref{lib:exceptionssection}{Exceptions}.

When an exception is raised, an object is provided that
describes the nature of the exceptional situation.  The report uses
the condition system described in library section~\extref{lib:conditionssection}{Conditions} to
describe exceptional situations, classifying them by condition types.

Some exceptional situations allow continuing the program if the
exception handler takes appropriate action.  The corresponding
exceptions are called \textit{continuable}\index{continuable exception}.
For most of the exceptional situations described in this report,
portable programs cannot rely upon the exception being continuable
at the place where the situation was detected.
For those exceptions, the exception handler that is invoked by the
exception should not return.
In some cases, however, continuing is permissible, and the
handler may return.  See library section~\extref{lib:exceptionssection}{Exceptions}.

Implementations must raise an exception
when they are unable to continue correct execution
of a correct program due to some \defining{implementation restriction}.  For
example, an implementation that does not support infinities
must raise an exception with condition type
{\cf\&implementation-restriction} when it evaluates an expression
whose result would be an infinity.

Some possible implementation restrictions such as the lack of
representations for NaNs and infinities (see
section~\ref{infinitiesnanssection}) are anticipated by this report,
and implementations typically must raise an exception of the
appropriate condition type if they encounter such a situation.

This report uses the phrase ``an exception is raised'' synonymously
with ``an exception must be raised''.
This report uses the phrase ``an exception with condition type \var{t}''
to indicate that the object provided with the
exception is a condition object of the specified type.
The phrase ``a continuable exception is raised'' indicates an
exceptional situation that permits the exception handler to return.

\section{Argument checking}
\label{argumentcheckingsection}

\mainindex{argument checking}
Many procedures specified in this report or as part of a
standard library restrict the arguments they accept.
Typically, a procedure accepts only specific numbers and types of arguments.
Many syntactic forms similarly restrict the values to which one or
more of their subforms can evaluate.
These restrictions imply responsibilities\mainindex{responsibility} for
both the programmer and the implementation.
Specifically, the programmer is responsible for ensuring
that the values indeed adhere to the restrictions described
in the specification.  The implementation must check
that the restrictions in the specification are indeed met, to the
extent that it is reasonable, possible, and necessary to allow the
specified operation to complete successfully.  The implementation's
responsibilities are specified in more detail in
chapter~\ref{entryformatchapter} and throughout the report.

Note that it is not always possible for an implementation to completely check
the restrictions set forth in a specification.  For example, if an
operation is specified to accept a procedure with specific properties,
checking of these properties is undecidable in general.  Similarly,
some operations accept both lists and procedures that are
called by these operations.  Since lists can be mutated by the procedures
through the \rsixlibrary{mutable-pairs} library (see library
chapter~\extref{lib:pairmutationchapter}{Mutable pairs}), an argument that is a list
when the operation starts may become a non-list during the execution of the operation.
Also, the procedure might escape to a different continuation,
preventing the operation from performing more checks.
Requiring the operation to check that the argument is a list after
each call to such a procedure would be impractical.  Furthermore, some
operations that accept lists only need to traverse these lists
partially to perform their function; requiring the implementation to
traverse the remainder of the list to verify that all specified
restrictions have been met might
violate reasonable performance assumptions.  For these reasons, the
programmer's obligations may exceed the checking obligations of the
implementation.

When an implementation detects a violation of a restriction for an
argument, it must raise an exception with condition type
{\cf\&assertion} in a way consistent with the safety of execution as
described in section~\ref{safetysection}.

\section{Syntax violations}

The subforms of a special form usually need to obey certain syntactic
restrictions.  As forms may be subject to macro expansion, which may
not terminate, the question of whether they obey the specified
restrictions is undecidable in general.

When macro expansion terminates, however, implementations must detect
violations of the syntax.  A \defining{syntax violation} is an error
with respect to the syntax of library bodies, top-level bodies,
or the ``\exprtype'' entries in the
specification of the base library or the standard libraries.
Moreover, attempting to assign to an immutable variable (i.e., the
variables exported by a library; see
section~\ref{importsareimmutablesection}) is also
considered a syntax violation.

If a top-level or library form in a program is not syntactically
correct, then the implementation must raise an exception with
condition type {\cf\&syntax}, and execution of that top-level program
or library must not be allowed to begin.

\section{Safety}
\label{safetysection}

The standard libraries whose exports are described by this document
are said to be \defining{safe libraries}.  Libraries and top-level
programs that import only from safe libraries are also said to be safe.

As defined by this document, the Scheme programming language
is safe in the following sense:
The execution of a safe top-level program
cannot go so badly wrong as to crash or to continue to
execute while behaving in ways that are
inconsistent with the semantics described in this document,
unless an exception is raised.

Violations of an implementation restriction must raise an
exception with condition type {\cf\&implementation-\hp{}restriction},
as must all
violations and errors that would otherwise threaten system
integrity in ways that might result in execution that is
inconsistent with the semantics described in this document.

The above safety properties are guaranteed only for top-level programs
and libraries that are said to be safe.  In particular,
implementations may provide access to unsafe libraries in ways that
cannot guarantee safety.

\section{Boolean values}
\label{booleanvaluessection}

Although there is a separate boolean type, any Scheme value can be
used as a boolean value for the purpose of a conditional test.  In a
conditional test, all values count as true in such a test except for
\schfalse{}.  This report uses the word ``true'' to refer to any
Scheme value except \schfalse{}, and the word ``false'' to refer to
\schfalse{}. \mainindex{true} \mainindex{false}

\section{Multiple return values}
\label{multiplereturnvaluessection}

A Scheme expression can evaluate to an arbitrary finite number of
values.  These values are passed to the expression's continuation.

Not all continuations accept any number of values. For example, a continuation that
accepts the argument to a procedure call is guaranteed to accept
exactly one value.  The effect of passing some other number of values
to such a continuation is unspecified.  The {\cf call-with-values}
procedure
described in section~\ref{controlsection} makes it possible to create
continuations that accept specified numbers of return values.
If the number of
return values passed to a continuation created by a call to
{\cf call-with-values} is not accepted by its consumer
that was passed in that call, then an exception is raised.
A more complete description of the number of values accepted by
different continuations and the consequences of passing an unexpected
number of values is given in the description of the {\cf values}
procedure in section~\ref{values}.

A number of forms in the base library have sequences of expressions
as subforms that are evaluated sequentially, with the return values of
all but the last expression being discarded.  The continuations
discarding these values accept any number of values.

\section{Unspecified behavior}

\vest If an expression is said to ``return unspecified values'',
then the expression must evaluate without raising an exception, but
the values returned depend on the implementation; this report
explicitly does not say how many or what values should be returned.
Programmers should not rely on a specific number of return values or
the specific values themselves.
\mainindex{unspecified behavior}\mainindex{unspecified values}


\section{Storage model}
\label{storagemodel}

Variables and objects such as pairs, vectors, bytevectors, strings,
hashtables, and records implicitly
refer to locations\mainindex{location} or sequences of locations.  A string, for
example, contains as many locations as there are characters in the string. 
(These locations need not correspond to a full machine word.) A new value may be
stored into one of these locations using the {\tt string-set!} procedure, but
the string contains the same locations as before.

An object fetched from a location, by a variable reference or by
a procedure such as {\cf car}, {\cf vector-ref}, or {\cf string-ref}, is
equivalent in the sense of \ide{eqv?} % and \ide{eq?} ??
(section~\ref{equivalencesection})
to the object last stored in the location before the fetch.

Every location is marked to show whether it is in use.
No variable or object ever refers to a location that is not in use.
Whenever this report speaks of storage being allocated for a variable
or object, what is meant is that an appropriate number of locations are
chosen from the set of locations that are not in use, and the chosen
locations are marked to indicate that they are now in use before the variable
or object is made to refer to them.

It is desirable for constants\index{constant} (i.e. the values of
literal expressions) to reside in read-only memory.  To express this,
it is convenient to imagine that every object that refers to locations
is associated with a flag telling whether that object is
mutable\index{mutable} or immutable\index{immutable}.  Literal
constants, the strings returned by \ide{symbol->string}, records with
no mutable fields, and other values explicitly designated as immutable
are immutable objects, while all objects created by the other
procedures listed in this report are mutable.  An attempt to store a
new value into a location referred to by an immutable object
should raise an exception with condition type {\cf\&assertion}.


\section{Proper tail recursion}
\label{proper tail recursion}

Implementations of Scheme must be
{\em properly tail-recursive}\mainindex{proper tail recursion}.
Procedure calls that occur in certain syntactic
contexts called \textit{tail contexts}\index{tail context} 
are \textit{tail calls}\mainindex{tail call}.  A Scheme implementation is
properly tail-recursive if it supports an unbounded number of active
tail calls.  A call is {\em active} if the called procedure may still
return.  Note that this includes regular returns as well as returns
through continuations captured earlier by
{\cf call-with-current-continuation} that are later invoked.
In the absence of captured continuations, calls could
return at most once and the active calls would be those that had not
yet returned.
A formal definition of proper tail recursion can be found
in Clinger's paper~\cite{propertailrecursion}.  The rules for identifying tail calls
in constructs from the \rsixlibrary{base} library are described in
section~\ref{basetailcontextsection}.

\section{Dynamic extent and the dynamic environment}
\label{dynamicenvironmentsection}

For a procedure call, the time between when it is initiated and when
it returns is called its \defining{dynamic extent}.  In Scheme, {\cf
  call-with-current-continuation}
(section~\ref{call-with-current-continuation}) allows reentering a
dynamic extent after its procedure call has returned.  Thus, the
dynamic extent of a call may not be a single, connected time period.

Some operations described in the report acquire information in
addition to their explicit arguments from the \defining{dynamic
  environment}.  For example, {\cf call-\hp{}with-\hp{}current-\hp{}continuation}
accesses an implicit context established
by {\cf dynamic-wind} (section~\ref{dynamic-wind}), and the {\cf
  raise} procedure (library
section~\extref{lib:exceptionssection}{Exceptions}) accesses the
current exception handler.  The operations that modify the dynamic
environment do so dynamically, for the dynamic extent of a call to a
procedure like {\cf dynamic-wind} or {\cf with-exception-handler}.
When such a call returns, the previous dynamic environment is
restored.  The dynamic environment can be thought of as part of the
dynamic extent of a call.  Consequently, it is captured by {\cf
  call-with-current-continuation}, and restored by invoking the escape
procedure it creates.

%%% Local Variables: 
%%% mode: latex
%%% TeX-master: "r6rs"
%%% End: 
   \par
%\vfill\eject
\chapter{Entry format}
\label{entryformatchapter}

The chapters that describe bindings in the base library and the standard
libraries are organized
into entries.  Each entry describes one language feature or a group of
related features, where a feature is either a syntactic construct or a
built-in procedure.  An entry begins with one or more header lines of the form

\noindent\pproto{\var{template}}{\var{category}}\unpenalty

The \var{category} defines the kind of binding described by the entry,
typically either ``\exprtype'' or ``procedure''.
An entry may specify various restrictions on subforms or arguments.
For background on this, see section~\ref{argumentcheckingsection}.

\section{Syntax entries}

If \var{category} is ``\exprtype'', the entry describes a 
special syntactic construct, and the template gives the syntax of the
forms of the construct.  
The template is written in a notation similar to a right-hand
side of the BNF rules in chapter~\ref{readsyntaxchapter}, and describes
the set of forms equivalent to the forms matching the
template as syntactic data.  Some ``\exprtype'' entries carry a
suffix ({\cf expand}), specifying that the syntactic keyword of the
construct is exported with level
$1$.  Otherwise, the syntactic keyword is exported with level $0$; see
section~\ref{phasessection}.

Components of the form described by a template are designated
by syntactic variables, which are written using angle brackets, for
example, \hyper{expression}, \hyper{variable}.  Case is insignificant
in syntactic variables.  Syntactic variables
stand for other forms, or
sequences of them.  A syntactic variable may refer to a non-terminal
in the grammar for syntactic data (see section~\ref{datumsyntax}),
in which case only forms matching
that non-terminal are permissible in that position.
For example, \hyper{identifier} stands for a form which must be an
identifier.
Also,
\hyper{expression} stands for any form which is a
syntactically valid expression.  Other non-terminals that are used in
templates are defined as part of the specification.

The notation
\begin{tabbing}
\qquad \hyperi{thing} $\ldots$
\end{tabbing}
indicates zero or more occurrences of a \hyper{thing}, and
\begin{tabbing}
\qquad \hyperi{thing} \hyperii{thing} $\ldots$
\end{tabbing}
indicates one or more occurrences of a \hyper{thing}.

It is the programmer's responsibility to ensure that each component of
a form has the shape specified by a template.  Descriptions of syntax
may express other restrictions on the components of a form.
Typically, such a restriction is formulated as a phrase of the form
``\hyper{x} must be\mainindex{must be} a \ldots''.  Again, these
specify the programmer's responsibility.  It is the implementation's
responsibility to check that these restrictions are satisfied, as long
as the macro transformers involved in expanding the form terminate.
If the implementation detects that a component does not meet the
restriction, an exception with condition type {\cf\&syntax} is raised.

\section{Procedure entries}

If \var{category} is ``procedure'', then the entry describes a procedure, and
the header line gives a template for a call to the procedure.  Parameter
names in the template are \var{italicized}.  Thus the header line

\noindent\pproto{(vector-ref \var{vector} \var{k})}{procedure}\unpenalty

indicates that the built-in procedure {\tt vector-ref} takes
two arguments, a vector \var{vector} and an exact non-negative integer
object \var{k} (see below).  The header lines

\noindent%
\pproto{(make-vector \var{k})}{procedure}
\pproto{(make-vector \var{k} \var{fill})}{procedure}\unpenalty

indicate that the {\tt make-vector} procedure takes
either one or two arguments.  The parameter names are
case-insensitive: \var{Vector} is the same as \var{vector}.

As with syntax templates, an ellipsis \dotsfoo{} at the end of a header
line, as in

\noindent\pproto{(= \vari{z} \varii{z} \variii{z} \dotsfoo)}{procedure}\unpenalty

indicates that the procedure takes arbitrarily many arguments of the
same type as specified for the last parameter name.  In this case,
{\cf =} accepts two or more arguments that must all be complex
number objects.

\label{typeconventions}
A procedure that detects an argument that it is not specified to
handle must raise an exception with condition type
{\cf\&assertion}.  Also, the argument specifications are exhaustive: if the
number of arguments provided in a procedure call does not match 
any number of arguments accepted by the procedure, an exception with
condition type {\cf\&assertion} must be raised.

For succinctness, the report follows the convention
that if a parameter name is also the name of a type, then the corresponding argument must be of the named type.
For example, the header line for {\tt vector-ref} given above dictates that the
first argument to {\tt vector-ref} must be a vector.  The following naming
conventions imply type restrictions:

\texonly\begin{center}\endtexonly
  \begin{tabular}{ll}
    \var{obj}&any object\\
    \var{z}&complex number object\\
    \var{x}&real number object\\
    \var{y}&real number object\\
    \var{q}&rational number object\\
    \var{n}&integer object\\
    \var{k}&exact non-negative integer object\\
    \var{bool}&boolean (\schfalse{} or \schtrue{})\\
    \var{octet}&exact integer object in $\{0, \ldots, 255\}$\\
    \var{byte}&exact integer object in $\{-128, \ldots, 127\}$\\
    \var{char}&character (see section~\ref{charactersection})\\
    \var{pair}&pair (see section~\ref{listsection})\\
    \var{vector}&vector (see section~\ref{vectorsection})\\
    \var{string}&string (see section~\ref{stringsection})\\
    \var{condition}&condition (see library section~\extref{lib:conditionssection}{Conditions})\\
    \var{bytevector}&bytevector (see library chapter~\extref{lib:bytevectorschapter}{Bytevectors})\\
    \var{proc}&procedure (see section~\ref{proceduressection})
  \end{tabular}
\texonly\end{center}\endtexonly

Other type restrictions are expressed through parameter-naming
conventions that are described in specific chapters.  For example,
library chapter~\extref{lib:numberchapter}{Arithmetic} uses a number of special
parameter variables for the various subsets of the numbers.

With the listed type restrictions, it is the programmer's responsibility to
ensure that the corresponding argument is of the specified type.
It is the implementation's responsibility to check for
that type.

A parameter called \var{list} means that it is the
programmer's responsibility to pass an argument that is a list (see
section~\ref{listsection}).  It is the implementation's responsibility
to check that the argument is appropriately structured for the
operation to perform its function, to the extent that this is possible
and reasonable.  The implementation must at least check that the
argument is either an empty list or a pair.

Descriptions of procedures may express other restrictions on the
arguments of a procedure.  Typically, such a restriction is formulated
as a phrase of the form ``\var{x} must be a \ldots'' (or otherwise
using the word ``must'').

\section{Implementation responsibilities}

In addition to the restrictions implied by naming conventions, an
entry may list additional explicit restrictions.
These explicit restrictions usually describe both the
programmer's responsibilities, who must ensure that the subforms of a
form are appropriate, or that an appropriate
argument is passed, and the implementation's responsibilities, which
must check that subform adheres to the specified restrictions (if
macro expansion terminates), or if the argument is appropriate.  A description
may explicitly list the implementation's responsibilities for some
arguments or subforms in a paragraph labeled ``\textit{Implementation
  responsibilities}''.  In this case, the responsibilities specified
for these subforms or arguments in the rest of the description are only for the
programmer.  A paragraph describing implementation responsibility does not
affect the implementation's responsibilities for checking subforms or arguments not
mentioned in the paragraph.

\section{Other kinds of entries}

If \var{category} is something other than ``syntax'' and
``procedure'', then the entry describes a non-procedural value, and
the \var{category} describes the type of that value.  The header line

\noindent\rvproto{\&who}{condition type}\\
indicates that {\cf\&who} is a condition type.  The header line

\noindent\rvproto{unquote}{auxiliary syntax}\\
indicates that {\cf unquote} is a syntax binding that may occur
only as part of specific surrounding expressions.  Any use as an
independent syntactic construct or identifier is a syntax violation.
As with ``\exprtype'' entries, some ``auxiliary syntax'' entries  carry a
suffix ({\cf expand}), specifying that the syntactic keyword of the
construct is exported with level $1$.
\section{Equivalent entries}

The description of an entry occasionally states that it is \textit{the
  same} as another entry.  This means that both entries are
equivalent.  Specifically, it means that if both entries have the same
name and are thus exported from different libraries, the entries from
both libraries can be imported under the same name without conflict.

\section{Evaluation examples}

The symbol ``\evalsto'' used in program examples can be read
``evaluates to''.  For example,

\begin{scheme}
(* 5 8)      \ev  40%
\end{scheme}

means that the expression {\tt(* 5 8)} evaluates to the object {\tt 40}.
Or, more precisely:  the expression given by the sequence of characters
``{\tt(* 5 8)}'' evaluates, in an environment that imports the relevant library, to an object
that may be represented externally by the sequence of characters ``{\tt
40}''.  See section~\ref{datumsyntaxsection} for a discussion of external
representations of objects.

The ``\evalsto'' symbol is also used when the evaluation of an
expression causes a violation.  For example,

\begin{scheme}
(integer->char \sharpsign{}xD800) \xev \exception{\&assertion}%
\end{scheme}
%
means that the evaluation of the expression {\cf (integer->char
  \sharpsign{}xD800)} must raise an exception with condition type
{\cf\&assertion}.

Moreover, the ``\evalsto'' symbol is also used to explicitly say that
the value of an expression in unspecified.  For example:
%
\begin{scheme}
(eqv? "" "")             \ev  \unspecified%
\end{scheme}

Mostly, examples merely illustrate the behavior specified in the
entry.  In some cases, however, they disambiguate otherwise ambiguous
specifications and are thus normative.  Note that, in some cases,
specifically in the case of inexact number objects, the return value is only
specified conditionally or approximately.  For example:
%
\begin{scheme}
(atan -inf.0)                  \lev -1.5707963267948965 ; \textrm{approximately}%
\end{scheme}

\section{Naming conventions}

By convention, the names of procedures that store values into previously
allocated locations (see section~\ref{storagemodel}) usually end in
``\ide{!}''.

By convention, ``\ide{->}'' appears within the names of procedures that
take an object of one type and return an analogous object of another type.
For example, {\cf list->vector} takes a list and returns a vector whose
elements are the same as those of the list.

By convention, the names of predicates---procedures that always return
a boolean value---end in ``\ide{?}'' when the name contains any
letters; otherwise, the predicate's name does not end with a question
mark.

By convention, the components of compound names are separated by ``\ide{-}''
In particular, prefixes that are actual words or can be pronounced as
though they were actual words are followed by a hyphen, except when
the first character following the hyphen would be something other than
a letter, in which case the hyphen is omitted.  Short,
unpronounceable prefixes (``\ide{fx}'' and ``\ide{fl}'') are not
followed by a hyphen.

By convention, the names of condition types start with
``{\cf\&}''\index{&@\texttt{\&}}.

%%% Local Variables: 
%%% mode: latex
%%% TeX-master: "r6rs"
%%% End: 
 \par
\chapter{库(Libraries)}
\label{librarychapter}
\mainindex{library,库}
库是一个程序可以被独立发布的部分。库系统支持库内的宏定义,宏导出,且区别需要定义和导出的不同阶段。本章为库定义了符号,且为库的扩展(expansion )和执行定义了语义。

\section{库形式}
\label{librarysyntaxsection}

一个库定义必须有下面的形式:\mainschindex{library,库}\mainschindex{import}\mainschindex{export}

\begin{scheme}
(library \hyper{library~name}
  (export \hyper{export~spec} \ldots)
  (import \hyper{import~spec} \ldots)
  \hyper{library~body})%
\end{scheme}

一个库的声明包含下面的要素:

\begin{itemize}
\item \hyper{library~name}指定了库的名字(可能还有版本)。
\item {\cf export}子形式指定一个导出的列表,这个列表命名了定义在或导入到这个库的绑定的子集。
\item {\cf import}子形式指定了作为一个导入依赖列表导入的绑定,其中每一个依赖指定:
\begin{itemize}
\item 导入的库的名字,且可以可选地包含它的版本,
\item 相应的级别,比如,扩展(expand)或运行时(run time)(见\ref{phasessection}小节)和
\item 为了在导入库中可用的库导出的子集,和为在导入库中可用的为每一个库导出准备的名字。
\end{itemize}
\item \hyper{library body}是库的内部,由一系列定义和紧随其后的表达式组成。标识符可以既是本地的(未导出的)也是导出绑定的,且表达式是为它们的效果求值的初始化表达式。
\end{itemize}

一个标识符可以从两个或更多的库中使用相同的本地名字导入,或者使用两个不同的级别从相同的库中导入,只要每个库导出的绑定是一样的(也就是说,绑定在一个库中被定义,它只能通过导出和再导出经过导入)。否则,没有标识符可以被导入多次,被定义多次,或者同时被定义和导入。除了被显式地导入到库中或在库中定义的标识符,其它标识符在库中都是不可见的。

一个\hyper{library name}在一个实现中唯一地标识一个库,且在实现中的其它所有的库的{\cf import}子句(clauses)(见下面)中是全局可见的。一个\hyper{library name}有下面的形式:

\begin{scheme}
(\hyperi{identifier} \hyperii{identifier} \ldots \hyper{version})%
\end{scheme}

其中\hyper{version}或者是空的,或者有以下的形式:
%
\begin{scheme}
(\hyper{sub-version} \ldots)%
\end{scheme}

每一个\hyper{sub-version}必须表示一个精确的非负整数对象。一个空的\hyper{version}等价于{\cf ()}。

一个\hyper{export~spec}命名一个集合,这个集合包含导入的和本地的定义,这个集合将被导出,可能还会使用不同的外部名字。一个\hyper{export~spec}必须是如下的形式当中的一个:

\begin{scheme}
\hyper{identifier}
(rename (\hyperi{identifier} \hyperii{identifier}) \ldots)%
\end{scheme}

在一个\hyper{export~spec}中,一个\hyper{identifier}命名一个被定义在或被导入到库中的单独的绑定,其中,这个导入的外部名字和库中绑定的名字是一样的。一个{\cf rename}指明在每一个{\cf (\hyperi{identifier} \hyperii{identifier})}这样的配对中,被命名为\hyperi{identifier}的绑定使用\hyperii{identifier}作为外部名字。

每一个\hyper{import~spec}指定一个被导入到库中的绑定的集合,在这个集合中,级别是可见的,且通过它可以知道本地的名字。一个\hyper{import spec}必须是下面的一个:
%
\begin{scheme}
\hyper{import set}
(for \hyper{import~set} \hyper{import~level} \ldots)%
\end{scheme}

一个\hyper{import level}是下面中的一个:
\begin{scheme}
run
expand
(meta \hyper{level})%
\end{scheme}

其中\hyper{level}表示一个精确的整数对象。

在一个\hyper{import level}中,{\cf run}是{\cf (meta 0)}的缩写,且{\cf expand}是{\cf (meta 1)}的缩写。级别和阶段(phases)在\ref{phasessection}小节讨论。

一个\hyper{import~set}命名一个来自另一个库的绑定的集合,且可能为导入的绑定指定一个本地的名字。它必须是下面的一个:

\begin{scheme}
\hyper{library~reference}
(library \hyper{library~reference})
(only \hyper{import~set} \hyper{identifier} \ldots)
(except \hyper{import~set} \hyper{identifier} \ldots)
(prefix \hyper{import~set} \hyper{identifier})
(rename \hyper{import~set} (\hyperi{identifier} \hyperii{identifier}) \ldots)%
\end{scheme}

一个\hyper{library~reference}通过它的名字和可选的版本标识一个库。它有下面形式中的一个:

\begin{scheme}
(\hyperi{identifier} \hyperii{identifier} \ldots)
(\hyperi{identifier} \hyperii{identifier} \ldots \hyper{version~reference})%
\end{scheme}

一个第一个\hyper{identifier}是{\cf for}, {\cf library}, {\cf only}, {\cf except}, {\cf prefix},或{\cf rename}的\hyper{library~reference}只允许出现在一个{\cf library \hyper{import~set}}中。否则{\cf <import set> (library <library reference>)}等价于\hyper{library~reference}。

一个没有\hyper{version~reference}(上面的第一个形式)的\hyper{library~reference}等价于\hyper{version~reference}是{\cf ()}的\hyper{library~reference}。

一个\hyper{version~reference}指定它匹配的\hyper{version}的一个子集。\hyper{library~reference}指定所有相同名字且版本匹配\hyper{version~reference}的库。一个\hyper{version~reference}有下面的形式:
%
\begin{scheme}
(\hyperi{sub-version reference} \ldots \hypern{sub-version reference})
(and \hyper{version reference} \ldots)
(or \hyper{version reference} \ldots)
(not \hyper{version reference})%
\end{scheme}
%
第一个形式的一个\hyper{version reference}匹配\hyper{version}至少$n$个要素的,且\hyper{sub-version reference}匹配对应的\hyper{sub-version}。一个{\cf and} \hyper{version reference}匹配一个版本,如果所有的跟在{\cf and}后的\hyper{version references}都匹配的话。相应地,{\cf or} \hyper{version reference}匹配一个版本,如果跟在{\cf or}后的一个\hyper{version reference}匹配的话。一个{\cf not} \hyper{version reference}匹配一个版本,如果跟在其后的\hyper{version reference}都不匹配的话。

一个\hyper{sub-version reference}有下列形式之一:

\begin{scheme}
\hyper{sub-version}
(>= \hyper{sub-version})
(<= \hyper{sub-version})
(and \hyper{sub-version~reference} \ldots)
(or \hyper{sub-version~reference} \ldots)
(not \hyper{sub-version~reference})%
\end{scheme}

第一个形式的一个\hyper{sub-version reference}匹配一个版本,如果二者相等。第一个形式的{\cf >=} \hyper{sub-version reference}匹配一个子版本如果它大于等于跟在它后面的\hyper{sub-version};{\cf <=}与其类似。一个{\cf and} \hyper{sub-version reference}匹配一个子版本,如果所有随后的\hyper{sub-version reference}都匹配的话。相应地,一个{\cf or} \hyper{sub-version reference}匹配一个子版本,如果随后的\hyper{sub-version reference}中的一个匹配的话。一个{\cf not} \hyper{sub-version reference}匹配一个子版本,如果随后的\hyper{sub-version reference}都不匹配的话。

例子:

\texonly\begin{center}\endtexonly
  \begin{tabular}{lll}
    版本参考 & 版本 & 匹配?
    \\
    {\cf ()} & {\cf (1)} & 是\\
    {\cf (1)} & {\cf (1)} & 是\\
    {\cf (1)} & {\cf (2)} & 否\\
    {\cf (2 3)} & {\cf (2)} & 否\\
    {\cf (2 3)} & {\cf (2 3)} & 是\\
    {\cf (2 3)} & {\cf (2 3 5)} & 是\\
    {\cf (or (1 (>= 1)) (2))} & {\cf (2)} & 是\\
    {\cf (or (1 (>= 1)) (2))} & {\cf (1 1)} & 是\\
    {\cf (or (1 (>= 1)) (2))} & {\cf (1 0)} & 否\\
    {\cf ((or 1 2 3))} & {\cf (1)} & 是\\
    {\cf ((or 1 2 3))} & {\cf (2)} & 是\\
    {\cf ((or 1 2 3))} & {\cf (3)} & 是\\
    {\cf ((or 1 2 3))} & {\cf (4)} & 否
  \end{tabular}
\texonly\end{center}\endtexonly

当多于一个库被库参考引用的时候,库的选择有实现定义的方法决定。

为了避免诸如不兼容的类型和重复的状态这些问题,实现必须禁止库名字由相同标识符序列组成但版本不匹配的两个库在同一个程序中共存。

默认情况下,一个被导入的库导出的所有的绑定在一个使用被导入的库给的绑定的名字的导入库中是可见的。被导入的绑定和那些绑定的名字的精确的集合可以通过下面描述的{\cf only}, {\cf except}, {\cf prefix}, 和{\cf rename}形式进行调整。

\begin{itemize}
\item 一个{\cf only}形式产生来自另一个\hyper{import~set}的绑定的一个子集,只包括被列出的\hyper{identifier}。被包含的\hyper{identifier}必须在原始的\hyper{import~set}中。
\item 一个{\cf except}形式产生来自另一个\hyper{import~set}的绑定的一个子集,包括除了被列出的所有的\hyper{identifier}。所有被排除的\hyper{identifier}必须在原始的\hyper{import~set}中。
\item 一个{\cf prefix}从另一个\hyper{import~set}中添加一个\hyper{identifier}前缀到每个名字中。
\item 一个{\cf rename}形式,{\cf (rename (\hyperi{identifier} \hyperii{identifier}) \ldots)},从一个中间的\hyper{import~set}删除{\cf \hyperi{identifier} \ldots}的绑定,然后将对应的{\cf \hyperii{identifier} \ldots}添加回到最终的\hyper{import~set}中。每一个\hyperi{identifier}必须在原始的\hyper{import~set}中,每一个\hyperii{identifier}必须不在中间的\hyper{import~set}中,且\hyperii{identifier}必须是不一样的。
\end{itemize}
不符合上面的约束是一个语法错误。

\label{librarybodysection}
一个{\cf 库}形式的\hyper{library~body}由被分类为\textit{定义(definitions)}\mainindex{definition,定义}或\textit{表达式(expressions)}\mainindex{expression,表达式}的形式组成。哪些形式属于哪些类型是根据被导入的库和表达式的结果来决定的---见第\ref{expansionchapter}章。通常,不是定义(见\ref{defines}小节,有基本库可见的定义)的形式是表达式。

一个\hyper{library~body}和\hyper{body}差不多(见\ref{bodiessection}小节),除了一个\hyper{library~body}不需要包括任何表达式。它必须有下面的形式:

\begin{scheme}
\hyper{definition} \ldots \hyper{expression} \ldots%
\end{scheme}

当{\cf begin}, {\cf let-syntax}, 或{\cf letrec-syntax}形式先于第一个表达式出现在顶层内部的时候,它们被拼接成内部;见\ref{begin}小节。内部的一些或所以,包括在{\cf begin}, {\cf let-syntax}, 或{\cf letrec-syntax}形式里面的部分,可以通过一个句法抽象指定(见\ref{macrosection}小节)。

转换表达式和绑定像第\ref{expansionchapter}章描述的,按从左向右的顺序被求值和被创建。变量的表达式按从左向右求值,就像在一个隐式的{\cf letrec*}中,且内部的表达式也在变量定义的表达式之后从左向右进行求值。每一个导出的变量和它本地副本值得初始化都会创建一个新的位置。两次返回给内部最后表达式的继续的行为是未定义的。

\begin{note}
出现在库语法的名字{\cf library},\allowbreak{}{\cf export},\allowbreak{}{\cf import},
{\cf for},\allowbreak{}{\cf run},\allowbreak{}{\cf expand},\allowbreak{}{\cf meta},
{\cf import},\allowbreak{}{\cf export},\allowbreak{}{\cf only},\allowbreak{}{\cf except},\allowbreak{}{\cf
  prefix},\allowbreak{}{\cf rename},\allowbreak{}{\cf and},\allowbreak{}{\cf or},\allowbreak{}{\cf not},\allowbreak{}{\cf >=}\allowbreak{}和\allowbreak{}{\cf <=}\allowbreak{}
是语法的一部分,且不是被保留的,也就是说,相同的名字可以在库中用作其它的用途,甚至以不同的含义导出或导入到其它库中,这些都不会影响它们在{\cf 库}形式中的使用。
\end{note}

在一个库的里面定义的绑定在库外面的代码中是不可见的,除非绑定被显式地从这个库中导出。可是,一个导出的宏可能\emph{隐式地导出(implicitly export)}一个另外的标识符,其是在库中定义的或导入到库中的。也就是说,它可以插入标识符的一个引用到其产生的代码中。

\label{importsareimmutablesection}
所有的显式地导出的变量在导出的和导入的的库中都是不可变的。因此,不管是在导入的还是导出的库中,一个显式地导出的变量出现在一个{\cf set!}表达式的左边是一个语法错误。

所有的隐式地导出的变量在导出的和导入的的库中也都是不可变的。因此,在一个变量被定义的库的外面通过宏产生的代码中,如果这个变量出现在一个{\cf set!}表达式的左边,那么这是一个语法错误。如果一个被赋值的变量的引用出现在这个变量被定义的库的外面通过宏产生的代码中,那么这也是一个语法错误,其中,一个被赋值的变量指出现在导出库中一个{\cf set!}表达式左边的变量。

所有其它的定义在一个库中的变量是可变的。

\section{导入和导出级别(levels)}
\label{phasessection}

扩展一个库可能需要来自其它库的运行时信息。比如,如果一个宏转换调用一个来自库$A$的过程,那么在库$B$中扩展宏的任何使用之前,库$A$必须被实例化(instantiated)。当库$B$作为一个程序的部分被最终运行的时候,库$A$可能不被需要,或者,它可能也被库$B$的运行时需要。库机制使用阶段来区别这些时间,这会在本小节解释。

每个库可用通过扩展时(expand-time)信息(最低限度地,它的导入的库,导出的关键词的一个列表,导出的变量的一个列表,计算转换表达式的代码)和运行时信息(最低限度地,计算变量定义的右边表达式的代码,计算内部表达式的代码)表现其特征。扩展时信息必须对任何导出绑定的扩展引用可见,且运行时信息必须对任何导出变量绑定的求值引用可见。

\mainindex{phase,阶段B}
%
一个\emph{阶段(phase)}是一段时间,在这段时间里库中的表达式被求值。在一个库的内部,顶层表达式和{\cf define}形式的右边在运行时,也就是阶段$0$,被求值,且{\cf define-syntax}形式的右边在扩展时,也就是阶段$1$,被求值。当{\cf define-syntax}, {\cf let-syntax}, 或{\cf letrec-syntax}出现在在阶段*n*被求值的代码中时,它们的右边将在阶段$n+1$被求值。

这些阶段和库它自己使用的阶段是相关的。库的一个\defining{实例(instance)}对应于一个与另一个库相关的特定阶段的它的变量定义和表达式的求值过程---一个叫做\defining{实例化(instantiation)}的过程。比如,如果库$B$中的一个顶层表达式引用库$A$中导出的一个变量,那么它在阶段$0$(相对于$B$的阶段)从$A$的一个实例中引用这个导出。但是,$B$中一个阶段$1$的表达式引用$A$中相同的绑定,那么,它在阶段$1$(相对于$B$的阶段)从$A$的一个实例中引用这个导出。

一个库的\defining{访问(visit)}对应于在一个特定阶段与另一个库相关的语法定义的计算过程—一个叫做\defining{访问(visit)}的过程。比如,如果库$B$中的一个顶层表达式引用来库$A$导出的一个宏,那么,它在阶段$0$(相对应$B$的阶段)从$A$的访问中引用这个导出,其对应于在阶段$1$宏转换表达式的求职过程。

\mainindex{level}\mainindex{import level}
%
一个\emph{级别}是一个标识符的词法属性,其决定了它可以在哪个阶段被引用。在一个库中,每个用定义绑定的标识符的级别是$0$;也就是说,在库中只能在阶段$0$引用这个标识符。除了导出库中标识符的级别之外,每一个导入的绑定的级别由导入库中{\cf import}的封装的{\cf for}形式决定。导入和导出级别通过所有级别组合的成对相加的方式组合。比如,一个以$p_a$和$p_b$级别导出,以$q_a$,$q_b$和$q_c$级别导入的被导入的标识符的引用在以下级别是有效的:$p_a + q_a$, $p_a + q_b$, $p_a + q_c$, $p_b + q_a$, $p_b + q_b$, 和$p_b + q_c$。一个没有封装的{\cf for}的\hyper{import~set}等价于{\cf (for \hyper{import~set} run)},它和{\cf (for \hyper{import~set} (meta 0))}是一样的。

对于所有的定义在导出库中的绑定来说,一个被导出的的绑定的级别是$0$。一个重新导出的绑带,也就是一个从其它库导入的导出,的级别和重新导出库中的绑定的有效导入级别是一样的。

对于定义在库报告中的库来说,几乎所有的绑定,导出级别是$0$。例外是来自\rsixlibrary{base}库的{\cf syntax-rules},\allowbreak{}{\cf identifier-syntax},\allowbreak{}{\cf ...}\allowbreak{}和\allowbreak{}{\cf \_}\allowbreak{}形式以级别$1$被导出,来自\rsixlibrary{base}库的{\cf set!}形式以级别$0$和$1$被导出,来自复合\thersixlibrary{}库(见库的第\extref{lib:complibchapter}{Composite library}章)的所有绑定以级别$0$和$1$被导出。

一个库中的宏扩展可以引出标识符的一个引用,其中这个标识符没有显式地导入到这个库中。在这种情况下,引用的阶段必须符合标识符的作为偏移的级别,这个偏移是源库(也就是提供标识符词法上下文的库)和封装引用的库的阶段的不同。比如,假设扩展一个库调用一个宏转换,且宏转换的求值引用一个标识符,这个标识符被从另一个库中被导出(所以,库的阶段$1$的实例被使用);进一步假设绑定的值是一个表示一个指示级别$n$绑定的标识符的语法对象;那么,在库被扩展的时候,标识符必须只能在$n+1$阶段被使用。这个级别和阶段的组合就是为什么标识符负的级别是有用的,甚至,尽管库只存在非负的阶段。

如果在一个程序的扩展形式中,一个库的定义中的任何一个在阶段$0$被引用,那么被引用的库的一个作为阶段$0$实例会在程序的定义和表达式被求值之前被创建。这条规则的应用是透明的:如果一个库的扩展形式在阶段$0$从另一个库引用一个标识符,那么,在引用库在阶段$n$被实例化之前,被引用的库必须在阶段$n$被实例化。当一个标识符在任何大于$0$的阶段$n$被引用,那么于此相反,定义库会在阶段$n$被实例化,其在引用被求值之前的一些未定义的时间。同样地,在一个库的扩展期间,当一个宏关键词在阶段$n$被引用的时候,定义库在阶段$n$被访问,其是引用被计算之前的一定未定义的时间。

一个实现可以为不同的阶段区别实例/访问,或者在任何阶段使用一个实例/访问就像在任何其它阶段的一个实例/访问一样。更进一步,一个实现可以扩展每一个{\cf 库}形式以区别在任何阶段的库的访问和/或在大于$0$的阶段的库的实例。一个实现可以创建更多库的实例/访问在比要求的安全引用的更多的阶段。当一个标识符作为一个表达式出现在一个与标识符级别不一致的阶段时,那么一个实现可以抛出一个异常,异常的抛出可以在扩展时也可以在运行时,或者,它也可以允许这个引用。因此,一个库可能是不可移植的,当其含义依赖于一个库的实例在库的整个阶段或{\cf library}扩展时是有区别的还是共享的时候。

\section{例子}

各种\hyper{import~spec}和\hyper{export~spec}的例子:

(注:下面的例子已根据勘误表更正。)

\begin{scheme}
(library (stack)
  (export make push! pop! empty!)
  (import (rnrs)
          (rnrs mutable-pairs))

  (define (make) (list '()))
  (define (push! s v) (set-car! s (cons v (car s))))
  (define (pop! s) (let ([v (caar s)])
                     (set-car! s (cdar s))
                     v))
  (define (empty! s) (set-car! s '())))

(library (balloons)
  (export make push pop)
  (import (rnrs))

  (define (make w h) (cons w h))
  (define (push b amt)
    (cons (- (car b) amt) (+ (cdr b) amt)))
  (define (pop b) (display "Boom! ")
                  (display (* (car b) (cdr b)))
                  (newline)))

(library (party)
  ;; 所有的导出:
  ;; make, push, push!, make-party, pop!
  (export (rename (balloon:make make)
                  (balloon:push push))
          push!
          make-party
          (rename (party-pop! pop!)))
  (import (rnrs)
          (only (stack) make push! pop!) ; 非空的!
          (prefix (balloons) balloon:))

  ;; 以气球的堆栈创建一个派对(party),
  ;; 从两个气球开始
  (define (make-party)
    (let ([s (make)]) ; from stack
      (push! s (balloon:make 10 10))
      (push! s (balloon:make 12 9))
      s))
  (define (party-pop! p)
    (balloon:pop (pop! p))))


(library (main)
  (export)
  (import (rnrs) (party))

  (define p (make-party))
  (pop! p)        ; 显示"Boom! 108"
  (push! p (push (make 5 5) 1))
  (pop! p))       ; 显示"Boom! 24"%
\end{scheme}

宏和阶段的例子:

\begin{schemenoindent}
(library (my-helpers id-stuff)
  (export find-dup)
  (import (rnrs))

  (define (find-dup l)
    (and (pair? l)
         (let loop ((rest (cdr l)))
           (cond
            [(null? rest) (find-dup (cdr l))]
            [(bound-identifier=? (car l) (car rest))
             (car rest)]
            [else (loop (cdr rest))])))))

(library (my-helpers values-stuff)
  (export mvlet)
  (import (rnrs) (for (my-helpers id-stuff) expand))

  (define-syntax mvlet
    (lambda (stx)
      (syntax-case stx ()
        [(\_ [(id ...) expr] body0 body ...)
         (not (find-dup (syntax (id ...))))
         (syntax
           (call-with-values
               (lambda () expr)
             (lambda (id ...) body0 body ...)))]))))

(library (let-div)
  (export let-div)
  (import (rnrs)
          (my-helpers values-stuff)
          (rnrs r5rs))

  (define (quotient+remainder n d)
    (let ([q (quotient n d)])
      (values q (- n (* q d)))))
  (define-syntax let-div
    (syntax-rules ()
     [(\_ n d (q r) body0 body ...)
      (mvlet [(q r) (quotient+remainder n d)]
        body0 body ...)])))%
\end{schemenoindent}


%%% Local Variables:
%%% mode: latex
%%% TeX-master: "r6rs"
%%% End:
 \par
\chapter{Top-level programs}
\label{programchapter}

A \defining{top-level program} specifies an entry point for defining and running
a Scheme program.  A top-level program specifies a set of libraries to import and
code to run.  Through the imported libraries, whether directly or through the
transitive closure of importing, a top-level program defines a complete Scheme
program.

\section{Top-level program syntax}
\label{programsyntaxsection}

A top-level program is a delimited piece of text, typically a file, that 
has the following form:
%
\begin{scheme}
\hyper{import form} \hyper{top-level body}%
\end{scheme}
%
An \hyper{import form} has the following form:
%
\begin{scheme}
(import \hyper{import spec} \dotsfoo)%
\end{scheme}
%
A \hyper{top-level body} has the following form:
\begin{scheme}
\hyper{top-level body form} \dotsfoo%
\end{scheme}
%
A \hyper{top-level body form} is either a \hyper{definition} or an
\hyper{expression}.

The \hyper{import form} is identical to the import clause in
libraries (see section~\ref{librarysyntaxsection}), 
and specifies a set of libraries to import.  A \hyper{top-level 
 body} is like a \hyper{library body} (see
section~\ref{librarybodysection}), except that 
definitions and expressions may occur in any order.  Thus, the syntax
specified by \hyper{top-level body form} refers to the result of macro
expansion.

When uses of {\cf begin}, {\cf let-syntax}, or {\cf letrec-syntax}
from the \rsixlibrary{base} library
occur in a top-level body prior to the first
expression, they are spliced into the body; see section~\ref{begin}.
Some or all of the body, including portions wrapped in {\cf begin},
{\cf let-syntax}, or {\cf letrec-syntax}
forms, may be specified by a syntactic abstraction
(see section~\ref{macrosection}).

\section{Top-level program semantics}

A top-level program is executed by treating the program similarly to a library, and
evaluating its definitions and expressions.
The semantics of a top-level body may be roughly explained by
a simple translation into a library body: 
Each \hyper{expression} that appears before a
definition in
the top-level body is converted into a dummy definition 
%
\begin{scheme}
(define \hyper{variable} (begin \hyper{expression} \hyper{unspecified}))%
\end{scheme}
%
where \hyper{variable} is a fresh identifier and \hyper{unspecified}
is a side-effect-free expression returning an unspecified value.
(It is generally impossible to determine which forms are 
definitions and expressions without concurrently expanding the body, so
the actual translation is somewhat more complicated; see
chapter~\ref{expansionchapter}.)

On platforms that support it, a top-level program may access its command line 
by calling the {\cf command-line} procedure (see library 
section~\extref{lib:command-line}{Command-line access and exit values}).

%%% Local Variables: 
%%% mode: latex
%%% TeX-master: "r6rs"
%%% End: 
 \par
\chapter{基本语法(Primitive syntax)}

在一个{\cf library}形式或顶层程序的{\cf import}形式的后面,构成库或顶层程序内部的形式依赖于被导入的库。尤其,被导入的句法关键词决定可用的句法抽象以及每个形式是一个定义还是一个表达式。可是,一些形式总是可用的,它们独立于被导入的库,包括常量字面量,变量引用,过程调用,和宏的使用。

\section{基本表达式类型}
\label{primitiveexpressionsection}

本节的条目都是都是描述表达式,其可以出现在\hyper{expression}句法变量的地方。也可以参见第\ref{expressionsection}小节。

\subsection*{常量字面量(Constant literals)}\unsection

\begin{entry}{%
\pproto{\hyper{number}}{\exprtype}
\pproto{\hyper{boolean}}{\exprtype}
\pproto{\hyper{character}}{\exprtype}
\pproto{\hyper{string}}{\exprtype}
\pproto{\hyper{bytevector}}{\exprtype}}\mainindex{literal}

一个由一个数字对象,或一个布尔,或一个字符,或一个字符串,或一个字节向量组成的表达式求值“等于它自己(to itself)”。

\begin{scheme}
145932     \ev  145932
\schtrue   \ev  \schtrue
"abc"      \ev  "abc"
\#vu8(2 24 123) \ev \#vu8(2 24 123)%
\end{scheme}

就像\ref{storagemodel}小节描述的那样,一个字面量表达式的值是不可改变的。
\end{entry}

\subsection*{变量引用(Variable references)}\unsection
\begin{entry}{%
\pproto{\hyper{variable}}{\exprtype}}

由一个变量\index{variable,变量}组成的表达式(\ref{variablesection}小节)是一个变量引用,如果它不是一个宏使用(见下面)的话。这个变量引用的值是被存储在这个变量绑定的位置的值。引用一个未绑定的(unbound)\index{unbound,未绑定的}变量是一个语法错误。

下面的例子假设基本库已经被导入:
%
\begin{scheme}
(define x 28)
x   \ev  28%
\end{scheme}
\end{entry}

\subsection*{过程调用(Procedure calls)}\unsection

\begin{entry}{%
\pproto{(\hyper{operator} \hyperi{operand} \dotsfoo)}{\exprtype}}

一个过程调用由被调用的过程表达式,传递给这个过程的参数以及两边的下括号组成。在一个表达式上下文中,如果\hyper{operator}不是一个作为句法关键字绑定的标识符(见下面的\ref{macrosection}小节),那么这个形式是一个过程调用。

当一个过程调用被求值的时候,操作和操作数被(以一个未定义的顺序)求值,且最终产生的过程接受最终产生的参数。\mainindex{call,调用}\mainindex{procedure call,过程调用}

下面的例子假设\rsixlibrary{base}库已被导入:
%
\begin{scheme}%
(+ 3 4)                          \ev  7
((if \schfalse + *) 3 4)         \ev  12%
\end{scheme}
%
如果\hyper{operator}的值不是一个过程,那么一个条件类型是{\cf\&assertion}的异常被抛出。并且,如果一个\hyper{operator}接受的参数的数量不是\hyper{operand}的数量,那么一个条件类型是{\cf\&assertion}的异常被抛出。

\begin{note} 相比于其它的Lisp方言,求值的顺序是未定义的,且操作表达式和操作数表达式总是以相同的求值规则被求值。

此外,尽管求值的顺序是未定义,但是对操作和操作数表达式进行的任何并发求值的结果必须与按某种顺序求值的结果一致。对每个过程调用的求值顺序可以有不同的选择。
\end{note}

\begin{note}在许多Lisp方言中,空组合式{\tt ()}是合法的表达式。在Scheme中,写成表/点对形式的表达式至少要包含一个子表达式,因此{\tt ()}不是一个语法上有效的表达式。
\end{note}

\end{entry}

\section{宏(Macros)}
\label{macrosection}

库的顶层程序可以定义和使用新的叫做{\em 句法抽象(syntactic abstraction)}或{\em 宏(macro)}\mainindex{syntactic abstraction,句法抽象}\mainindex{macro,宏}派生表达式和定义。一个句法抽象通过绑定一个关键词到一个{\em 宏转换(macro transformer)},或简称为{\em 转换(transformer)}\index{macro transformer,宏转换}\index{transformer,转换},来创建。转换决定一个宏的使用(被称为\defining{宏使用(macro use)})怎样被转录我一个更加基本的形式。

大部分宏有形式:
\begin{scheme}
(\hyper{keyword} \hyper{datum} \dotsfoo)%
\end{scheme}%
其中\hyper{keyword}是唯一决定形式种类的标识符。这个标识符叫做宏的{\em 句法关键词(syntactic keyword)}\index{syntactic keyword,句法关键词},或简称为宏的{\em 关键词}\index{keyword,关键词}。\hyper{datum}的数量和每一个的语法由句法抽象决定。

宏使用可以表现为不正确的表形式,单独的标识符或{\cf set!}形式,其中{\cf set!}形式的第二个子形式是关键字(见第\ref{identifier-syntax}小节,库的第\extref{lib:make-variable-transformer}{{\cf make-variable-transformer}}小节):
\begin{scheme}
(\hyper{keyword} \hyper{datum} \dotsfoo . \hyper{datum})
\hyper{keyword}
(set! \hyper{keyword} \hyper{datum})%
\end{scheme}

在\ref{define-syntax}和\ref{let-syntax}小节描述的{\cf define-syntax}, {\cf let-syntax} 和{\cf letrec-syntax}形式创建关键词的绑定,将它们与宏转换的形式联系起来,且控制它们可见的范作用域。

在\ref{syntaxrulessection}小节描述的{\cf syntax-rules} 和{\cf identifier-syntax}形式通过一个模式语言创建转换。并且,库的第\extref{lib:syntaxcasechapter}{{\cf syntax-case}}章描述的{\cf syntax-case}形式允许通过任意的Scheme代码创建转换。

关键词和变量占据相同的名字空间。也就是,在相同的作用域内,一个标识符可以被绑定到一个变量或一个关键词中的零个或一个,不能同时绑定到两个,且任何一种本地绑定可以覆盖任何一种其它的绑定。

使用{\cf syntax-rules} 和{\cf identifier-syntax}定义的宏是“卫生的(hygienic)”和“引用透明的(referentially transparent)”,且因此保持了Scheme词法作用域的特点\cite{Kohlbecker86,hygienic,Bawden88,macrosthatwork,syntacticabstraction}:\mainindex{hygienic,卫生的} \mainindex{referentially transparent,引用透明的}

\begin{itemize}
\item 如果一个宏转换为一个没有出现在宏使用中的标识符(变量或关键词)插入一个绑定,就实际效果而言,该标识符在其整个作用域中将被改名使用,以避免与其他标识符冲突。

\item 如果一个宏转换器插入了对某标识符的自由引用,那么,无论是否存在包围该宏的使用的局部绑定,该引用都将指向定义转换器时可见的绑定。
\end{itemize}

使用{\cf syntax-case}工具定义的宏也是卫生的,除非{\cf datum->syntax}(见库的第\extref{lib:conversionssection}{Syntax-object and datum conversions}小节)被使用。

%%% Local Variables:
%%% mode: latex
%%% TeX-master: "r6rs"
%%% End:
 \par
\chapter{Expansion process}
\label{expansionchapter}

Macro uses (see section~\ref{macrosection}) are expanded into \textit{core
forms}\mainindex{core form} at the start of evaluation (before compilation or interpretation)
by a syntax \emph{expander}.
The set of core forms is implementation-dependent, as is the
representation of these forms in the expander's output.
If the expander encounters a syntactic abstraction, it invokes
the associated transformer to expand the syntactic abstraction, then
repeats the expansion process for the form returned by the transformer.
If the expander encounters a core form, it recursively
processes its subforms that are in expression or definition context,
if any, and reconstructs the form from the
expanded subforms.
Information about identifier bindings is maintained during expansion
to enforce lexical scoping for variables and keywords.

To handle definitions, the expander processes the initial
forms in a \hyper{body} (see section~\ref{bodiessection}) or
\hyper{library body} (see section~\ref{librarybodysection})
from left to
right.  How the expander processes each form encountered 
depends upon the kind of form.

\begin{description}
\item[macro use]
The expander invokes the associated transformer to transform the macro
use, then recursively performs whichever of these actions are appropriate
for the resulting form.

\item[{\cf define-syntax} form]
The expander expands and evaluates the right-hand-side expression and binds the
keyword to the resulting transformer.

\item[{\cf define} form]
The expander records the fact that the defined identifier is a variable but defers
expansion of the right-hand-side expression until after all of the
definitions have been processed.

\item[{\cf begin} form]
The expander splices the subforms into the list of body forms it is
processing.  (See section~\ref{begin}.)

\item[{\cf let-syntax} or {\cf letrec-syntax} form]
The expander splic\-es the inner body forms into the list of (outer) body forms it is
processing, arranging for the keywords bound by the {\cf let-syntax}
and {\cf letrec-syntax} to be visible only in the inner body forms.

\item[expression, i.e., nondefinition]
The expander completes the expansion of the deferred right-hand-side expressions
and the current and remaining expressions in the body, and then
creates the equivalent of a {\cf letrec*} form from the defined variables,
expanded right-hand-side expressions, and expanded body expressions.
\end{description}

For the right-hand side of the definition of a variable, expansion is
deferred until after all of the definitions have been seen.  Consequently,
each keyword and variable reference within the right-hand side
resolves to the local binding, if any.

A definition in the sequence of forms must not define any identifier whose
binding is used to determine the meaning of the undeferred portions of the
definition or any definition that precedes it in the sequence of forms.
For example, the bodies of the following expressions violate this
restriction.

\begin{scheme}
(let ()
  (define define 17)
  (list define))

(let-syntax ([def0 (syntax-rules ()
                     [(\_ x) (define x 0)])])
  (let ([z 3])
    (def0 z)
    (define def0 list)
    (list z)))

(let ()
  (define-syntax foo
    (lambda (e)
      (+ 1 2)))
  (define + 2)
  (foo))%
\end{scheme}

The following do not violate the restriction.

\begin{scheme}
(let ([x 5])
  (define lambda list)
  (lambda x x))         \ev  (5 5)

(let-syntax ([def0 (syntax-rules ()
                     [(\_ x) (define x 0)])])
  (let ([z 3])
    (define def0 list)
    (def0 z)
    (list z)))          \ev  (3)

(let ()
  (define-syntax foo
    (lambda (e)
      (let ([+ -]) (+ 1 2))))
  (define + 2)
  (foo))                \ev  -1%
\end{scheme}

The implementation should treat a violation of the restriction as a
syntax violation.

% Andre's proposed implementation:
% To detect this violation, the expander can record each
% identifier whose denotation is determined during expansion
% of the body, together with the denotation.
% Before an identifier is bound, its current denotation is compared
% against denotations already used for the same (in the sense of
% bound-identifier=?) identifier in the scope of the intended binding,
% to determine if its current denotation has already been used
% during the expansion of the body.

Note that this algorithm does not directly reprocess any form.
It requires a single left-to-right pass over the definitions followed by a
single pass (in any order) over the body expressions and deferred
right-hand sides.

Example:

\begin{scheme}
(lambda (x)
  (define-syntax defun
    (syntax-rules ()
      [(\_ x a e) (define x (lambda a e))]))
  (defun even? (n) (or (= n 0) (odd? (- n 1))))
  (define-syntax odd?
    (syntax-rules () [(\_ n) (not (even? n))]))
  (odd? (if (odd? x) (* x x) x)))%
\end{scheme}

In the example, the definition of {\cf defun} is encountered first, and the keyword
{\cf defun} is associated with the transformer resulting from
the expansion and evaluation of the corresponding right-hand side.
A use of {\cf defun} is encountered next and expands into a
{\cf define} form.
Expansion of the right-hand side of this define form is deferred.
The definition of {\cf odd?} is next and results in the association
of the keyword {\cf odd?} with the transformer resulting from
expanding and evaluating the corresponding right-hand side.
A use of {\cf odd?} appears next and is expanded; the resulting
call to {\cf not} is recognized as an expression
because {\cf not} is bound as a variable.
At this point, the expander completes the expansion of the current
expression (the call to {\cf not}) and the deferred right-hand side of the
{\cf even?} definition;
the uses of {\cf odd?} appearing in these expressions are expanded
using the transformer associated with the keyword {\cf odd?}.
The final output is the equivalent of

\begin{scheme}
(lambda (x)
  (letrec* ([even?
              (lambda (n)
                (or (= n 0)
                    (not (even? (- n 1)))))])
    (not (even? (if (not (even? x)) (* x x) x)))))%
\end{scheme}

although the structure of the output is implementation-dependent.

Because definitions and expressions can be interleaved in a
\hyper{top-level body} (see chapter~\ref{programchapter}),
the expander's processing of a \hyper{top-level body} is somewhat
more complicated.
It behaves as described above for a \hyper{body} or
\hyper{library body} with the following exceptions:
When the expander finds a nondefinition,
it defers its expansion and continues scanning for definitions.
Once it reaches the end of the set of forms, it processes the
deferred right-hand-side and body expressions, then
generates the equivalent of a {\cf letrec*} form from the defined variables,
expanded right-hand-side expressions, and expanded body expressions.
For each body expression \hyper{expression} that appears before a variable definition
in the body, a dummy binding is created at the corresponding place within
the set of {\cf letrec*} bindings, with a fresh temporary variable on the
left-hand side and the equivalent of {\cf (begin \hyper{expression}
  \hyper{unspecified})},
where \hyper{unspecified} is a side-effect-free expression returning
an unspecified value,
on the right-hand side, so that
left-to-right evaluation order is preserved.
The {\cf begin} wrapper allows \hyper{expression} to evaluate to an
arbitrary number of values.

%%% Local Variables: 
%%% mode: latex
%%% TeX-master: "r6rs"
%%% End: 
 \par
%\vfill\eject
\chapter{Base library}
\label{baselibrarychapter}

This chapter describes Scheme's \defrsixlibrary{base} library, which exports many of
the procedure and syntax bindings that are traditionally associated
with Scheme.

Section~\ref{basetailcontextsection} defines the rules that identify
tail calls and tail contexts in constructs from the \rsixlibrary{base}
library.

\section{Base types}
\label{disjointness}

No object satisfies more than one of the following predicates:

\begin{scheme}
boolean?          pair?
symbol?           number?
char?             string?
vector?           procedure?
null?%
\end{scheme}

These predicates define the base types {\em boolean}, {\em pair}, {\em
symbol}, {\em number}, {\em char} (or {\em character}), {\em string}, {\em
vector}, and {\em procedure}.  Moreover, the empty list is a special
object of its own type.
\mainindex{type}\schindex{boolean?}\schindex{pair?}\schindex{symbol?}
\schindex{number?}\schindex{char?}\schindex{string?}\schindex{vector?}
\schindex{procedure?}\index{empty list}\schindex{null?}

Note that, although there is a separate boolean type, any Scheme value
can be used as a boolean value for the purpose of a conditional test;
see section~\ref{booleanvaluessection}.

\section{Definitions}
\label{defines}

Definitions\mainindex{definition} may appear within a
\meta{top-level body} (section~\ref{programsyntaxsection}),
at the top of a \meta{library body} (section~\ref{librarysyntaxsection}),
or at the top of a \meta{body} (section~\ref{bodiessection}).

A \hyper{definition} may be a variable definition
(section~\ref{variabledefinitionsection}) or
keyword definition
(section~\ref{variabledefinitionsection}).
Macro uses that expand into definitions or groups of
definitions (packaged in a {\cf begin}, {\cf let-syntax}, or
{\cf letrec-syntax} form; see section~\ref{begin}) may also appear
wherever other definitions may appear.

\subsection{Variable definitions}
\label{variabledefinitionsection}

The {\cf define} form described in this section is a
\hyper{definition}\mainindex{definition} used to create variable bindings
and may appear anywhere other definitions may appear.

\begin{entry}{%
\proto{define}{ \hyper{variable} \hyper{expression}}{\exprtype}
\rproto{define}{ \hyper{variable}}{\exprtype}
\pproto{(define (\hyper{variable} \hyper{formals}) \hyper{body})}{\exprtype}
\pproto{(define (\hyper{variable} .\ \hyper{formal}) \hyper{body})}{\exprtype}}


The first from of {\cf define} binds \hyper{variable} to a new
location before assigning the value of \hyper{expression} to it.
\begin{scheme}
(define add3
  (lambda (x) (+ x 3)))
(add3 3)                            \ev  6
(define first car)
(first '(1 2))                      \ev  1%
\end{scheme}
%
The continuation of \hyper{expression} should not be invoked more than
once.

\implresp Implementations should detect that the
continuation of \hyper{expression} is invoked more than once.
If the implementation detects this, it must raise an
exception with condition type {\cf\&assertion}.

The second form of {\cf define} is equivalent to
\begin{scheme}
(define \hyper{variable} \hyper{unspecified})%
\end{scheme}
where \hyper{unspecified} is a side-effect-free expression returning
an unspecified value.

In the third form of {\cf define}, \hyper{formals} must be either a
sequence of zero or more variables, or a sequence of one or more
variables followed by a dot {\cf .} and another variable (as
in a lambda expression, see section~\ref{lambda}).  This form is equivalent to
\begin{scheme}
(define \hyper{variable}
  (lambda (\hyper{formals}) \hyper{body}))\rm.%
\end{scheme}

In the fourth form of {\cf define}, 
\hyper{formal} must be a single
variable.  This form is equivalent to
\begin{scheme}
(define \hyper{variable}
  (lambda \hyper{formal} \hyper{body}))\rm.%
\end{scheme}
\end{entry}

\subsection{Syntax definitions}
\label{syntaxdefinitionsection}

The {\cf define-syntax} form described in this section is a
\hyper{definition}\mainindex{definition} used to create keyword bindings
and may appear anywhere other definitions may appear.

\begin{entry}{%
\proto{define-syntax}{ \hyper{keyword} \hyper{expression}}{\exprtype}}

Binds \hyper{keyword} to the value of
\hyper{expression}, which must evaluate, at macro-expansion
time, to a transformer.  Macro transformers can be created using the
{\cf syntax-rules} and {\cf identifier-syntax} forms described in
section~\ref{syntaxrulessection}.  See library
section~\extref{lib:transformerssection}{Transformers} for a more
complete description of transformers.

Keyword bindings established by {\cf define-syntax} are visible
throughout the body in which they appear, except where shadowed by
other bindings, and nowhere else, just like variable bindings established
by {\cf define}.
All bindings established by a set of definitions, whether
keyword or variable definitions, are visible within the definitions
themselves.

\implresp The implementation should detect if the value of
\hyper{expression} cannot possibly be a transformer.

Example:

\begin{scheme}
(let ()
  (define even?
    (lambda (x)
      (or (= x 0) (odd? (- x 1)))))
  (define-syntax odd?
    (syntax-rules ()
      ((odd?  x) (not (even? x)))))
  (even? 10))                       \ev \schtrue{}%
\end{scheme}

An implication of the left-to-right processing order
(section~\ref{expansionchapter}) is that one definition can
affect whether a subsequent form is also a definition.  

Example:

\begin{scheme}
(let ()
  (define-syntax bind-to-zero
    (syntax-rules ()
      ((bind-to-zero id) (define id 0))))
  (bind-to-zero x)
  x) \ev 0%
\end{scheme}

The behavior is unaffected by any binding for
{\cf bind-to-zero} that might appear outside of the {\cf let}
expression.
\end{entry}

\section{Bodies}
\label{bodiessection}

\index{body}The \hyper{body} of a \ide{lambda}, \ide{let}, \ide{let*},
\ide{let-values}, \ide{let*-values}, \ide{letrec}, or \ide{letrec*}
expression, or that of a definition with a body
consists of zero or more definitions followed by one or more
expressions.

{\cf \hyper{definition} \ldots{} \hyperi{expression} \hyperii{expression} \ldots}

Each identifier defined by a
definition is local to the \hyper{body}.  That is, the identifier is
bound, and the region of the binding is the
entire \hyper{body} (see section~\ref{variablesection}).

Example:
%
\begin{scheme}
(let ((x 5))
  (define foo (lambda (y) (bar x y)))
  (define bar (lambda (a b) (+ (* a b) a)))
  (foo (+ x 3)))                \ev  45%
\end{scheme}
%
When {\cf begin}, {\cf let-syntax}, or {\cf letrec-syntax} forms
occur in a body prior to the first
expression, they are spliced into the body; see section~\ref{begin}.
Some or all of the body, including portions wrapped in {\cf begin},
{\cf let-syntax}, or {\cf letrec-syntax}
forms, may be specified by a macro use
(see section~\ref{macrosection}).

An expanded \hyper{body} (see chapter~\ref{expansionchapter})
containing variable definitions can
always be converted into an equivalent {\cf letrec*}
expression.  For example, the {\cf let} expression in the above
example is equivalent to

\begin{scheme}
(let ((x 5))
  (letrec* ((foo (lambda (y) (bar x y)))
            (bar (lambda (a b) (+ (* a b) a))))
    (foo (+ x 3))))%
\end{scheme}

\section{Expressions}
\label{expressionsection}

The entries in this section describe the expressions of the \rsixlibrary{base}
library, which may occur in the position of the \hyper{expression}
syntactic variable in addition to the primitive
expression types as described in
section~\ref{primitiveexpressionsection}.

\subsection{Quotation}\unsection
\label{quotesection}

\begin{entry}{%
\proto{quote}{ \hyper{datum}}{\exprtype}}

\syntax \hyper{Datum} should be a syntactic datum.

\semantics
{\cf (quote \hyper{datum})} evaluates to the datum value
represented by \hyper{datum}
(see
section~\ref{datumsyntaxsection}).  This notation is used to include
constants.

\begin{scheme}%
(quote a)                     \ev  a
(quote \sharpsign(a b c))     \ev  \#(a b c)
(quote (+ 1 2))               \ev  (+ 1 2)%
\end{scheme}

As noted in section~\ref{abbreviationsection}, {\cf (quote \hyper{datum})}
may be abbreviated as \singlequote\hyper{datum}:

\begin{scheme}
'"abc"               \ev  "abc"
'145932              \ev  145932
'a                   \ev  a
'\#(a b c)           \ev  \#(a b c)
'()                  \ev  ()
'(+ 1 2)             \ev  (+ 1 2)
'(quote a)           \ev  (quote a)
''a                  \ev  (quote a)%
\end{scheme}

As noted in section~\ref{storagemodel}, constants are immutable.

\begin{note}
  Different constants that are the value of a {\cf quote} expression may
  share the same locations.
\end{note}
\end{entry}

\subsection{Procedures}\unsection
\label{lamba}

\begin{entry}{%
\proto{lambda}{ \hyper{formals} \hyper{body}}{\exprtype}}

\syntax
\hyper{Formals} must be a formal parameter list as described below,
and \hyper{body} must be as described in section~\ref{bodiessection}.

\semantics
\vest A \lambdaexp{} evaluates to a procedure.  The environment in
effect when the \lambdaexp{} is evaluated is remembered as part of the
procedure.  When the procedure is later called with some 
arguments, the environment in which the \lambdaexp{} was evaluated is
extended by binding the variables in the parameter list to
fresh locations, and the resulting argument values are stored
in those locations.  Then, the expressions in the body of the \lambdaexp{}
(which may contain definitions and thus represent a {\cf
  letrec*} form, see section~\ref{bodiessection}) are evaluated
sequentially in the extended environment.
The results of the last expression in the body are returned as
the results of the procedure call.

\begin{scheme}
(lambda (x) (+ x x))      \ev  {\em{}a procedure}
((lambda (x) (+ x x)) 4)  \ev  8

((lambda (x)
   (define (p y)
     (+ y 1))
   (+ (p x) x))
 5) \ev 11

(define reverse-subtract
  (lambda (x y) (- y x)))
(reverse-subtract 7 10)         \ev  3

(define add4
  (let ((x 4))
    (lambda (y) (+ x y))))
(add4 6)                        \ev  10%
\end{scheme}

\hyper{Formals} must have one of the following forms:

\begin{itemize}
\item {\tt(\hyperi{variable} \dotsfoo)}:
The procedure takes a fixed number of arguments; when the procedure is
called, the arguments are stored in the bindings of the
corresponding variables.

\item \hyper{variable}:
The procedure takes any number of arguments; when the procedure is
called, the sequence of arguments is converted into a newly
allocated list, and the list is stored in the binding of the
\hyper{variable}.

\item {\tt(\hyperi{variable} \dotsfoo{} \hyper{variable$_{n}$}\ {\bf.}\
\hyper{variable$_{n+1}$})}:
If a period {\cf .} precedes the last variable, then
the procedure takes $n$ or more arguments, where $n$ is the
number of parameters before the period (there must
be at least one).
The value stored in the binding of the last variable is a
newly allocated
list of the arguments left over after all the other 
arguments have been matched up against the other parameters.
\end{itemize}

\begin{scheme}
((lambda x x) 3 4 5 6)          \ev  (3 4 5 6)
((lambda (x y . z) z)
 3 4 5 6)                       \ev  (5 6)%
\end{scheme}

Any \hyper{variable} must not appear more than once in
\hyper{formals}.
\end{entry}


\subsection{Conditionals}\unsection

\begin{entry}{%
\proto{if}{ \hyper{test} \hyper{consequent} \hyper{alternate}}{\exprtype}
\rproto{if}{ \hyper{test} \hyper{consequent}}{\exprtype}}  %\/ if hyper = italic

\syntax
\hyper{Test}, \hyper{consequent}, and \hyper{alternate} must be 
expressions.

\semantics
An {\cf if} expression is evaluated as follows: first,
\hyper{test} is evaluated.  If it yields a true value\index{true} (see
section~\ref{booleanvaluessection}), then \hyper{consequent} is evaluated and
its values are returned.  Otherwise \hyper{alternate} is evaluated and its
values are returned.  If \hyper{test} yields \schfalse{} and no
\hyper{alternate} is specified, then the result of the expression \isunspecified.

\begin{scheme}
(if (> 3 2) 'yes 'no)           \ev  yes
(if (> 2 3) 'yes 'no)           \ev  no
(if (> 3 2)
    (- 3 2)
    (+ 3 2))                    \ev  1
(if \#f \#f)                    \ev \theunspecified%
\end{scheme}

The \hyper{consequent} and \hyper{alternate} expressions are in
tail context if the {\cf if} expression itself is; see
section~\ref{basetailcontextsection}.
\end{entry}


\subsection{Assignments}\unsection
\label{assignment}

\begin{entry}{%
\proto{set!}{ \hyper{variable} \hyper{expression}}{\exprtype}}

\hyper{Expression} is evaluated, and the resulting value is stored in
the location to which \hyper{variable} is bound.  \hyper{Variable} must
be bound either in some region\index{region} enclosing the {\cf set!}\ expression
or at the top level.  The result of the {\cf set!}
expression \isunspecified.

\begin{scheme}
(let ((x 2))
  (+ x 1)
  (set! x 4)
  (+ x 1)) \ev  5%
\end{scheme}

It is a syntax violation if \hyper{variable} refers to an
immutable binding.

\begin{note}
  The identifier {\cf set!} is exported with level $1$ as well.  See
  section~\ref{identifier-syntax}.
\end{note}
\end{entry}

\subsection{Derived conditionals}\unsection

\begin{entry}{%
\proto{cond}{ \hyperi{cond clause} \hyperii{cond clause} \dotsfoo}{\exprtype}
\litproto{=>}
\litproto{else}}

\syntax
Each \hyper{cond clause} must be of the form
\begin{scheme}
(\hyper{test} \hyperi{expression} \dotsfoo)%
\end{scheme}
where \hyper{test} is an expression.  Alternatively, a \hyper{cond clause} may be
of the form
\begin{scheme}
(\hyper{test} => \hyper{expression})%
\end{scheme}
The last \hyper{cond clause} may be
an ``{\cf else} clause'', which has the form
\begin{scheme}
(else \hyperi{expression} \hyperii{expression} \dotsfoo)\rm.%
\end{scheme}

\semantics
A {\cf cond} expression is evaluated by evaluating the \hyper{test}
expressions of successive \hyper{cond clause}s in order until one of them
evaluates to a true value\index{true} (see
section~\ref{booleanvaluessection}).  When a \hyper{test} evaluates to a true
value, then the remaining \hyper{expression}s in its \hyper{cond clause} are
evaluated in order, and the results of the last \hyper{expression} in the
\hyper{cond clause} are returned as the results of the entire {\cf cond}
expression.  If the selected \hyper{cond clause} contains only the
\hyper{test} and no \hyper{expression}s, then the value of the
\hyper{test} is returned as the result.  If the selected \hyper{cond clause} uses the
\ide{=>} alternate form, then the \hyper{expression} is evaluated.
Its value must be a procedure.  This procedure should accept one argument; it is
called on the value of the \hyper{test} and the values returned by this
procedure are returned by the {\cf cond} expression.
If all \hyper{test}s evaluate
to \schfalse, and there is no {\cf else} clause, then 
the conditional expression returns \unspecifiedreturn; if there is an {\cf else}
clause, then its \hyper{expression}s are evaluated, and the values of
the last one are returned.

\begin{scheme}
(cond ((> 3 2) 'greater)
      ((< 3 2) 'less))         \ev  greater%

(cond ((> 3 3) 'greater)
      ((< 3 3) 'less)
      (else 'equal))            \ev  equal%

(cond ('(1 2 3) => cadr)
      (else \schfalse{}))         \ev  2%
\end{scheme}

For a \hyper{cond clause} of one of the following forms
%
\begin{scheme}
(\hyper{test} \hyperi{expression} \dotsfoo)
(else \hyperi{expression} \hyperii{expression} \dotsfoo)%
\end{scheme}
%
the last \hyper{expression} is in tail context if the {\cf cond} form
itself is.  For a \hyper{cond clause} of the form
\begin{scheme}
(\hyper{test} => \hyper{expression})%
\end{scheme}
the (implied) call to the procedure that results from the evaluation
of \hyper{expression} is in a tail context if the {\cf cond} form
itself is. See section~\ref{basetailcontextsection}.

A sample definition of {\cf cond} in terms of simpler forms is in
appendix~\ref{derivedformsappendix}.
\end{entry}


\begin{entry}{%
\proto{case}{ \hyper{key} \hyperi{case clause} \hyperii{case clause} \dotsfoo}{\exprtype}}

\syntax
\hyper{Key} must be an expression.  Each \hyper{case clause} must have one of
the following forms:
\begin{scheme}
((\hyperi{datum} \dotsfoo) \hyperi{expression} \hyperii{expression} \dotsfoo)
(else \hyperi{expression} \hyperii{expression} \dotsfoo)%
\end{scheme}
\schindex{else}
The second form, which specifies an ``{\cf else} clause'',
may only appear as the last \hyper{case clause}.
Each \hyper{datum} is an external representation of some object.
The data represented by the \hyper{datum}s need not be distinct.

\semantics
A {\cf case} expression is evaluated as follows.  \hyper{Key} is
evaluated and its result is compared using {\cf eqv?} (see
section~\ref{eqv?}) against the data
represented by the \hyper{datum}s of each \hyper{case clause} in turn, proceeding
in order from left to right through the set of clauses.  If the
result of evaluating \hyper{key} is equivalent to a datum of a \hyper{case clause}, the
corresponding \hyper{expression}s are evaluated from left
to right and the results of the last expression in the \hyper{case clause} are
returned as the results of the {\cf case} expression.  Otherwise, the
comparison process continues.  If the result of
evaluating \hyper{key} is different from every datum in each set, then if
there is an {\cf else} clause its expressions are evaluated and the
results of the last are the results of the {\cf case} expression;
otherwise the {\cf case} expression returns \unspecifiedreturn.

\begin{scheme}
(case (* 2 3)
  ((2 3 5 7) 'prime)
  ((1 4 6 8 9) 'composite))     \ev  composite
(case (car '(c d))
  ((a) 'a)
  ((b) 'b))                     \ev  \theunspecified
(case (car '(c d))
  ((a e i o u) 'vowel)
  ((w y) 'semivowel)
  (else 'consonant))            \ev  consonant%
\end{scheme}

The last \hyper{expression} of a \hyper{case clause} is in tail
context if the {\cf case} expression itself is; see
section~\ref{basetailcontextsection}.

% A sample definition of {\cf case} in terms of simpler forms is in
% appendix~\ref{derivedformsappendix}.
\end{entry}


\begin{entry}{%
\proto{and}{ \hyperi{test} \dotsfoo}{\exprtype}}

\syntax The \hyper{test}s must be expressions.

\semantics If there are no \hyper{test}s, \schtrue{} is returned.
Otherwise, the \hyper{test} expressions are evaluated from left to
right until a \hyper{test} returns \schfalse{} or the last
\hyper{test} is reached.  In the former case, the {\cf and} expression
returns \schfalse{} without evaluating the remaining expressions.
In the latter case, the last expression is evaluated and its values
are returned.

\begin{scheme}
(and (= 2 2) (> 2 1))           \ev  \schtrue
(and (= 2 2) (< 2 1))           \ev  \schfalse
(and 1 2 'c '(f g))             \ev  (f g)
(and)                           \ev  \schtrue%
\end{scheme}

The {\cf and} keyword could be defined in terms of {\cf if} using {\cf
  syntax-rules} (see section~\ref{syntaxrulessection}) as follows:

\begin{scheme}
(define-syntax \ide{and}
  (syntax-rules ()
    ((and) \sharpfoo{t})
    ((and test) test)
    ((and test1 test2 ...)
     (if test1 (and test2 ...) \sharpfoo{f}))))%
\end{scheme}

The last \hyper{test} expression is in tail context if the {\cf and}
expression itself is; see section~\ref{basetailcontextsection}.
\end{entry}


\begin{entry}{%
\proto{or}{ \hyperi{test} \dotsfoo}{\exprtype}}

\syntax The \hyper{test}s must be expressions.

\semantics If there are no \hyper{test}s, \schfalse{} is returned.
Otherwise, the \hyper{test} expressions are evaluated from left to
right until a \hyper{test} returns a true value \var{val}
(see section~\ref{booleanvaluessection}) or the last
\hyper{test} is reached.  In the former case, the {\cf or} expression
returns \var{val} without evaluating the remaining expressions.
In the latter case, the last expression is evaluated and its values
are returned.

\begin{scheme}
(or (= 2 2) (> 2 1))            \ev  \schtrue
(or (= 2 2) (< 2 1))            \ev  \schtrue
(or \schfalse \schfalse \schfalse) \ev  \schfalse
(or '(b c) (/ 3 0))             \ev  (b c)%
\end{scheme}

The {\cf or} keyword could be defined in terms of {\cf if} using {\cf
  syntax-rules} (see section~\ref{syntaxrulessection}) as follows:

\begin{scheme}
(define-syntax \ide{or}
  (syntax-rules ()
    ((or) \sharpfoo{f})
    ((or test) test)
    ((or test1 test2 ...)
     (let ((x test1))
       (if x x (or test2 ...))))))%
\end{scheme}

The last \hyper{test} expression is in tail context if the {\cf or}
expression itself is; see section~\ref{basetailcontextsection}.
\end{entry}


\subsection{Binding constructs}

The binding constructs described in this section
create local bindings for variables that are visible only in a
delimited region.  The syntax of the 
constructs
{\cf let}, {\cf let*}, {\cf letrec}, and {\cf letrec*}
 is identical, but they differ in the regions\index{region}
(see section~\ref{variablesection}) they establish
for their variable bindings and in the order in which the values for
the bindings are computed.  In a {\cf let} expression, the initial
values are computed before any of the variables become bound; in a
{\cf let*} expression, the bindings and evaluations are performed
sequentially.  In a {\cf letrec} or {\cf letrec*}
expression, all the bindings are in
effect while their initial values are being computed, thus allowing
mutually recursive definitions.  In a {\cf letrec} expression, the
initial values are computed before being assigned to the variables;
in a {\cf letrec*}, the evaluations and assignments are performed
sequentially.

In addition, the binding constructs {\cf let-values} and {\cf
  let*-values} generalize {\cf let} and {\cf let*} to allow multiple
variables to be bound to the results of expressions that evaluate to
multiple values.
They are analogous to {\cf let} and {\cf let*} in the
way they establish regions: in a {\cf let-values} expression, the
initial values are computed before any of the variables become bound;
in a {\cf let*-values} expression, the bindings are performed
sequentially. 

Sample definitions of all the binding forms of this section in terms
of simpler forms are in appendix~\ref{derivedformsappendix}.

\begin{entry}{%
\proto{let}{ \hyper{bindings} \hyper{body}}{\exprtype}}

\syntax
\hyper{Bindings} must have the form
\begin{scheme}
((\hyperi{variable} \hyperi{init}) \dotsfoo)\rm,%
\end{scheme}
where each \hyper{init} is an expression, and \hyper{body} 
is as described in section~\ref{bodiessection}.  
Any variable must not appear more than once in the \hyper{variable}s.

\semantics
The \hyper{init}s are evaluated in the current environment (in some
unspecified order), the \hyper{variable}s are bound to fresh locations
holding the results, the \hyper{body} is evaluated in the extended
environment, and the values of the last expression of \hyper{body}
are returned.  Each binding of a \hyper{variable} has \hyper{body} as its
region.\index{region}

\begin{scheme}
(let ((x 2) (y 3))
  (* x y))                      \ev  6

(let ((x 2) (y 3))
  (let ((x 7)
        (z (+ x y)))
    (* z x)))                   \ev  35%
\end{scheme}

See also named {\cf let}, section \ref{namedlet}.

\end{entry}


\begin{entry}{%
\proto{let*}{ \hyper{bindings} \hyper{body}}{\exprtype}}\nobreak

\nobreak
\syntax
\hyper{Bindings} must have the form
\begin{scheme}
((\hyperi{variable} \hyperi{init}) \dotsfoo)\rm,%
\end{scheme}
where each \hyper{init} is an expression, and \hyper{body} 
is as described in section~\ref{bodiessection}.

\semantics
The {\cf let*} form is similar to {\cf let}, but the \hyper{init}s are
evaluated and bindings created sequentially from left to right, with
the region\index{region} of each binding including the bindings to
its right as well as \hyper{body}.  Thus the second \hyper{init} is evaluated
in an environment in which the first binding is visible and initialized,
and so on.

\begin{scheme}
(let ((x 2) (y 3))
  (let* ((x 7)
         (z (+ x y)))
    (* z x)))             \ev  70%
\end{scheme}

\begin{note}
  While the variables bound by a {\cf let} expression must be distinct,
  the variables bound by a {\cf let*} expression need not be distinct.
\end{note}
\end{entry}

\begin{entry}{%
\proto{letrec}{ \hyper{bindings} \hyper{body}}{\exprtype}}

\syntax
\hyper{Bindings} must have the form
\begin{scheme}
((\hyperi{variable} \hyperi{init}) \dotsfoo)\rm,%
\end{scheme}
where each \hyper{init} is an expression, and \hyper{body} 
is as described in section~\ref{bodiessection}.  Any
variable must not appear more than once in the
\hyper{variable}s.

\semantics
The \hyper{variable}s are bound to fresh locations, the \hyper{init}s
are evaluated in the resulting environment (in
some unspecified order), each \hyper{variable} is assigned to the result
of the corresponding \hyper{init}, the \hyper{body} is evaluated in the
resulting environment, and the values of the last expression in
\hyper{body} are returned.  Each binding of a \hyper{variable} has the
entire {\cf letrec} expression as its region\index{region}, making it possible to
define mutually recursive procedures.

\begin{scheme}
%(letrec ((x 2) (y 3))
%  (letrec ((foo (lambda (z) (+ x y z))) (x 7))
%    (foo 4)))                   \ev  14
%
(letrec ((even?
          (lambda (n)
            (if (zero? n)
                \schtrue
                (odd? (- n 1)))))
         (odd?
          (lambda (n)
            (if (zero? n)
                \schfalse
                (even? (- n 1))))))
  (even? 88))   
                \ev  \schtrue%
\end{scheme}

It should be possible
to evaluate each \hyper{init} without assigning or referring to the
value of any \hyper{variable}.  In the most
common uses of {\cf letrec}, all the \hyper{init}s are \lambdaexp{}s
and the restriction is satisfied automatically.
Another restriction is that the continuation of each \hyper{init} should not be invoked
more than once.

\implresp Implementations must detect references to a \hyper{variable} during the
evaluation of the \hyper{init} expressions (using one particular
evaluation order and order of evaluating the \hyper{init} expressions).
If an implementation detects such a violation of the
restriction, it must raise an exception with condition type
{\cf\&assertion}.
Implementations may or may not detect that the continuation of each
\hyper{init} is invoked more than once.  However, if the
implementation detects this, it must raise an exception with condition
type {\cf\&assertion}.
\end{entry}

\begin{entry}{%
\proto{letrec*}{ \hyper{bindings} \hyper{body}}{\exprtype}}

\syntax
\hyper{Bindings} must have the form
\begin{scheme}
((\hyperi{variable} \hyperi{init}) \dotsfoo)\rm,%
\end{scheme}
where each \hyper{init} is an expression, and \hyper{body} 
is as described in section~\ref{bodiessection}. 
Any variable must not appear more than once in the
\hyper{variable}s.

\semantics
The \hyper{variable}s are bound to fresh locations,
each \hyper{variable} is assigned in left-to-right order to the
result of evaluating the corresponding \hyper{init}, the \hyper{body} is
evaluated in the resulting environment, and the values of the last
expression in \hyper{body} are returned. 
Despite the left-to-right evaluation and assignment order, each binding of
a \hyper{variable} has the entire {\cf letrec*} expression as its
region\index{region}, making it possible to define mutually recursive
procedures.

\begin{scheme}
(letrec* ((p
           (lambda (x)
             (+ 1 (q (- x 1)))))
          (q
           (lambda (y)
             (if (zero? y)
                 0
                 (+ 1 (p (- y 1))))))
          (x (p 5))
          (y x))
  y)
                \ev  5%
\end{scheme}

It must be possible
to evaluate each \hyper{init} without assigning or referring to the value
of the corresponding \hyper{variable} or the \hyper{variable} of any of
the bindings that follow it in \hyper{bindings}.
Another restriction is that the continuation of each \hyper{init} should not be invoked
more than once.

\implresp Implementations must, during the evaluation of an
\hyper{init} expression, detect references to the
value of the corresponding \hyper{variable} or the \hyper{variable} of
any of the bindings that follow it in \hyper{bindings}.
If an implementation detects such a
violation of the restriction, it must raise an exception with
condition type {\cf\&assertion}.  Implementations may or may not
detect that the continuation of each \hyper{init} is invoked more than
once.  However, if the implementation detects this, it must raise an
exception with condition type {\cf\&assertion}.
\end{entry}

\begin{entry}{%
\proto{let-values}{ \hyper{mv-bindings} \hyper{body}}{\exprtype}}

\syntax
\hyper{Mv-bindings} must have the form
\begin{scheme}
((\hyperi{formals} \hyperi{init}) \dotsfoo)\rm,%
\end{scheme}
where each \hyper{init} is an expression, and \hyper{body} 
is as described in section~\ref{bodiessection}.  
Any variable must not appear more
than once in the set of \hyper{formals}.

\semantics The \hyper{init}s are evaluated in the current environment
(in some unspecified order), and the variables occurring in the
\hyper{formals} are bound to fresh locations containing the values
returned by the \hyper{init}s, where the \hyper{formals} are matched
to the return values in the same way that the \hyper{formals} in a
\lambdaexp{} are matched to the arguments in a procedure call.
Then, the \hyper{body} is evaluated in the extended environment, and the
values of the last expression of \hyper{body} are returned.
Each binding of a variable has \hyper{body} as its
region.\index{region}
If the \hyper{formals} do not match, an exception with condition type
{\cf\&assertion} is raised.

\begin{scheme}
(let-values (((a b) (values 1 2))
             ((c d) (values 3 4)))
  (list a b c d)) \ev (1 2 3 4)

(let-values (((a b . c) (values 1 2 3 4)))
  (list a b c))            \ev (1 2 (3 4))

(let ((a 'a) (b 'b) (x 'x) (y 'y))
  (let-values (((a b) (values x y))
               ((x y) (values a b)))
    (list a b x y)))       \ev (x y a b)%
\end{scheme}
\end{entry}

\begin{entry}{%
\proto{let*-values}{ \hyper{mv-bindings} \hyper{body}}{\exprtype}}

\syntax
\hyper{Mv-bindings} must have the form
\begin{scheme}
((\hyperi{formals} \hyperi{init}) \dotsfoo)\rm,%
\end{scheme}
where each \hyper{init} is an expression, and \hyper{body} 
is as described in section~\ref{bodiessection}.
In each \hyper{formals}, any variable must not appear more than once.

\semantics
The {\cf let*-values} form is similar to {\cf let-values}, but the \hyper{init}s are
evaluated and bindings created sequentially from left to right, with
the region\index{region} of the bindings of each \hyper{formals} including
the bindings to its right as well as \hyper{body}. 
Thus the second \hyper{init} is evaluated in an environment in which the
bindings of the first \hyper{formals} is visible and initialized, and so
on.

\begin{scheme}
(let ((a 'a) (b 'b) (x 'x) (y 'y))
  (let*-values (((a b) (values x y))
                ((x y) (values a b)))
    (list a b x y)))  \ev (x y x y)%
\end{scheme}

\begin{note}
  While all of the variables bound by a {\cf let-values} expression
  must be distinct, the variables bound by different \hyper{formals} of a
  {\cf let*-values} expression need not be distinct.
\end{note}

\end{entry}

\subsection{Sequencing}\unsection

\begin{entry}{%
\proto{begin}{ \hyper{form} \dotsfoo}{\exprtype}
\rproto{begin}{ \hyper{expression} \hyper{expression} \dotsfoo}{\exprtype}}

The \hyper{begin} keyword has two different roles, depending on its
context:
\begin{itemize}
\item It may appear as a form in a \hyper{body} (see
  section~\ref{bodiessection}), \hyper{library body} (see
  section~\ref{librarybodysection}), or \hyper{top-level body} (see
  chapter~\ref{programchapter}), or directly nested in a {\cf begin}
  form that appears in a body.  In this case, the {\cf begin} form
  must have the shape specified in the first header line.  This use of
  {\cf begin} acts as a \defining{splicing} form---the forms inside
  the \hyper{body} are spliced into the surrounding body, as if the
  {\cf begin} wrapper were not actually present.
  
  A {\cf begin} form in a \hyper{body} or \hyper{library body} must
  be non-empty if it appears after the first \hyper{expression}
  within the body.
\item It may appear as an ordinary expression and must have the shape
  specified in the second header line.  In this case, the
  \hyper{expression}s are evaluated sequentially from left to right,
  and the values of the last \hyper{expression} are returned.
  This expression type is used to sequence side effects such as
  assignments or input
  and output.
\end{itemize}

\begin{scheme}
(define x 0)

(begin (set! x 5)
       (+ x 1))                  \ev  6

(begin (display "4 plus 1 equals ")
       (display (+ 4 1)))      \ev  \unspecified
 \>{\em and prints}  4 plus 1 equals 5%
\end{scheme}
\end{entry}

\section{Equivalence predicates}
\label{equivalencesection}

A \defining{predicate} is a procedure that always returns a boolean
value (\schtrue{} or \schfalse).  An \defining{equivalence predicate} is
the computational analogue of a mathematical equivalence relation (it is
symmetric, reflexive, and transitive).  Of the equivalence predicates
described in this section, {\cf eq?}\ is the finest or most
discriminating, and {\cf equal?}\ is the coarsest.  The {\cf eqv?} predicate is
slightly less discriminating than {\cf eq?}.  \todo{Pitman doesn't like
this paragraph.  Lift the discussion from the Maclisp manual.  Explain
why there's more than one predicate.}


\begin{entry}{%
\proto{eqv?}{ \vari{obj} \varii{obj}}{procedure}}

The {\cf eqv?} procedure defines a useful equivalence relation on objects.
Briefly, it returns \schtrue{} if \vari{obj} and \varii{obj} should
normally be regarded as the same object and \schfalse{} otherwise.  This relation is left slightly
open to interpretation, but the following partial specification of
{\cf eqv?} must hold for all implementations.

The {\cf eqv?} procedure returns \schtrue{} if one of the following holds:

\begin{itemize}
\item \vari{Obj} and \varii{obj} are both booleans and are the same
  according to the {\cf boolean=?} procedure (section~\ref{boolean=?}).

\item \vari{Obj} and \varii{obj} are both symbols and are the same
  according to the {\cf symbol=?} procedure (section~\ref{symbol=?}).

\item \vari{Obj} and \varii{obj} are both exact\index{exact} number objects
  and are numerically equal (see {\cf =}, 
  section~\ref{genericarithmeticsection}).

\item \vari{Obj} and \varii{obj} are both inexact\index{inexact}
  number objects, are numerically
  equal (see {\cf =}, section~\ref{genericarithmeticsection}), and
  yield the same results (in the sense of {\cf eqv?}) when passed
  as arguments to any other procedure that can be defined
  as a finite composition of Scheme's standard arithmetic
  procedures.

\item \vari{Obj} and \varii{obj} are both characters and are the same
character according to the {\cf char=?} procedure
(section~\ref{charactersection}).

\item Both \vari{obj} and \varii{obj} are the empty list.

\item \vari{Obj} and \varii{obj} are objects such as pairs, vectors, bytevectors
  (library chapter~\extref{lib:bytevectorschapter}{Bytevectors}),
  strings, hashtables, records (library
  chapter~\extref{lib:recordschapter}{Records}), ports (library
  section~\extref{lib:portsiosection}{Port I/O}), or hashtables
  (library chapter~\extref{lib:hashtablechapter}{Hash tables}) that
  refer to the same locations in the store (section~\ref{storagemodel}).

\item \vari{Obj} and \varii{obj} are record-type descriptors that are
  specified to be {\cf eqv?} in library
  section~\extref{lib:recordsproceduralsection}{Procedural layer}.
\end{itemize}

The {\cf eqv?} procedure returns \schfalse{} if one of the following holds:

\begin{itemize}
\item \vari{Obj} and \varii{obj} are of different types
(section~\ref{disjointness}).

\item \vari{Obj} and \varii{obj} are booleans for which the {\cf
    boolean=?} procedure returns \schfalse.

\item \vari{Obj} and \varii{obj} are symbols for which the {\cf
    symbol=?} procedure returns \schfalse.

\item One of \vari{obj} and \varii{obj} is an exact number object but the other is
        an inexact number object.

\item \vari{Obj} and \varii{obj} are rational number objects for which the {\cf =} procedure
  returns \schfalse{}.

\item \vari{Obj} and \varii{obj} yield different results (in the sense of
  {\cf eqv?}) when passed as arguments to any other procedure
  that can be defined as a finite composition of Scheme's
  standard arithmetic procedures.

\item \vari{Obj} and \varii{obj} are characters for which the {\cf char=?}
  procedure returns \schfalse{}.

\item One of \vari{obj} and \varii{obj} is the empty list, but the other is not.

\item \vari{Obj} and \varii{obj} are objects such as pairs, vectors,
  bytevectors (library
  chapter~\extref{lib:bytevectorschapter}{Bytevectors}), strings,
  records (library
  chapter~\extref{lib:recordschapter}{Records}), ports (library
  section~\extref{lib:portsiosection}{Port I/O}), or hashtables
  (library chapter~\extref{lib:hashtablechapter}{Hashtables}) that
  refer to distinct locations.

\item \vari{Obj} and \varii{obj} are pairs, vectors, strings, or
  records, or hashtables, where the applying the same accessor (i.e.\
  {\cf car}, {\cf cdr}, {\cf vector-ref}, {\cf string-ref}, or record
  accessors) to both yields results for which {\cf eqv?} returns
  \schfalse.

\item \vari{Obj} and \varii{obj} are procedures that would behave differently
(return different values or have different side effects) for some arguments.

\end{itemize}

\begin{note}
  The {\cf eqv?} procedure returning \schtrue{} when \vari{obj} and
  \varii{obj} are number objects does not imply that {\cf =} would also
  return \schtrue{} when called with \vari{obj} and \varii{obj} as
  arguments.
\end{note}


\begin{scheme}
(eqv? 'a 'a)                     \ev  \schtrue
(eqv? 'a 'b)                     \ev  \schfalse
(eqv? 2 2)                       \ev  \schtrue
(eqv? '() '())                   \ev  \schtrue
(eqv? 100000000 100000000)       \ev  \schtrue
(eqv? (cons 1 2) (cons 1 2))     \ev  \schfalse
(eqv? (lambda () 1)
      (lambda () 2))             \ev  \schfalse
(eqv? \#f 'nil)                  \ev  \schfalse%
\end{scheme}

The following examples illustrate cases in which the above rules do
not fully specify the behavior of {\cf eqv?}.  All that can be said
about such cases is that the value returned by {\cf eqv?} must be a
boolean.

\begin{scheme}
(let ((p (lambda (x) x)))
  (eqv? p p))                    \ev  \unspecified
(eqv? "" "")             \ev  \unspecified
(eqv? '\#() '\#())         \ev  \unspecified
(eqv? (lambda (x) x)
      (lambda (x) x))    \ev  \unspecified
(eqv? (lambda (x) x)
      (lambda (y) y))    \ev  \unspecified
(eqv? +nan.0 +nan.0)             \ev \unspecified%
\end{scheme}

The next set of examples shows the use of {\cf eqv?}\ with procedures
that have local state.  Calls to {\cf gen-counter} must return a
distinct procedure every time, since each procedure has its own
internal counter.  Calls to {\cf gen-loser} return procedures that
behave equivalently when called.  However, {\cf eqv?} may
not detect this equivalence.

\begin{scheme}
(define gen-counter
  (lambda ()
    (let ((n 0))
      (lambda () (set! n (+ n 1)) n))))
(let ((g (gen-counter)))
  (eqv? g g))           \ev  \unspecified
(eqv? (gen-counter) (gen-counter))
                        \ev  \schfalse
(define gen-loser
  (lambda ()
    (let ((n 0))
      (lambda () (set! n (+ n 1)) 27))))
(let ((g (gen-loser)))
  (eqv? g g))           \ev  \unspecified
(eqv? (gen-loser) (gen-loser))
                        \ev  \unspecified

(letrec ((f (lambda () (if (eqv? f g) 'both 'f)))
         (g (lambda () (if (eqv? f g) 'both 'g))))
  (eqv? f g)) \ev  \unspecified

(letrec ((f (lambda () (if (eqv? f g) 'f 'both)))
         (g (lambda () (if (eqv? f g) 'g 'both))))
  (eqv? f g)) \ev  \schfalse%
\end{scheme}

Implementations may
share structure between constants where appropriate.
Furthermore, a constant may be copied at any time by the implementation so
as to exist simultaneously in different sets of locations, as noted in
section~\ref{quote}.
Thus the value of {\cf eqv?} on constants is sometimes
implementation-dependent.

\begin{scheme}
(eqv? '(a) '(a))                 \ev  \unspecified
(eqv? "a" "a")                   \ev  \unspecified
(eqv? '(b) (cdr '(a b)))         \ev  \unspecified
(let ((x '(a)))
  (eqv? x x))                    \ev  \schtrue%
\end{scheme}
\end{entry}


\begin{entry}{%
\proto{eq?}{ \vari{obj} \varii{obj}}{procedure}}

The {\cf eq?} predicate is similar to {\cf eqv?}\ except that in some cases it is
capable of discerning distinctions finer than those detectable by
{\cf eqv?}.

The {\cf eq?}\ and {\cf eqv?} predicates are guaranteed to have the
same behavior on symbols, booleans, the empty list, pairs, procedures,
non-empty strings, bytevectors, and vectors, and records.  The
behavior of {\cf eq?} on number objects and characters is
implementation-dependent, but it always returns either \schtrue{} or
\schfalse{}, and returns \schtrue{} only when {\cf eqv?}\ would also
return \schtrue.  The {\cf eq?} predicate may also behave differently
from {\cf eqv?} on empty vectors, empty bytevectors, and empty strings.

\begin{scheme}
(eq? 'a 'a)                     \ev  \schtrue
(eq? '(a) '(a))                 \ev  \unspecified
(eq? (list 'a) (list 'a))       \ev  \schfalse
(eq? "a" "a")                   \ev  \unspecified
(eq? "" "")                     \ev  \unspecified
(eq? '() '())                   \ev  \schtrue
(eq? 2 2)                       \ev  \unspecified
(eq? \#\backwhack{}A \#\backwhack{}A) \ev  \unspecified
(eq? car car)                   \ev  \schtrue
(let ((n (+ 2 3)))
  (eq? n n))      \ev  \unspecified
(let ((x '(a)))
  (eq? x x))      \ev  \schtrue
(let ((x '\#()))
  (eq? x x))      \ev  \unspecified
(let ((p (lambda (x) x)))
  (eq? p p))      \ev  \unspecified%
\end{scheme}

\todo{Needs to be explained better above.  How can this be made to be
not confusing?  A table maybe?}

\end{entry}

\begin{entry}{%
\proto{equal?}{ \vari{obj} \varii{obj}}{procedure}}

The {\cf equal?}  predicate returns \schtrue{} if and only if the
(possibly infinite) unfoldings of its arguments into regular trees are
equal as ordered trees.

The {\cf equal?} predicate treats pairs and vectors
as nodes with outgoing edges, uses {\cf
  string=?} to compare strings, uses {\cf
  bytevector=?} to compare bytevectors (see library chapter~\extref{lib:bytevectorschapter}{Bytevectors}),
  and uses {\cf eqv?} to compare other nodes.

\begin{scheme}
(equal? 'a 'a)                  \ev  \schtrue
(equal? '(a) '(a))              \ev  \schtrue
(equal? '(a (b) c)
        '(a (b) c))             \ev  \schtrue
(equal? "abc" "abc")            \ev  \schtrue
(equal? 2 2)                    \ev  \schtrue
(equal? (make-vector 5 'a)
        (make-vector 5 'a))     \ev  \schtrue
(equal? '\#vu8(1 2 3 4 5)
        (u8-list->bytevector
         '(1 2 3 4 5))          \ev  \schtrue
(equal? (lambda (x) x)
        (lambda (y) y))  \ev  \unspecified

(let* ((x (list 'a))
       (y (list 'a))
       (z (list x y)))
  (list (equal? z (list y x))
        (equal? z (list x x))))             \lev  (\schtrue{} \schtrue{})%
\end{scheme}

\begin{note}
  The {\cf equal?} procedure must always terminate, even if its
  arguments contain cycles.
\end{note}

\end{entry}

\section{Procedure predicate}

\begin{entry}{%
\proto{procedure?}{ obj}{procedure}}

Returns \schtrue{} if \var{obj} is a procedure, otherwise returns \schfalse.

\begin{scheme}
(procedure? car)            \ev  \schtrue
(procedure? 'car)           \ev  \schfalse
(procedure? (lambda (x) (* x x)))   
                            \ev  \schtrue
(procedure? '(lambda (x) (* x x)))  
                            \ev  \schfalse%
\end{scheme}

\end{entry}

\section{Arithmetic}
\label{genericarithmeticsection}

The procedures described here implement arithmetic that is
generic over
the numerical tower described in chapter~\ref{numbertypeschapter}.
The generic procedures described in this section
accept both exact and inexact number objects as arguments,
performing coercions and selecting the appropriate operations
as determined by the numeric subtypes of their arguments.

Library chapter~\extref{lib:numberchapter}{Arithmetic} describes
libraries that define other numerical procedures.

\subsection{Propagation of exactness and inexactness}
\label{propagationsection}

The procedures listed below must return the mathematically correct exact result
provided all their arguments are exact:

\begin{scheme}
+            -            *
max          min          abs
numerator    denominator  gcd
lcm          floor        ceiling
truncate     round        rationalize
real-part    imag-part    make-rectangular%
\end{scheme}

The procedures listed below must return the correct exact result
provided all their arguments are exact, and no divisors are zero:

\begin{scheme}
/
div          mod           div-and-mod
div0         mod0          div0-and-mod0%
\end{scheme}

Moreover, the procedure {\cf expt} must return the correct exact
result provided its first argument is an exact real number object and
its second argument is an exact integer object.

The general rule is that the generic operations return the correct
exact result when all of their arguments are exact and the result is
mathematically well-defined, but return an inexact result when any
argument is inexact.  Exceptions to this rule include
{\cf sqrt}, {\cf exp}, {\cf log},
{\cf sin}, {\cf cos}, {\cf tan},
{\cf asin}, {\cf acos}, {\cf atan},
{\cf expt}, {\cf make-polar}, {\cf magnitude}, and {\cf angle}, which
may (but are not required to) return inexact results even when
given exact arguments, as indicated in the specification of these
procedures.

One general exception to the rule above is that an implementation may
return an exact result despite inexact arguments if that exact result
would be the correct result for all possible substitutions of exact
arguments for the inexact ones.  An example is {\cf (* 1.0 0)} which
may return either {\cf 0} (exact) or {\cf 0.0} (inexact).

\subsection{Representability of infinities and NaNs}
\label{infinitiesnanssection}

The specification of the numerical operations is written as though
infinities and NaNs are representable, and specifies many operations
with respect to these number objects in ways that are consistent with the
IEEE-754 standard for binary floating-point arithmetic.  
An implementation of Scheme may or may not represent infinities and
NaNs; however,
an implementation must raise a continuable exception with
condition type {\cf\&no-infinities} or {\cf\&no-nans} (respectively;
see library section~\extref{lib:flonumssection}{Flonums})
whenever it is unable to represent an infinity or NaN as specified. 
In this case, the continuation of the exception
handler is the continuation that otherwise would have received
the infinity or NaN value.  This requirement also applies to
conversions between number objects and external representations, including
the reading of program source code.

\subsection{Semantics of common operations}

Some operations are the semantic basis for several arithmetic
procedures.  The behavior of these operations is described in this
section for later reference.

\subsubsection{Integer division}
\label{integerdivision}

Scheme's operations for performing integer
division rely on mathematical operations $\mathrm{div}$,
$\mathrm{mod}$, $\mathrm{div}_0$, and
$\mathrm{mod}_0$, that are defined as follows:

$\mathrm{div}$, $\mathrm{mod}$, $\mathrm{div}_0$, and $\mathrm{mod}_0$
each accept two real numbers $x_1$ and $x_2$ as operands, where
$x_2$ must be nonzero.

$\mathrm{div}$ returns an integer, and $\mathrm{mod}$ returns a real.
Their results are specified by
%
\begin{eqnarray*}
x_1~\mathrm{div}~x_2 &=& n_d\\
x_1~\mathrm{mod}~x_2 &=& x_m
\end{eqnarray*}
%
where
%
\begin{displaymath}
\begin{array}{c}
x_1 = n_d \cdot x_2 + x_m\\
0 \leq x_m < |x_2|
\end{array}
\end{displaymath}
%
Examples:
\begin{eqnarray*}
123~\mathrm{div}~10    &=&  12\\
123~\mathrm{mod}~10    &=&  3\\
123~\mathrm{div}~\textrm{$-10$}   &=&  -12\\
123~\mathrm{mod}~\textrm{$-10$}   &=&  3\\
-123~\mathrm{div}~10    &=&  -13\\
-123~\mathrm{mod}~10    &=&  7\\
-123~\mathrm{div}~\textrm{$-10$}   &=&  13\\
-123~\mathrm{mod}~\textrm{$-10$}   &=&  7
\end{eqnarray*}
%
$\mathrm{div}_0$ and $\mathrm{mod}_0$ are like $\mathrm{div}$ and
$\mathrm{mod}$, except the result of $\mathrm{mod}_0$ lies within a
half-open interval centered on zero.  The results are specified by
%
\begin{eqnarray*}
x_1~\mathrm{div}_0~x_2 &=& n_d\\
x_1~\mathrm{mod}_0~x_2 &=& x_m
\end{eqnarray*}
%
where:
%
\begin{displaymath}
\begin{array}{c}
x_1 = n_d \cdot x_2 + x_m\\
-|\frac{x_2}{2}| \leq x_m < |\frac{x_2}{2}|
\end{array}
\end{displaymath}
%
Examples:
%
\begin{eqnarray*}
123~\mathrm{div}_0~10    &=&  12\\
123~\mathrm{mod}_0~10    &=&  3\\
123~\mathrm{div}_0~\textrm{$-10$}   &=&  -12\\
123~\mathrm{mod}_0~\textrm{$-10$}   &=&  3\\
-123~\mathrm{div}_0~10    &=&  -12\\
-123~\mathrm{mod}_0~10    &=&  -3\\
-123~\mathrm{div}_0~\textrm{$-10$}   &=&  12\\
-123~\mathrm{mod}_0~\textrm{$-10$}   &=&  -3
\end{eqnarray*}

\subsubsection{Transcendental functions}
\label{transcendentalfunctions}

In general, the transcendental functions $\log$, $\sin^{-1}$
(arcsine), $\cos^{-1}$ (arccosine), and $\tan^{-1}$ are multiply
defined.  The value of $\log z$ is defined to be the one whose
imaginary part lies in the range from $-\pi$ (inclusive if $-0.0$ is
distinguished, exclusive otherwise) to $\pi$ (inclusive).  $\log 0$ is
undefined.

The value of $\log z$ for non-real $z$ is defined in terms of log on real numbers as 

\begin{displaymath}
\log z = \log |z| + (\mathrm{angle}~z)i
\end{displaymath}
%
where $\mathrm{angle}~z$ is the angle of $z = a\cdot e^{ib}$ specified
as:
$$\mathrm{angle}~z = b+2\pi n$$
with $-\pi \leq \mathrm{angle}~z\leq \pi$ and $\mathrm{angle}~z =
b+2\pi n$ for some integer $n$.

With the one-argument version of $\log$ defined this way, the values
of the two-argument-version of $\log$, $\sin^{-1} z$, $\cos^{-1} z$,
$\tan^{-1} z$, and the two-argument version of $\tan^{-1}$ are
according to the following formul\ae:
\begin{eqnarray*}
\log z~b &=& \frac{\log z}{\log b}\\
\sin^{-1} z &=& -i \log (i z + \sqrt{1 - z^2})\\
\cos^{-1} z &=& \pi / 2 - \sin^{-1} z\\
\tan^{-1} z &=& (\log (1 + i z) - \log (1 - i z)) / (2 i)\\
\tan^{-1} x~y &=& \mathrm{angle}(x+ yi)
\end{eqnarray*}

The range of $\tan^{-1} x~y$ is as in the following table. The
asterisk (*) indicates that the entry applies to implementations that
distinguish minus zero.

\begin{center}
\begin{tabular}{clll}
& $y$ condition & $x$ condition & range of result $r$\\\hline
& $y = 0.0$ & $x > 0.0$ & $0.0$\\
$\ast$ & $y = +0.0$  & $x > 0.0$ & $+0.0$\\     
$\ast$ & $y = -0.0$ & $x > 0.0$ & $-0.0$\\
& $y > 0.0$ & $x > 0.0$ & $0.0 < r < \frac{\pi}{2}$\\
& $y > 0.0$ & $x = 0.0$ & $\frac{\pi}{2}$\\
& $y > 0.0$ & $x < 0.0$ & $\frac{\pi}{2} < r < \pi$\\
& $y = 0.0$ & $x < 0$ & $\pi$\\
$\ast$ & $y = +0.0$ & $x < 0.0$ & $\pi$\\
$\ast$ & $y = -0.0$ & $x < 0.0$ & $-\pi$\\      
&$y < 0.0$ & $x < 0.0$ & $-\pi< r< -\frac{\pi}{2}$\\
&$y < 0.0$ & $x = 0.0$ & $-\frac{\pi}{2}$\\
&$y < 0.0$ & $x > 0.0$ & $-\frac{\pi}{2} < r< 0.0$\\    
&$y = 0.0$ & $x = 0.0$ & undefined\\
$\ast$& $y = +0.0$ & $x = +0.0$ & $+0.0$\\
$\ast$& $y = -0.0$ & $x = +0.0$& $-0.0$\\
$\ast$& $y = +0.0$ & $x = -0.0$ & $\pi$\\
$\ast$& $y = -0.0$ & $x = -0.0$ & $-\pi$\\
$\ast$& $y = +0.0$ & $x = 0$ & $\frac{\pi}{2}$\\
$\ast$& $y = -0.0$ & $x = 0$    & $-\frac{\pi}{2}$
\end{tabular}
\end{center}

\subsection{Numerical operations}

\subsubsection{Numerical type predicates}

\begin{entry}{%
\proto{number?}{ obj}{procedure}
\proto{complex?}{ obj}{procedure}
\proto{real?}{ obj}{procedure}
\proto{rational?}{ obj}{procedure}
\proto{integer?}{ obj}{procedure}}

These numerical type predicates can be applied to any kind of
argument.  They return \schtrue{} if the object is a number object
of the named type, and \schfalse{} otherwise.
In general, if a type predicate is true of a number object then all higher
type predicates are also true of that number object.  Consequently, if a type
predicate is false of a number object, then all lower type predicates are
also false of that number object.

If \var{z} is a complex number object, then {\cf (real? \var{z})} is true if
and only if {\cf (zero? (imag-part \var{z}))} and {\cf (exact?
  (imag-part \var{z}))} are both true.

If \var{x} is a real number object, then {\cf (rational? \var{x})} is true if
and only if there exist exact integer objects \vari{k} and \varii{k} such that
{\cf (= \var{x} (/ \vari{k} \varii{k}))} and {\cf (= (numerator
  \var{x}) \vari{k})} and {\cf (= (denominator \var{x}) \varii{k})} are
all true.  Thus infinities and NaNs are not rational number objects.

If \var{q} is a rational number objects, then {\cf (integer?
\var{q})} is true if and only if {\cf (= (denominator
\var{q}) 1)} is true.  If \var{q} is not a rational number object,
then {\cf (integer? \var{q})} is \schfalse.

\begin{scheme}
(complex? 3+4i)                        \ev  \schtrue{}
(complex? 3)                           \ev  \schtrue{}
(real? 3)                              \ev  \schtrue{}
(real? -2.5+0.0i)                      \ev  \schfalse{}
(real? -2.5+0i)                        \ev  \schtrue{}
(real? -2.5)                           \ev  \schtrue{}
(real? \sharpsign{}e1e10)                         \ev  \schtrue{}
(rational? 6/10)                       \ev  \schtrue{}
(rational? 6/3)                        \ev  \schtrue{}
(rational? 2)                          \ev  \schtrue{}
(integer? 3+0i)                        \ev  \schtrue{}
(integer? 3.0)                         \ev  \schtrue{}
(integer? 8/4)                         \ev  \schtrue{}

(number? +nan.0)                       \ev  \schtrue{}
(complex? +nan.0)                      \ev  \schtrue{}
(real? +nan.0)                         \ev  \schtrue{}
(rational? +nan.0)                     \ev  \schfalse{}
(complex? +inf.0)                      \ev  \schtrue{}
(real? -inf.0)                         \ev  \schtrue{}
(rational? -inf.0)                     \ev  \schfalse{}
(integer? -inf.0)                      \ev  \schfalse{}%
\end{scheme}

\begin{note}
Except for {\cf number?}, the behavior of these type predicates
on inexact number objects is
unreliable, because any inaccuracy may
affect the result.
\end{note}
\end{entry}

\begin{entry}{%
\proto{real-valued?}{ obj}{procedure}
\proto{rational-valued?}{ obj}{procedure}
\proto{integer-valued?}{ obj}{procedure}}

These numerical type predicates can be applied to any kind of
argument.  The {\cf real-valued?} procedure
returns \schtrue{} if the object is a number object and is equal in the
sense of {\cf =} to some real number object, or if the object is a NaN, or a
complex number object whose real part is a NaN and whose imaginary
part is zero
in the sense of {\cf zero?}.  The {\cf rational-valued?} and {\cf
  integer-valued?} procedures return \schtrue{} if the object is a
number object and is equal in the sense of {\cf =} to some object of the
named type, and otherwise they return \schfalse{}.

\begin{scheme}
(real-valued? +nan.0)                  \ev  \schtrue{}
(real-valued? +nan.0+0i)                  \ev  \schtrue{}
(real-valued? -inf.0)                  \ev  \schtrue{}
(real-valued? 3)                       \ev  \schtrue{}
(real-valued? -2.5+0.0i)               \ev  \schtrue{}
(real-valued? -2.5+0i)                 \ev  \schtrue{}
(real-valued? -2.5)                    \ev  \schtrue{}
(real-valued? \sharpsign{}e1e10)                  \ev  \schtrue{}

(rational-valued? +nan.0)              \ev  \schfalse{}
(rational-valued? -inf.0)              \ev  \schfalse{}
(rational-valued? 6/10)                \ev  \schtrue{}
(rational-valued? 6/10+0.0i)           \ev  \schtrue{}
(rational-valued? 6/10+0i)             \ev  \schtrue{}
(rational-valued? 6/3)                 \ev  \schtrue{}

(integer-valued? 3+0i)                 \ev  \schtrue{}
(integer-valued? 3+0.0i)               \ev  \schtrue{}
(integer-valued? 3.0)                  \ev  \schtrue{}
(integer-valued? 3.0+0.0i)             \ev  \schtrue{}
(integer-valued? 8/4)                  \ev  \schtrue{}%
\end{scheme}

\begin{note}
  These procedures test whether a given number object can be coerced
  to the specified type without loss of numerical accuracy.
  Specifically, the behavior of these predicates differs from the
  behavior of {\cf real?}, {\cf rational?}, and {\cf integer?} on
  complex number objects whose imaginary part is inexact zero.
\end{note}

\begin{note}
The behavior of these type predicates on inexact number objects is
unreliable, because any inaccuracy may
affect the result.
\end{note}
\end{entry}

\begin{entry}{%
\proto{exact?}{ z}{procedure}
\proto{inexact?}{ z}{procedure}}

These numerical predicates provide tests for the exactness of a
quantity.  For any number object, precisely one of these predicates is
true.

\begin{scheme}
(exact? 5)                   \ev  \schtrue{}
(inexact? +inf.0)            \ev  \schtrue{}%
\end{scheme}
\end{entry}

\subsubsection{Generic conversions}

\begin{entry}{%
\proto{inexact}{ z}{procedure}
\proto{exact}{ z}{procedure}}

The {\cf inexact} procedure returns an inexact representation of \var{z}.  If
inexact number objects of the appropriate type have bounded precision, then
the value returned is an inexact number object that is nearest to the
argument.  If an exact argument has no reasonably close inexact
equivalent, an exception with condition type
{\cf\&implementation-violation} may be
raised.

\begin{note}
  For a real number object whose magnitude is finite but so large that it has
  no reasonable finite approximation as an inexact number, a
  reasonably close inexact equivalent may be {\cf +inf.0} or {\cf
    -inf.0}.  Similarly, the inexact representation of a complex
  number object whose components are finite may have infinite components.
\end{note}

The {\cf exact} procedure returns an exact representation of \var{z}.  The value
returned is the exact number object that is numerically closest to the
argument; in most cases, the result of this procedure should be
numerically equal to its argument.  If an inexact argument has no
reasonably close exact equivalent, an exception with condition type
{\cf\&implementation-violation} may be
raised.

These procedures implement the natural one-to-one correspondence
between exact and inexact integer objects throughout an
implementation-dependent range.

The {\cf inexact} and {\cf exact} procedures are idempotent.
\end{entry}

\subsubsection{Arithmetic operations}

\begin{entry}{%
\proto{=}{ \vari{z} \varii{z} \variii{z} \dotsfoo}{procedure}
\proto{<}{ \vari{x} \varii{x} \variii{x} \dotsfoo}{procedure}
\proto{>}{ \vari{x} \varii{x} \variii{x} \dotsfoo}{procedure}
\proto{<=}{ \vari{x} \varii{x} \variii{x} \dotsfoo}{procedure}
\proto{>=}{ \vari{x} \varii{x} \variii{x} \dotsfoo}{procedure}}

These procedures return \schtrue{} if their arguments are
(respectively): equal, monotonically increasing, monotonically
decreasing, monotonically nondecreasing, or monotonically
nonincreasing, and \schfalse{} otherwise.

\begin{scheme}
(= +inf.0 +inf.0)           \ev  \schtrue{}
(= -inf.0 +inf.0)           \ev  \schfalse{}
(= -inf.0 -inf.0)           \ev  \schtrue{}%
\end{scheme}

For any real number object \var{x} that is neither infinite nor NaN:

\begin{scheme}
(< -inf.0 \var{x} +inf.0))        \ev  \schtrue{}
(> +inf.0 \var{x} -inf.0))        \ev  \schtrue{}%
\end{scheme}

For any number object \var{z}:
%
\begin{scheme}
(= +nan.0 \var{z})               \ev  \schfalse{}%
\end{scheme}
%
For any real number object \var{x}:
%
\begin{scheme}
(< +nan.0 \var{x})               \ev  \schfalse{}
(> +nan.0 \var{x})               \ev  \schfalse{}%
\end{scheme}

These predicates must be transitive.

\begin{note}
The traditional implementations of these predicates in Lisp-like
languages are not transitive.
\end{note}

\begin{note}
While it is possible to compare inexact number objects using these
predicates, the results may be unreliable because a small inaccuracy
may affect the result; this is especially true of {\cf =} and {\cf zero?} (below).

When in doubt, consult a numerical analyst.
\end{note}
\end{entry}

\begin{entry}{%
\proto{zero?}{ z}{procedure}
\proto{positive?}{ x}{procedure}
\proto{negative?}{ x}{procedure}
\proto{odd?}{ n}{procedure}
\proto{even?}{ n}{procedure}
\proto{finite?}{ x}{procedure}
\proto{infinite?}{ x}{procedure}
\proto{nan?}{ x}{procedure}}

These numerical predicates test a number object for a particular property,
returning \schtrue{} or \schfalse{}.  The {\cf zero?}
procedure
tests if the number object is {\cf =} to zero, {\cf positive?} tests whether it is
greater than zero, {\cf negative?} tests whether it is less than zero, {\cf
  odd?} tests whether it is odd, {\cf even?} tests whether it is even, {\cf
  finite?} tests whether it is not an infinity and not a NaN, {\cf
  infinite?} tests whether it is an infinity, {\cf nan?} tests whether it is a
NaN.

\begin{scheme}
(zero? +0.0)                  \ev  \schtrue{}
(zero? -0.0)                  \ev  \schtrue{}
(zero? +nan.0)                \ev  \schfalse{}
(positive? +inf.0)            \ev  \schtrue{}
(negative? -inf.0)            \ev  \schtrue{}
(positive? +nan.0)            \ev  \schfalse{}
(negative? +nan.0)            \ev  \schfalse{}
(finite? +inf.0)              \ev  \schfalse{}
(finite? 5)                   \ev  \schtrue{}
(finite? 5.0)                 \ev  \schtrue{}
(infinite? 5.0)               \ev  \schfalse{}
(infinite? +inf.0)            \ev  \schtrue{}%
\end{scheme}

\begin{note}
  As with the predicates above, the results may be unreliable because
  a small inaccuracy may affect the result.
\end{note}
\end{entry}

\begin{entry}{%
\proto{max}{ \vari{x} \varii{x} \dotsfoo}{procedure}
\proto{min}{ \vari{x} \varii{x} \dotsfoo}{procedure}}

These procedures return the maximum or minimum of their arguments.

\begin{scheme}
(max 3 4)                              \ev  4
(max 3.9 4)                            \ev  4.0%
\end{scheme}

For any real number object \var{x}:

\begin{scheme}
(max +inf.0 \var{x})                         \ev  +inf.0
(min -inf.0 \var{x})                         \ev  -inf.0%
\end{scheme}

\begin{note}
If any argument is inexact, then the result is also inexact (unless
the procedure can prove that the inaccuracy is not large enough to affect the
result, which is possible only in unusual implementations).  If {\cf min} or
{\cf max} is used to compare number objects of mixed exactness, and the numerical
value of the result cannot be represented as an inexact number object without loss of
accuracy, then the procedure may raise an exception with condition
type {\cf\&implementation-restriction}.
\end{note}

\end{entry}

\begin{entry}{%
\proto{+}{ \vari{z} \dotsfoo}{procedure}
\proto{*}{ \vari{z} \dotsfoo}{procedure}}

These procedures return the sum or product of their arguments.

\begin{scheme}
(+ 3 4)                                \ev  7
(+ 3)                                  \ev  3
(+)                                    \ev  0
(+ +inf.0 +inf.0)                      \ev  +inf.0
(+ +inf.0 -inf.0)                      \ev  +nan.0

(* 4)                                  \ev  4
(*)                                    \ev  1
(* 5 +inf.0)                           \ev  +inf.0
(* -5 +inf.0)                          \ev  -inf.0
(* +inf.0 +inf.0)                      \ev  +inf.0
(* +inf.0 -inf.0)                      \ev  -inf.0
(* 0 +inf.0)                           \ev  0 \textnormal{\textit{or}} +nan.0
(* 0 +nan.0)                           \ev  0 \textnormal{\textit{or}} +nan.0
(* 1.0 0)                              \ev  0 \textnormal{\textit{or}} 0.0%
\end{scheme}

For any real number object \var{x} that is neither infinite nor NaN:

\begin{scheme}
(+ +inf.0 \var{x})                           \ev  +inf.0
(+ -inf.0 \var{x})                           \ev  -inf.0%
\end{scheme}

For any real number object \var{x}:

\begin{scheme}
(+ +nan.0 \var{x})                           \ev  +nan.0%
\end{scheme}

For any real number object \var{x} that is not an exact 0:

\begin{scheme}
(* +nan.0 \var{x})                           \ev  +nan.0%
\end{scheme}

If any of these procedures are applied to mixed non-rational real and
non-real complex arguments, they either raise an exception with
condition type {\cf\&implementation-restriction} or return an
unspecified number object.

Implementations that distinguish $-0.0$ should adopt behavior
consistent with the following examples:

\begin{scheme}
(+ 0.0 -0.0)  \ev 0.0
(+ -0.0 0.0)  \ev 0.0
(+ 0.0 0.0)   \ev 0.0
(+ -0.0 -0.0) \ev -0.0%
\end{scheme}
\end{entry}

\begin{entry}{%
\proto{-}{ z}{procedure}
\rproto{-}{ \vari{z} \varii{z} \dotsfoo}{procedure}}

With two or more arguments, this procedures returns the difference of
its arguments, associating to the left.  With one argument, however,
it returns the additive inverse of its argument.

\begin{scheme}
(- 3 4)                                \ev  -1
(- 3 4 5)                              \ev  -6
(- 3)                                  \ev  -3
(- +inf.0 +inf.0)                      \ev  +nan.0%
\end{scheme}

If this procedure is applied to mixed non-rational real and
non-real complex arguments, it either raises an exception with
condition type {\cf\&implementation-restriction} or returns an
unspecified number object.

Implementations that distinguish $-0.0$ should adopt behavior
consistent with the following examples:

\begin{scheme}
(- 0.0)       \ev -0.0
(- -0.0)      \ev 0.0
(- 0.0 -0.0)  \ev 0.0
(- -0.0 0.0)  \ev -0.0
(- 0.0 0.0)   \ev 0.0
(- -0.0 -0.0) \ev 0.0%
\end{scheme}
\end{entry}

\begin{entry}{%
\proto{/}{ z}{procedure}
\rproto{/}{ \vari{z} \varii{z} \dotsfoo}{procedure}}

\domain{If all of the arguments are exact, then the divisors must all
  be nonzero.}
With two or more arguments, this procedure returns the 
quotient of its arguments, associating to the left.  With one
argument, however, it returns the multiplicative inverse
of its argument.

\begin{scheme}
(/ 3 4 5)                              \ev  3/20
(/ 3)                                  \ev  1/3
(/ 0.0)                                \ev  +inf.0
(/ 1.0 0)                              \ev  +inf.0
(/ -1 0.0)                             \ev  -inf.0
(/ +inf.0)                             \ev  0.0
(/ 0 0)                                \xev \exception{\&assertion}
(/ 3 0)                                \xev \exception{\&assertion}
(/ 0 3.5)                              \ev  0.0
(/ 0 0.0)                              \ev  +nan.0
(/ 0.0 0)                              \ev  +nan.0
(/ 0.0 0.0)                            \ev  +nan.0%
\end{scheme}

If this procedure is applied to mixed non-rational real and
non-real complex arguments, it either raises an exception with
condition type {\cf\&implementation-restriction} or returns an
unspecified number object.
\end{entry}

\begin{entry}{%
\proto{abs}{ x}{procedure}}

Returns the absolute value of its argument.

\begin{scheme}
(abs -7)                               \ev  7
(abs -inf.0)                           \ev  +inf.0%
\end{scheme}

\end{entry}

\begin{entry}{%
\proto{div-and-mod}{ \vari{x} \varii{x}}{procedure}
\proto{div}{ \vari{x} \varii{x}}{procedure}
\proto{mod}{ \vari{x} \varii{x}}{procedure}
\proto{div0-and-mod0}{ \vari{x} \varii{x}}{procedure}
\proto{div0}{ \vari{x} \varii{x}}{procedure}
\proto{mod0}{ \vari{x} \varii{x}}{procedure}}

These procedures implement number-theoretic integer division and
return the results of the corresponding mathematical operations
specified in section~\ref{integerdivision}.  In each case, \vari{x}
must be neither infinite nor a NaN, and \varii{x} must be nonzero;
otherwise, an exception with condition type {\cf\&assertion} is raised.

\begin{scheme}
(div \vari{x} \varii{x})         \ev \(\vari{x}~\mathrm{div}~\varii{x}\)
(mod \vari{x} \varii{x})         \ev \(\vari{x}~\mathrm{mod}~\varii{x}\)
(div-and-mod \vari{x} \varii{x})     \ev \(\vari{x}~\mathrm{div}~\varii{x}, \vari{x}~\mathrm{mod}~\varii{x}\)\\\>\>\>; \textrm{two return values}
(div0 \vari{x} \varii{x})        \ev \(\vari{x}~\mathrm{div}_0~\varii{x}\)
(mod0 \vari{x} \varii{x})        \ev \(\vari{x}~\mathrm{mod}_0~\varii{x}\)
(div0-and-mod0 \vari{x} \varii{x})   \lev \(\vari{x}~\mathrm{div}_0~\varii{x}, \vari{x}~\mathrm{mod}_0~\varii{x}\)\\\>\>; \textrm{two return values}%
\end{scheme}

\begin{entry}{%
\proto{gcd}{ \vari{n} \dotsfoo}{procedure}
\proto{lcm}{ \vari{n} \dotsfoo}{procedure}}

These procedures return the greatest common divisor or least common
multiple of their arguments.  The result is always non-negative.

\begin{scheme}
(gcd 32 -36)                           \ev  4
(gcd)                                  \ev  0
(lcm 32 -36)                           \ev  288
(lcm 32.0 -36)                         \ev  288.0
(lcm)                                  \ev  1%
\end{scheme}
\end{entry}

\begin{entry}{%
\proto{numerator}{ q}{procedure}
\proto{denominator}{ q}{procedure}}

These procedures return the numerator or denominator of their
argument; the result is computed as if the argument was represented as
a fraction in lowest terms.  The denominator is always positive.  The
denominator of $0$ is defined to be $1$.

\begin{scheme}
(numerator (/ 6 4))                    \ev  3
(denominator (/ 6 4))                  \ev  2
(denominator
  (inexact (/ 6 4)))                   \ev  2.0%
\end{scheme}
\end{entry}

\begin{entry}{%
\proto{floor}{ x}{procedure}
\proto{ceiling}{ x}{procedure}
\proto{truncate}{ x}{procedure}
\proto{round}{ x}{procedure}}

These procedures return inexact integer objects for inexact arguments that are
not infinities or NaNs, and exact integer objects for exact rational
arguments.  For such arguments, {\cf floor} returns the largest
integer object not larger than \var{x}.  The {\cf ceiling} procedure returns the smallest
integer object not smaller than \var{x}.  The {\cf truncate} procedure returns the integer
object closest to \var{x} whose absolute value is not larger than the
absolute value of \var{x}.  The {\cf round} procedure returns the
closest integer object to
\var{x}, rounding to even when \var{x} represents a number halfway between two
integers.

\begin{note}
If the argument to one of these procedures is inexact, then the result
is also inexact.  If an exact value is needed, the
result should be passed to the {\cf exact} procedure.
\end{note}

Although infinities and NaNs are not integer objects, these procedures return
an infinity when given an infinity as an argument, and a NaN when
given a NaN.

\begin{scheme}
(floor -4.3)                           \ev  -5.0
(ceiling -4.3)                         \ev  -4.0
(truncate -4.3)                        \ev  -4.0
(round -4.3)                           \ev  -4.0

(floor 3.5)                            \ev  3.0
(ceiling 3.5)                          \ev  4.0
(truncate 3.5)                         \ev  3.0
(round 3.5)                            \ev  4.0

(round 7/2)                            \ev  4
(round 7)                              \ev  7

(floor +inf.0)                         \ev  +inf.0
(ceiling -inf.0)                       \ev  -inf.0
(round +nan.0)                         \ev  +nan.0%
\end{scheme}

\end{entry}

\begin{entry}{%
\proto{rationalize}{ \vari{x} \varii{x}}{procedure}}

The {\cf rationalize} procedure returns the a number object
representing the {\em simplest} rational
number differing from \vari{x} by no more than \varii{x}.    A rational number $r_1$ is
{\em simpler} \mainindex{simplest rational} than another rational number
$r_2$ if $r_1 = p_1/q_1$ and $r_2 = p_2/q_2$ (in lowest terms) and $|p_1|
\leq |p_2|$ and $|q_1| \leq |q_2|$.  Thus $3/5$ is simpler than $4/7$.
Although not all rationals are comparable in this ordering (consider $2/7$
and $3/5$) any interval contains a rational number that is simpler than
every other rational number in that interval (the simpler $2/5$ lies
between $2/7$ and $3/5$).  Note that $0 = 0/1$ is the simplest rational of
all.
%
\begin{scheme}
(rationalize (exact .3) 1/10)          \lev 1/3
(rationalize .3 1/10)                  \lev \sharpsign{}i1/3  ; \textrm{approximately}

(rationalize +inf.0 3)                 \ev  +inf.0
(rationalize +inf.0 +inf.0)            \ev  +nan.0
(rationalize 3 +inf.0)                 \ev  0.0%
\end{scheme}
%
The first two examples hold only in implementations whose inexact real
number objects have sufficient precision.

\end{entry}

\begin{entry}{%
\proto{exp}{ z}{procedure}
\proto{log}{ z}{procedure}
\rproto{log}{ \vari{z} \varii{z}}{procedure}
\proto{sin}{ z}{procedure}
\proto{cos}{ z}{procedure}
\proto{tan}{ z}{procedure}
\proto{asin}{ z}{procedure}
\proto{acos}{ z}{procedure}
\proto{atan}{ z}{procedure}
\rproto{atan}{ \vari{x} \varii{x}}{procedure}}

These procedures compute the usual transcendental functions.  The {\cf
  exp} procedure computes the base-$e$ exponential of \var{z}. 
The {\cf log} procedure with a single argument computes the natural logarithm of
\var{z} (not the base-ten logarithm); {\cf (log \vari{z}
  \varii{z})} computes the base-\varii{z} logarithm of \vari{z}.
The {\cf asin}, {\cf acos}, and {\cf atan} procedures compute arcsine,
arccosine, and arctangent, respectively.  The two-argument variant of
{\cf atan} computes {\cf (angle (make-rectangular \varii{x}
\vari{x}))}.

See section~\ref{transcendentalfunctions} for the underlying
mathematical operations. These procedures may return inexact results
even when given exact arguments.

\begin{scheme}
(exp +inf.0)                   \ev +inf.0
(exp -inf.0)                   \ev 0.0
(log +inf.0)                   \ev +inf.0
(log 0.0)                      \ev -inf.0
(log 0)                        \xev \exception{\&assertion}
(log -inf.0)                   \lev +inf.0+3.141592653589793i\\\> ; \textrm{approximately}
(atan -inf.0)                  \lev -1.5707963267948965 ; \textrm{approximately}
(atan +inf.0)                  \lev 1.5707963267948965 ; \textrm{approximately}
(log -1.0+0.0i)                \lev 0.0+3.141592653589793i ; \textrm{approximately}
(log -1.0-0.0i)                \lev 0.0-3.141592653589793i ; \textrm{approximately}\\\>; \textrm{if -0.0 is distinguished}%
\end{scheme}
\end{entry}

\begin{entry}{%
\proto{sqrt}{ z}{procedure}}

Returns the principal square root of \var{z}.  For rational \var{z},
the result has either positive real part, or zero real part and
non-negative imaginary part.  With $\log$ defined as in
section~\ref{transcendentalfunctions}, the value of {\cf (sqrt
  \var{z})} could be expressed as $e^{\frac{\log z}{2}}$.

The {\cf sqrt} procedure may return an inexact result even when given an exact
argument.

\begin{scheme}
(sqrt -5)                   \lev  0.0+2.23606797749979i ; \textrm{approximately}
(sqrt +inf.0)               \ev  +inf.0
(sqrt -inf.0)               \ev  +inf.0i%
\end{scheme}
\end{entry}

\begin{entry}{%
\proto{exact-integer-sqrt}{ k}{procedure}}

The {\cf exact-integer-sqrt} procedure returns two non-negative exact
integer objects $s$ and $r$ where $\var{k} = s^2 +
r$ and $\var{k} < (s+1)^2$.

\begin{scheme}
(exact-integer-sqrt 4) \ev 2 0\\\>\>\>; \textrm{two return values}
(exact-integer-sqrt 5) \ev 2 1\\\>\>\>; \textrm{two return values}
\end{scheme}
\end{entry}

\begin{entry}{%
\proto{expt}{ \vari{z} \varii{z}}{procedure}}

Returns \vari{z} raised to the power \varii{z}.  For nonzero \vari{z},
this is $e^{z_2 \log z_1}$.
$0.0^{z}$ is $1.0$ if $\var{z} = 0.0$, and $0.0$ if {\cf
  (real-part \var{z})} is positive.  For other cases in which
the first argument is zero, either an exception is raised with
condition type {\cf\&implementation-restriction}, or an unspecified
number object is returned.

For an exact real number object \vari{z} and an exact
integer object \varii{z}, {\cf (expt \vari{z}
\varii{z})} must return an exact result.  For all other
values of \vari{z} and \varii{z}, {\cf (expt \vari{z}
\varii{z})} may return an inexact result, even when both
\vari{z} and \varii{z} are exact.

\begin{scheme}
(expt 5 3)                  \ev  125
(expt 5 -3)                 \ev  1/125
(expt 5 0)                  \ev  1
(expt 0 5)                  \ev  0
(expt 0 5+.0000312i)        \ev  0
(expt 0 -5)                 \ev  \unspecified
(expt 0 -5+.0000312i)       \ev  \unspecified
(expt 0 0)                  \ev  1
(expt 0.0 0.0)              \ev  1.0%
\end{scheme}
\end{entry}

\begin{entry}{%
\proto{make-rectangular}{ \vari{x} \varii{x}}{procedure}
\proto{make-polar}{ \variii{x} \variv{x}}{procedure}
\proto{real-part}{ z}{procedure}
\proto{imag-part}{ z}{procedure}
\proto{magnitude}{ z}{procedure}
\proto{angle}{ z}{procedure}}

Suppose $a_1$, $a_2$, $a_3$, and $a_4$ are real
numbers, and $c$ is a complex number such that the
following holds:
%
\begin{displaymath}
c = a_1 + a_2 i = a_3 e^{i a_4}
\end{displaymath}

Then, if \vari{x}, \varii{x}, \variii{x}, and \variv{x} are number
objects representing $a_1$, $a_2$, $a_3$, and $a_4$, respectively,
{\cf (make-rectangular \vari{x} \varii{x})} returns $c$, and {\cf
  (make-polar \variii{x} \variv{x})} returns $c$.
%
\begin{scheme}
(make-rectangular 1.1 2.2) \lev 1.1+2.2i ; \textrm{approximately}
(make-polar 1.1 2.2) \lev 1.1@2.2 ; \textrm{approximately}
\end{scheme}
%
Conversely, if $-\pi \leq a_4 \leq \pi$, and if $z$ is a number object
representing $c$, then {\cf (real-part \var{z})} returns $a_1$ {\cf
  (imag-part \var{z})} returns $a_2$, {\cf (magnitude \var{z})}
returns $a_3$, and {\cf (angle \var{z})} returns $a_4$.

\begin{scheme}
(real-part 1.1+2.2i)              \ev 1.1 ; \textrm{approximately}
(imag-part 1.1+2.2i)              \ev 2.2i ; \textrm{approximately}
(magnitude 1.1@2.2)              \ev 1.1 ; \textrm{approximately}
(angle 1.1@2.2)                  \ev 2.2 ; \textrm{approximately}

(angle -1.0)         \lev 3.141592653589793 ; \textrm{approximately}
(angle -1.0+0.0i)    \lev 3.141592653589793 ; \textrm{approximately}
(angle -1.0-0.0i)    \lev -3.141592653589793 ; \textrm{approximately}\\\>; \textrm{if -0.0 is distinguished}
(angle +inf.0)       \ev 0.0
(angle -inf.0)       \lev 3.141592653589793 ; \textrm{approximately}%
\end{scheme}

Moreover, suppose \vari{x}, \varii{x} are such that either \vari{x}
or \varii{x} is an infinity, then
%
\begin{scheme}
(make-rectangular \vari{x} \varii{x}) \ev \var{z}
(magnitude \var{z})              \ev +inf.0%
\end{scheme}
\end{entry}

The {\cf make-polar}, {\cf magnitude}, and
{\cf angle} procedures may return inexact results even when given exact
arguments.

\begin{scheme}
(angle -1)                    \lev 3.141592653589793 ; \textrm{approximately}
\end{scheme}
\end{entry}

\subsubsection{Numerical Input and Output}

\begin{entry}{%
\proto{number->string}{ z}{procedure}
\rproto{number->string}{ z radix}{procedure}
\rproto{number->string}{ z radix precision}{procedure}}

\var{Radix} must be an exact integer object, either 2, 8, 10, or 16.  If
omitted, \var{radix} defaults to 10.  If a \var{precision} is
specified, then \var{z} must be an inexact complex number object,
\var{precision} must be an exact positive integer object, and \var{radix}
must be 10.  The {\cf number->string} procedure takes a number object and a
radix and returns as a string an external representation of the given
number object in the given radix such that
%
\begin{scheme}
(let ((number \var{z}) (radix \var{radix}))
  (eqv? (string->number
          (number->string number radix)
          radix)
        number))%
\end{scheme}
%
is true.  If no possible result makes this expression
true, an exception with condition type
{\cf\&implementation-\hp{}restriction} is raised.

\begin{note}
The error case can occur only when \var{z} is not a complex number object
or is a complex number object with a non-rational real or imaginary part.
\end{note}

If a \var{precision} is specified, then the representations of the
inexact real components of the result, unless they are infinite or
NaN, specify an explicit \meta{mantissa width} \var{p}, and \var{p} is the
least $\var{p} \geq \var{precision}$ for which the above expression is
true.

If \var{z} is inexact, the radix is 10, and the above expression and
condition can be satisfied by a result that contains a decimal point,
then the result contains a decimal point and is expressed using the
minimum number of digits (exclusive of exponent, trailing zeroes, and
mantissa width) needed to make the above expression and condition
true~\cite{howtoprint,howtoread}; otherwise the format of the result
is unspecified.

The result returned by {\cf number->string} never contains an explicit
radix prefix.
\end{entry}

\begin{entry}{%
\proto{string->number}{ string}{procedure}
\rproto{string->number}{ string radix}{procedure}}

Returns a number object with maximally precise representation expressed by the
given \var{string}.  \var{Radix} must be an exact integer object, either 2, 8, 10,
or 16.  If supplied, \var{radix} is a default radix that may be overridden
by an explicit radix prefix in \var{string} (e.g., {\tt "\#o177"}).  If \var{radix}
is not supplied, then the default radix is 10.  If \var{string} is not
a syntactically valid notation for a number object or a notation for a
rational number object with a zero denominator, then {\cf string->number}
returns \schfalse{}.
%
\begin{scheme}
(string->number "100")                 \ev  100
(string->number "100" 16)              \ev  256
(string->number "1e2")                 \ev  100.0
(string->number "0/0")                 \ev  \schfalse
(string->number "+inf.0")              \ev  +inf.0
(string->number "-inf.0")              \ev  -inf.0
(string->number "+nan.0")              \ev  +nan.0%
\end{scheme}

\begin{note}
  The {\cf string->number} procedure always returns a number object or
  \schfalse{}; it never raises an exception.
\end{note}
\end{entry}


\section{Booleans}
\label{booleansection}

The standard boolean objects for true and false have external representations
\schtrue{} and \schfalse.\sharpindex{t}\sharpindex{f} However, of all
objects, only \schfalse{} counts as false in
conditional expressions.  See section~\ref{booleanvaluessection}.

\begin{note}
Programmers accustomed to other dialects of Lisp should be aware that
Scheme distinguishes both \schfalse{} and the empty list \index{empty list}
from each other and from the symbol \ide{nil}.
\end{note}

\begin{entry}{%
\proto{not}{ obj}{procedure}}

Returns \schtrue{} if \var{obj} is \schfalse, and returns
\schfalse{} otherwise.

\begin{scheme}
(not \schtrue)   \ev  \schfalse
(not 3)          \ev  \schfalse
(not (list 3))   \ev  \schfalse
(not \schfalse)  \ev  \schtrue
(not '())        \ev  \schfalse
(not (list))     \ev  \schfalse
(not 'nil)       \ev  \schfalse%
\end{scheme}

\end{entry}


\begin{entry}{%
\proto{boolean?}{ obj}{procedure}}

Returns \schtrue{} if \var{obj} is either \schtrue{} or
\schfalse{} and returns \schfalse{} otherwise.

\begin{scheme}
(boolean? \schfalse)  \ev  \schtrue
(boolean? 0)          \ev  \schfalse
(boolean? '())        \ev  \schfalse%
\end{scheme}

\begin{entry}{%
\proto{boolean=?}{ \vari{bool} \varii{bool} \variii{bool}
  \dotsfoo}{procedure}}

Returns \schtrue{} if the booleans are the same.
\end{entry}

\end{entry}

 
\section{Pairs and lists}
\label{listsection}

A \defining{pair} is a
compound structure with two fields called the car and cdr fields (for
historical reasons).  Pairs are created by the procedure {\cf cons}.
The car and cdr fields are accessed by the procedures {\cf car} and
{\cf cdr}.

Pairs are used primarily to represent lists.  A list can
be defined recursively as either the empty list\index{empty list} or a pair whose
cdr is a list.  More precisely, the set of lists is defined as the smallest
set \var{X} such that

\begin{itemize}
\item The empty list is in \var{X}.
\item If \var{list} is in \var{X}, then any pair whose cdr field contains
      \var{list} is also in \var{X}.
\end{itemize}

The objects in the car fields of successive pairs of a list are the
elements of the list.  For example, a two-element list is a pair whose car
is the first element and whose cdr is a pair whose car is the second element
and whose cdr is the empty list.  The length of a list is the number of
elements, which is the same as the number of pairs.

The empty list\mainindex{empty list} is a special object of its own type.
It is not a pair.  It has no elements and its length is zero.

\begin{note}
The above definitions imply that all lists have finite length and are
terminated by the empty list.
\end{note}

A chain of pairs not ending in the empty list is called an
\defining{improper list}.  Note that an improper list is not a list.
The list and dotted notations can be combined to represent
improper lists:

\begin{scheme}
(a b c . d)%
\end{scheme}

is equivalent to

\begin{scheme}
(a . (b . (c . d)))%
\end{scheme}

Whether a given pair is a list depends upon what is stored in the cdr
field.

\begin{entry}{%
\proto{pair?}{ obj}{procedure}}

Returns \schtrue{} if \var{obj} is a pair, and otherwise
returns \schfalse.

\begin{scheme}
(pair? '(a . b))        \ev  \schtrue
(pair? '(a b c))        \ev  \schtrue
(pair? '())             \ev  \schfalse
(pair? '\#(a b))         \ev  \schfalse%
\end{scheme}
\end{entry}


\begin{entry}{%
\proto{cons}{ \vari{obj} \varii{obj}}{procedure}}

Returns a newly allocated pair whose car is \vari{obj} and whose cdr is
\varii{obj}.  The pair is guaranteed to be different (in the sense of
{\cf eqv?}) from every existing object.

\begin{scheme}
(cons 'a '())           \ev  (a)
(cons '(a) '(b c d))    \ev  ((a) b c d)
(cons "a" '(b c))       \ev  ("a" b c)
(cons 'a 3)             \ev  (a . 3)
(cons '(a b) 'c)        \ev  ((a b) . c)%
\end{scheme}
\end{entry}


\begin{entry}{%
\proto{car}{ pair}{procedure}}

Returns the contents of the car field of \var{pair}.

\begin{scheme}
(car '(a b c))          \ev  a
(car '((a) b c d))      \ev  (a)
(car '(1 . 2))          \ev  1
(car '())               \xev \exception{\&assertion}%
\end{scheme}
 
\end{entry}


\begin{entry}{%
\proto{cdr}{ pair}{procedure}}

Returns the contents of the cdr field of \var{pair}.

\begin{scheme}
(cdr '((a) b c d))      \ev  (b c d)
(cdr '(1 . 2))          \ev  2
(cdr '())               \xev \exception{\&assertion}%
\end{scheme}
 
\end{entry}



\setbox0\hbox{\tt(cadr \var{pair})}
\setbox1\hbox{procedure}


\begin{entry}{%
\proto{caar}{ pair}{procedure}
\proto{cadr}{ pair}{procedure}
\texonly
\pproto{\hbox to 1\wd0 {\hfil$\vdots$\hfil}}{\hbox to 1\wd1 {\hfil$\vdots$\hfil}}
\endtexonly
\htmlonly $\vdots$ \endhtmlonly
\proto{cdddar}{ pair}{procedure}
\proto{cddddr}{ pair}{procedure}}

These procedures are compositions of {\cf car} and {\cf cdr}, where
for example {\cf caddr} could be defined by

\begin{scheme}
(define caddr (lambda (x) (car (cdr (cdr x))))){\rm.}%
\end{scheme}

Arbitrary compositions, up to four deep, are provided.  There are
twenty-eight of these procedures in all.

\end{entry}


\begin{entry}{%
\proto{null?}{ obj}{procedure}}

Returns \schtrue{} if \var{obj} is the empty list\index{empty list},
\schfalse otherwise.

\end{entry}

\begin{entry}{%
\proto{list?}{ obj}{procedure}}

Returns \schtrue{} if \var{obj} is a list, \schfalse{} otherwise.
By definition, all lists are chains of pairs that have finite length and are terminated by
the empty list.

\begin{scheme}
(list? '(a b c))     \ev  \schtrue
(list? '())          \ev  \schtrue
(list? '(a . b))     \ev  \schfalse%
\end{scheme}
\end{entry}


\begin{entry}{%
\proto{list}{ \var{obj} \dotsfoo}{procedure}}

Returns a newly allocated list of its arguments.

\begin{scheme}
(list 'a (+ 3 4) 'c)            \ev  (a 7 c)
(list)                          \ev  ()%
\end{scheme}
\end{entry}


\begin{entry}{%
\proto{length}{ list}{procedure}}

Returns the length of \var{list}.

\begin{scheme}
(length '(a b c))               \ev  3
(length '(a (b) (c d e)))       \ev  3
(length '())                    \ev  0%
\end{scheme}
\end{entry}


\begin{entry}{%
\proto{append}{ list \dotsfoo{} obj}{procedure}}

Returns a possibly improper list consisting of the elements of the first \var{list}
followed by the elements of the other \var{list}s, with \var{obj} as
the cdr of the final pair.
An improper list results if \var{obj} is not a
list.

\begin{scheme}
(append '(x) '(y))              \ev  (x y)
(append '(a) '(b c d))          \ev  (a b c d)
(append '(a (b)) '((c)))        \ev  (a (b) (c))
(append '(a b) '(c . d))        \ev  (a b c . d)
(append '() 'a)                 \ev  a%
\end{scheme}

If {\cf append} constructs a nonempty chain of pairs, it is always
newly allocated.  If no pairs are allocated, \var{obj} is returned.
\end{entry}


\begin{entry}{%
\proto{reverse}{ list}{procedure}}

Returns a newly allocated list consisting of the elements of \var{list}
in reverse order.

\begin{scheme}
(reverse '(a b c))              \ev  (c b a)
(reverse '(a (b c) d (e (f))))  \lev  ((e (f)) d (b c) a)%
\end{scheme}
\end{entry}


\begin{entry}{%
\proto{list-tail}{ list k}{procedure}}

\domain{\var{List} should be a list of size at least \var{k}.}
The {\cf list-tail} procedure returns the subchain of pairs of \var{list}
obtained by omitting the first \var{k} elements.

\begin{scheme}
(list-tail '(a b c d) 2)                 \ev  (c d)%
\end{scheme}

\implresp The implementation must check that \var{list} is a chain of
pairs whose length is at least \var{k}.  It should not check that it is a chain
of pairs beyond this length.
\end{entry}


\begin{entry}{%
\proto{list-ref}{ list k}{procedure}}

\domain{\var{List} must be a list whose length is at least $\var{k}+1$.}
The {\cf list-tail} procedure returns the \var{k}th element of \var{list}.

\begin{scheme}
(list-ref '(a b c d) 2)                 \ev c%
\end{scheme}

\implresp The implementation must check that \var{list} is a chain of
pairs whose length is at least $\var{k}+1$.  It should not check that it is a list
of pairs beyond this length.
\end{entry}


\begin{entry}{%
\proto{map}{ proc \vari{list} \varii{list} \dotsfoo}{procedure}}

\domain{The \var{list}s should all have the same length.  \var{Proc}
  should accept as many arguments as there are
  \var{list}s and return a single value.  \var{Proc} should not mutate
  any of the \var{list}s.}

The {\cf map} procedure applies \var{proc} element-wise to the elements of the
\var{list}s and returns a list of the results, in order.
\var{Proc} is always called in the same dynamic environment 
as {\cf map} itself.
The order in which \var{proc} is applied to the elements of the
\var{list}s is unspecified.
If multiple returns occur from {\cf map}, the 
values returned by earlier returns are not mutated.

\begin{scheme}
(map cadr '((a b) (d e) (g h)))   \lev  (b e h)

(map (lambda (n) (expt n n))
     '(1 2 3 4 5))                \lev  (1 4 27 256 3125)

(map + '(1 2 3) '(4 5 6))         \ev  (5 7 9)

(let ((count 0))
  (map (lambda (ignored)
         (set! count (+ count 1))
         count)
       '(a b)))                 \ev  (1 2) \var{or} (2 1)%
\end{scheme}

\implresp The implementation should check that the \var{list}s all
have the same length.  The implementation must check the restrictions
on \var{proc} to the extent performed by applying it as described.  An
implementation may check whether \var{proc} is an appropriate argument
before applying it.
\end{entry}


\begin{entry}{%
\proto{for-each}{ proc \vari{list} \varii{list} \dotsfoo}{procedure}}

\domain{The \var{list}s should all have the same length.  \var{Proc}
  should accept as many arguments as there are
  \var{list}s.  \var{Proc} should not mutate
  any of the \var{list}s.}

The {\cf for-each} procedure applies \var{proc}
element-wise to the elements of the
\var{list}s for its side effects,  in order from the first elements to the
last.
\var{Proc} is always called in the same dynamic environment 
as {\cf for-each} itself.
The return values of {\cf for-each} \areunspecified.

\begin{scheme}
(let ((v (make-vector 5)))
  (for-each (lambda (i)
              (vector-set! v i (* i i)))
            '(0 1 2 3 4))
  v)                                \ev  \#(0 1 4 9 16)

(for-each (lambda (x) x) '(1 2 3 4)) \lev \theunspecified

(for-each even? '()) \ev \theunspecified%
\end{scheme}

\implresp The implementation should check that the \var{list}s all
have the same length.  The implementation must check the restrictions
on \var{proc} to the extent performed by applying it as described.
An implementation may check whether \var{proc} is an appropriate argument
before applying it.

\begin{note}
Implementations of {\cf for-each} may or may not tail-call
\var{proc} on the last elements.
\end{note}

\end{entry}


\section{Symbols}
\label{symbolsection}

Symbols are objects whose usefulness rests on the fact that two
symbols are identical (in the sense of {\cf eq?}, {\cf eqv?} and {\cf equal?}) if and only if their
names are spelled the same way. 
A symbol literal is formed using {\cf quote}.

\begin{entry}{%
\proto{symbol?}{ obj}{procedure}}

Returns \schtrue{} if \var{obj} is a symbol, otherwise returns \schfalse.

\begin{scheme}
(symbol? 'foo)          \ev  \schtrue
(symbol? (car '(a b)))  \ev  \schtrue
(symbol? "bar")         \ev  \schfalse
(symbol? 'nil)          \ev  \schtrue
(symbol? '())           \ev  \schfalse
(symbol? \schfalse)     \ev  \schfalse%
\end{scheme}
\end{entry}


\begin{entry}{%
\proto{symbol->string}{ symbol}{procedure}}

Returns the name of \var{symbol} as an immutable string.  

\begin{scheme}
(symbol->string 'flying-fish)     
                                  \ev  "flying-fish"
(symbol->string 'Martin)          \ev  "Martin"
(symbol->string
   (string->symbol "Malvina"))     
                                  \ev  "Malvina"%
\end{scheme}
\end{entry}

\begin{entry}{%
\proto{symbol=?}{ \vari{symbol} \varii{symbol} \variii{symbol}
  \dotsfoo}{procedure}}

Returns \schtrue{} if the symbols are the same, i.e., if their names
are spelled the same.
\end{entry}

\begin{entry}{%
\proto{string->symbol}{ string}{procedure}}

Returns the symbol whose name is \var{string}. 

\begin{scheme}
(eq? 'mISSISSIppi 'mississippi)  \lev  \schfalse
(string->symbol "mISSISSIppi")  \lev%
  {\rm{}the symbol with name} "mISSISSIppi"
(eq? 'bitBlt (string->symbol "bitBlt"))     \lev  \schtrue
(eq? 'JollyWog
     (string->symbol
       (symbol->string 'JollyWog)))  \lev  \schtrue
(string=? "K. Harper, M.D."
          (symbol->string
            (string->symbol "K. Harper, M.D.")))  \lev  \schtrue%
\end{scheme}

\end{entry}


\section{Characters}
\label{charactersection}

\mainindex{Unicode}
\mainindex{scalar value}

\defining{Characters} are objects that represent Unicode scalar
values~\cite{Unicode}.

\begin{note}
  Unicode defines a standard mapping between sequences of
  \textit{Unicode scalar values}\mainindex{Unicode scalar
    value}\mainindex{scalar value} (integers in the range 0 to
  \#x10FFFF, excluding the range \#xD800 to \#xDFFF) in the latest
  version of the standard and human-readable ``characters''. More
  precisely, Unicode distinguishes between glyphs, which are printed
  for humans to read, and characters, which are abstract entities that
  map to glyphs (sometimes in a way that's sensitive to surrounding
  characters).  Furthermore, different sequences of scalar values
  sometimes correspond to the same character.  The relationships among
  scalar, characters, and glyphs are subtle and complex.

  Despite this complexity, most things that a literate human would
  call a ``character'' can be represented by a single Unicode scalar
  value (although several sequences of Unicode scalar values may
  represent that same character). For example, Roman letters, Cyrillic
  letters, Hebrew consonants, and most Chinese characters fall into
  this category.
  
  Unicode scalar values exclude the range \#xD800 to \#xDFFF, which
  are part of the range of Unicode \textit{code points}\mainindex{code
    point}.  However, the Unicode code points in this range,
  the so-called \textit{surrogates}\mainindex{surrogate}, are an
  artifact of the UTF-16 encoding, and can only appear in specific
  Unicode encodings, and even then only in pairs that encode scalar
  values.  Consequently, all characters represent code points, but the
  surrogate code points do not have representations as characters.
\end{note}

\begin{entry}{%
\proto{char?}{ obj}{procedure}}

Returns \schtrue{} if \var{obj} is a character, otherwise returns \schfalse.

\end{entry}

\begin{entry}{%
\proto{char->integer}{ char}{procedure}
\proto{integer->char}{ \vr{sv}}{procedure}}

\domain{\var{Sv} must be a Unicode scalar value, i.e., a non-negative exact
  integer object in $\left[0, \#x\textrm{D7FF}\right] \cup
  \left[\#x\textrm{E000}, \#x\textrm{10FFFF}\right]$.}

Given a character, {\cf char\coerce{}integer} returns its Unicode scalar value
as an exact integer object.  
For a Unicode scalar value \var{sv}, {\cf integer\coerce{}char}
returns its associated character.

\begin{scheme}
(integer->char 32) \ev \sharpsign\backwhack{}space
(char->integer (integer->char 5000))
\ev 5000
(integer->char \sharpsign{}\backwhack{}xD800) \xev \exception{\&assertion}%
\end{scheme}
\end{entry}


\begin{entry}{%
\proto{char=?}{ \vari{char} \varii{char} \variii{char} \dotsfoo}{procedure}
\proto{char<?}{ \vari{char} \varii{char} \variii{char} \dotsfoo}{procedure}
\proto{char>?}{ \vari{char} \varii{char} \variii{char} \dotsfoo}{procedure}
\proto{char<=?}{ \vari{char} \varii{char} \variii{char} \dotsfoo}{procedure}
\proto{char>=?}{ \vari{char} \varii{char} \variii{char} \dotsfoo}{procedure}}

\label{characterequality}
These procedures impose a total ordering on the set of characters
according to their Unicode scalar values.

\begin{scheme}
(char<? \sharpsign\backwhack{}z \sharpsign\backwhack{}\ss) \ev \schtrue
(char<? \sharpsign\backwhack{}z \sharpsign\backwhack{}Z) \ev \schfalse%
\end{scheme}

\end{entry}

\section{Strings}
\label{stringsection}

Strings are sequences of characters.  

\vest The {\em length} of a string is the number of characters that it
contains.  This number is fixed when the
string is created.  The \defining{valid indices} of a string are the
integers less than the length of the string.  The first
character of a string has index 0, the second has index 1, and so on.

\begin{entry}{%
\proto{string?}{ obj}{procedure}}

Returns \schtrue{} if \var{obj} is a string, otherwise returns \schfalse.
\end{entry}


\begin{entry}{%
\proto{make-string}{ k}{procedure}
\rproto{make-string}{ k char}{procedure}}

Returns a newly allocated string of
length \var{k}.  If \var{char} is given, then all elements of the string
are initialized to \var{char}, otherwise the contents of the
\var{string} are unspecified.

\end{entry}

\begin{entry}{%
\proto{string}{ char \dotsfoo}{procedure}}

Returns a newly allocated string composed of the arguments.

\end{entry}

\begin{entry}{%
\proto{string-length}{ string}{procedure}}

Returns the number of characters in the given \var{string} as an exact
integer object.
\end{entry}


\begin{entry}{%
\proto{string-ref}{ string k}{procedure}}

\domain{\var{K} must be a valid index of \var{string}.}
The {\cf string-ref} procedure returns character \vr{k} of \var{string} using zero-origin indexing.

\begin{note}
  Implementors should make {\cf string-ref} run in constant
  time.
\end{note}
\end{entry}

\begin{entry}{%
\proto{string=?}{ \vari{string} \varii{string} \variii{string} \dotsfoo}{procedure}}

Returns \schtrue{} if the strings are the same length and contain the same
characters in the same positions.  Otherwise, the {\cf string=?}
procedure returns \schfalse.

\begin{scheme}
(string=? "Stra\ss{}e" "Strasse") \lev \schfalse%
\end{scheme}
\end{entry}

\begin{entry}{%
\proto{string<?}{ \vari{string} \varii{string} \variii{string} \dotsfoo}{procedure}
\proto{string>?}{ \vari{string} \varii{string} \variii{string} \dotsfoo}{procedure}
\proto{string<=?}{ \vari{string} \varii{string} \variii{string} \dotsfoo}{procedure}
\proto{string>=?}{ \vari{string} \varii{string} \variii{string} \dotsfoo}{procedure}}

These procedures are the lexicographic extensions to strings of the
corresponding orderings on characters.  For example, {\cf string<?}\ is
the lexicographic ordering on strings induced by the ordering
{\cf char<?}\ on characters.  If two strings differ in length but
are the same up to the length of the shorter string, the shorter string
is considered to be lexicographically less than the longer string.

\begin{scheme}
(string<? "z" "\ss") \ev \schtrue
(string<? "z" "zz") \ev \schtrue
(string<? "z" "Z") \ev \schfalse%
\end{scheme}
\end{entry}


\begin{entry}{%
\proto{substring}{ string start end}{procedure}}

\domain{\var{String} must be a string, and \var{start} and \var{end}
must be exact integer objects satisfying
$$0 \leq \var{start} \leq \var{end} \leq \hbox{\tt(string-length \var{string})\rm.}$$}
The {\cf substring} procedure returns a newly allocated string formed from the characters of
\var{string} beginning with index \var{start} (inclusive) and ending with index
\var{end} (exclusive).
\end{entry}


\begin{entry}{%
\proto{string-append}{ \var{string} \dotsfoo}{procedure}}

Returns a newly allocated string whose characters form the concatenation of the
given strings.
\end{entry}


\begin{entry}{%
\proto{string->list}{ string}{procedure}
\proto{list->string}{ list}{procedure}}

\domain{\var{List} must be a list of characters.}
The {\cf string\coerce{}list} procedure returns a newly allocated list of the
characters that make up the given string.  The {\cf
  list\coerce{}string} procedure
returns a newly allocated string formed from the characters in 
\var{list}. The {\cf string\coerce{}list}
and {\cf list\coerce{}string} procedures are
inverses so far as {\cf equal?}\ is concerned.  
\end{entry}

\begin{entry}{%
\proto{string-for-each}{ proc \vari{string} \varii{string} \dotsfoo}{procedure}}

\domain{The \var{string}s must all have the same length.  \var{Proc}
  should accept as many arguments as there are {\it string}s.}
The {\cf string-for-each} procedure applies \var{proc}
element-wise to the characters of the
\var{string}s for its side effects,  in order from the first characters to the
last.
\var{Proc} is always called in the same dynamic environment 
as {\cf string-for-each} itself.
The return values of {\cf string-for-each} \areunspecified.

Analogous to {\cf for-each}.

\implresp The implementation must check the restrictions
on \var{proc} to the extent performed by applying it as described.
An
implementation may check whether \var{proc} is an appropriate argument
before applying it.
\end{entry}

\begin{entry}{%
\proto{string-copy}{ string}{procedure}}

Returns a newly allocated copy of the given \var{string}.

\end{entry}

\section{Vectors}
\label{vectorsection}

Vectors are heterogeneous structures whose elements are indexed
by integers.  A vector typically occupies less space than a list
of the same length, and the average time needed to access a randomly
chosen element is typically less for the vector than for the list.

\vest The {\em length} of a vector is the number of elements that it
contains.  This number is a non-negative integer that is fixed when the
vector is created.
The {\em valid indices}\index{valid indices} of a
vector are the exact non-negative integer objects less than the length of the
vector.  The first element in a vector is indexed by zero, and the last
element is indexed by one less than the length of the vector.

Like list constants, vector constants must be quoted:

\begin{scheme}
'\#(0 (2 2 2 2) "Anna")  \lev  \#(0 (2 2 2 2) "Anna")%
\end{scheme}

\begin{entry}{%
\proto{vector?}{ obj}{procedure}}
 
Returns \schtrue{} if \var{obj} is a vector.  Otherwise the procedure
returns \schfalse.
\end{entry}


\begin{entry}{%
\proto{make-vector}{ k}{procedure}
\rproto{make-vector}{ k fill}{procedure}}

Returns a newly allocated vector of \var{k} elements.  If a second
argument is given, then each element is initialized to \var{fill}.
Otherwise the initial contents of each element is unspecified.

\end{entry}


\begin{entry}{%
\proto{vector}{ obj \dotsfoo}{procedure}}

Returns a newly allocated vector whose elements contain the given
arguments.  Analogous to {\cf list}.

\begin{scheme}
(vector 'a 'b 'c)               \ev  \#(a b c)%
\end{scheme}
\end{entry}


\begin{entry}{%
\proto{vector-length}{ vector}{procedure}}

Returns the number of elements in \var{vector} as an exact integer object.
\end{entry}


\begin{entry}{%
\proto{vector-ref}{ vector k}{procedure}}

\domain{\var{K} must be a valid index of \var{vector}.}
The {\cf vector-ref} procedure returns the contents of element \vr{k} of
\var{vector}.

\begin{scheme}
(vector-ref '\#(1 1 2 3 5 8 13 21) 5)  \lev  8%
\end{scheme}
\end{entry}


\begin{entry}{%
\proto{vector-set!}{ vector k obj}{procedure}}

\domain{\var{K} must be a valid index of \var{vector}.}
The {\cf vector-set!} procedure stores \var{obj} in element \vr{k} of
\var{vector}, and returns \unspecifiedreturn.

Passing an immutable vector to {\cf vector-set!} should cause an exception
with condition type {\cf\&assertion} to be raised.

\begin{scheme}
(let ((vec (vector 0 '(2 2 2 2) "Anna")))
  (vector-set! vec 1 '("Sue" "Sue"))
  vec)      \lev  \#(0 ("Sue" "Sue") "Anna")

(vector-set! '\#(0 1 2) 1 "doe")  \lev  \unspecified
             ; \textrm{constant vector}
             ; \textrm{should raise} \exception{\&assertion}%
\end{scheme}

\end{entry}


\begin{entry}{%
\proto{vector->list}{ vector}{procedure}
\proto{list->vector}{ list}{procedure}}

The {\cf vector->list} procedure returns a newly allocated list of the objects contained
in the elements of \var{vector}.  The {\cf list->vector} procedure returns a newly
created vector initialized to the elements of the list \var{list}.

\begin{scheme}
(vector->list '\#(dah dah didah))  \lev  (dah dah didah)
(list->vector '(dididit dah))   \lev  \#(dididit dah)%
\end{scheme}
\end{entry}


\begin{entry}{%
\proto{vector-fill!}{ vector fill}{procedure}}

Stores \var{fill} in every element of \var{vector}
and returns \unspecifiedreturn.
\end{entry}

\begin{entry}{%
\proto{vector-map}{ proc \vari{vector} \varii{vector} \dotsfoo}{procedure}}

\domain{The \var{vector}s must all have the same length.  \var{Proc}
  should accept as many arguments as there are {\it vector}s and return a
  single value.}

The {\cf vector-map} procedure applies \var{proc} element-wise to the elements of the
\var{vector}s and returns a vector of the results, in order.
\var{Proc} is always called in the same dynamic environment 
as {\cf vector-map} itself.
The order in which \var{proc} is applied to the elements of the
\var{vector}s is unspecified.
If multiple returns occur from {\cf vector-map}, the return
values returned by earlier returns are not mutated.


Analogous to {\cf map}.

\implresp The implementation must check the restrictions
on \var{proc} to the extent performed by applying it as described.
An
implementation may check whether \var{proc} is an appropriate argument
before applying it.
\end{entry}


\begin{entry}{%
\proto{vector-for-each}{ proc \vari{vector} \varii{vector} \dotsfoo}{procedure}}

\domain{The \var{vector}s must all have the same length.  \var{Proc}
  should accept as many arguments as there are {\it vector}s.}
The {\cf vector-for-each} procedure applies \var{proc}
element-wise to the elements of the
\var{vector}s for its side effects,  in order from the first elements to the
last.
\var{Proc} is always called in the same dynamic environment 
as {\cf vector-for-each} itself.
The return values of {\cf vector-for-each} \areunspecified.

Analogous to {\cf for-each}.

\implresp The implementation must check the restrictions
on \var{proc} to the extent performed by applying it as described.
An
implementation may check whether \var{proc} is an appropriate argument
before applying it.
\end{entry}

\section{Errors and violations}
\label{errorviolation}

\begin{entry}{%
\proto{error}{ who message \vari{irritant} \dotsfoo}{procedure}
\proto{assertion-violation}{ who message \vari{irritant} \dotsfoo}{procedure}}

\domain{\var{Who} must be a string or a symbol or \schfalse{}.
  \var{Message} must be a string.
  The \var{irritant}s are arbitrary objects.}

These procedures raise an exception.  The {\cf error}
procedure should be called when an error has occurred, typically caused by
something that has gone wrong in the interaction of the program with the
external world or the user.  The {\cf assertion-violation} procedure
should be called when an invalid call to a procedure was made, either passing an
invalid number of arguments, or passing an argument that it is not
specified to handle.

The \var{who} argument should describe the procedure or operation that
detected the exception.  The \var{message} argument should describe
the exceptional situation.  The \var{irritant}s should be the arguments
to the operation that detected the operation.

The condition object provided with the exception (see
library chapter~\extref{lib:exceptionsconditionschapter}{Exceptions
  and conditions}) has the following condition types:
%
\begin{itemize}
\item If \var{who} is not \schfalse, the condition has condition type
  {\cf \&who}, with \var{who} as the value of its field.  In
  that case, \var{who} should be the name of the procedure or entity that
  detected the exception.  If it is \schfalse, the condition does not
  have condition type {\cf \&who}.
\item The condition has condition type {\cf \&message}, with
  \var{message} as the value of its field.
\item The condition has condition type {\cf \&irritants}, and its
  field has as its value a list of the \var{irritant}s.
\end{itemize}
%
Moreover, the condition created by {\cf error} has condition type 
{\cf \&error}, and the condition created by {\cf assertion-\hp{}violation} has
condition type {\cf \&assertion}.

\begin{scheme}
(define (fac n)
  (if (not (integer-valued? n))
      (assertion-violation
       'fac "non-integral argument" n))
  (if (negative? n)
      (assertion-violation
       'fac "negative argument" n))
  (letrec
    ((loop (lambda (n r)
             (if (zero? n)
                 r
                 (loop (- n 1) (* r n))))))
      (loop n 1)))

(fac 5) \ev 120
(fac 4.5) \xev \exception{\&assertion}
(fac -3) \xev \exception{\&assertion}%
\end{scheme}
\end{entry}

\begin{entry}{%
\proto{assert}{ \hyper{expression}}{\exprtype}}

An {\cf assert} form is evaluated by evaluating \hyper{expression}.
If \hyper{expression} returns a true value, that value is returned
from the {\cf assert} expression.  If \hyper{expression} returns
\schfalse, an exception with condition types {\cf \&assertion} and
{\cf \&message} is raised.  The message provided in the condition
object is implementation-dependent.

\begin{note}
  Implementations should exploit the fact that
  {\cf assert} is syntax to provide as much information as possible
  about the location of the assertion failure.
\end{note}
\end{entry}

\section{Control features}
\label{controlsection}
\label{valuessection}
 
This chapter describes various primitive procedures which control the
flow of program execution in special ways.

\begin{entry}{%
\proto{apply}{ proc \vari{arg} $\ldots$ rest-args}{procedure}}

\domain{\var{Rest-args} must be a list.
 \var{Proc} should accept $n$ arguments, where $n$ is
  number of \var{arg}s plus the length of \var{rest-args}.}
The {\cf apply} procedure calls \var{proc} with the elements of the list
{\cf(append (list \vari{arg} \dotsfoo) \var{rest-args})} as the actual
arguments.

If a call to {\cf apply} occurs in a tail context, the call
to \var{proc} is also in a tail context.

\begin{scheme}
(apply + (list 3 4))              \ev  7

(define compose
  (lambda (f g)
    (lambda args
      (f (apply g args)))))

((compose sqrt *) 12 75)              \ev  30%
\end{scheme}
\end{entry}


\begin{entry}{%
\proto{call-with-current-continuation}{ proc}{procedure}
\proto{call/cc}{ proc}{procedure}}

\label{continuations} \domain{\var{Proc} should accept one
argument.} The procedure {\cf call-with-current-continuation} 
(which is the same as the procedure {\cf call/cc}) packages
the current continuation as an ``escape
procedure''\mainindex{escape procedure} and passes it as an argument to
\var{proc}.  The escape procedure is a Scheme procedure that, if it is
later called, will abandon whatever continuation is in effect at that later
time and will instead reinstate the continuation that was in effect
when the escape procedure was created.  Calling the escape procedure
may cause the invocation of \var{before} and \var{after} procedures installed using
\ide{dynamic-wind}.

The escape procedure accepts the same number of arguments as the
continuation of the original call to {\cf call-\hp{}with-\hp{}current-\hp{}continuation}.

The escape procedure that is passed to \var{proc} has
unlimited extent just like any other procedure in Scheme.  It may be stored
in variables or data structures and may be called as many times as desired.

If a call to {\cf call-with-current-continuation} occurs in a tail
context, the call to \var{proc} is also in a tail context.

The following examples show only some ways in which
{\cf call-with-current-continuation} is used.  If all real uses were as
simple as these examples, there would be no need for a procedure with
the power of {\cf call-\hp{}with-\hp{}current-\hp{}continuation}.

\begin{scheme}
(call-with-current-continuation
  (lambda (exit)
    (for-each (lambda (x)
                (if (negative? x)
                    (exit x)))
              '(54 0 37 -3 245 19))
    \schtrue))                        \ev  -3

(define list-length
  (lambda (obj)
    (call-with-current-continuation
      (lambda (return)
        (letrec ((r
                  (lambda (obj)
                    (cond ((null? obj) 0)
                          ((pair? obj)
                           (+ (r (cdr obj)) 1))
                          (else (return \schfalse))))))
          (r obj))))))

(list-length '(1 2 3 4))            \ev  4

(list-length '(a b . c))            \ev  \schfalse%

(call-with-current-continuation procedure?)
                            \ev  \schtrue%
\end{scheme}

\begin{note}
  Calling an escape procedure reenters the dynamic extent of the call
  to {\cf call-with-current-continuation}, and thus restores its
  dynamic environment; see section~\ref{dynamicenvironmentsection}.
\end{note}

\end{entry}

\begin{entry}{%
\proto{values}{ obj $\ldots$}{procedure}}

Delivers all of its arguments to its continuation.
The {\cf values} procedure might be defined as follows:
\begin{scheme}
(define (values . things)
  (call-with-current-continuation 
    (lambda (cont) (apply cont things))))%
\end{scheme}

The continuations of all non-final expressions within a sequence of
expressions, such as in {\cf lambda}, {\cf begin}, {\cf let}, {\cf
  let*}, {\cf letrec}, {\cf letrec*}, {\cf let-values}, {\cf
  let*-values}, {\cf case}, and {\cf cond} forms, usually take an
arbitrary number of values.

Except for these and the continuations created by {\cf
  call-\hp{}with-\hp{}values}, {\cf let-values}, and {\cf let*-values},
continuations implicitly accepting a single value, such as the
continuations of \hyper{operator} and \hyper{operand}s of procedure
calls or the \hyper{test} expressions in conditionals, take exactly
one value.  The effect of passing an inappropriate number of values to
such a continuation is undefined.
\end{entry}

\begin{entry}{%
\proto{call-with-values}{ producer consumer}{procedure}}

\domain{\var{Producer} must be a procedure and should accept zero
  arguments.  \var{Consumer} must be a procedure and should accept as many
  values as \var{producer} returns.}
The {\cf call-\hp{}with-\hp{}values} procedure calls \var{producer} with no arguments and
a continuation that, when passed some values, calls the
\var{consumer} procedure with those values as arguments.
The continuation for the call to \var{consumer} is the
continuation of the call to {\tt call-with-values}.

\begin{scheme}
(call-with-values (lambda () (values 4 5))
                  (lambda (a b) b))
                                                   \ev  5

(call-with-values * -)                             \ev  -1%
\end{scheme}

If a call to {\cf call-with-values} occurs in a tail context, the call
to \var{consumer} is also in a tail context.

\implresp After \var{producer} returns, the implementation must check
that \var{consumer} accepts as many values as \var{consumer} has
returned.
\end{entry}

\begin{entry}{%
\proto{dynamic-wind}{ before thunk after}{procedure}}

\domain{\var{Before}, \var{thunk}, and \var{after} must be procedures,
  and each should accept zero arguments.  These procedures may return
  any number of values.}  The {\cf dynamic-wind} procedure calls
\var{thunk} without arguments, returning the results of this call.
Moreover, {\cf dynamic-wind} calls \var{before} without arguments
whenever the dynamic extent of the call to \var{thunk} is entered, and
\var{after} without arguments whenever the dynamic extent of the call
to \var{thunk} is exited.  Thus, in the absence of calls to escape
procedures created by {\cf call-with-current-continuation}, {\cf
  dynamic-wind} calls \var{before}, \var{thunk}, and \var{after}, in
that order.

While the calls to \var{before} and \var{after} are not considered to be
within the dynamic extent of the call to \var{thunk}, calls to the \var{before}
and \var{after} procedures of any other calls to {\cf dynamic-wind} that occur
within the dynamic extent of the call to \var{thunk} are considered to be
within the dynamic extent of the call to \var{thunk}.

More precisely, an escape procedure transfers control out of the
dynamic extent of a set of zero or more active {\cf dynamic-wind}
calls $x\ \dots$ and transfer control into the dynamic extent
of a set of zero or more active {\cf dynamic-wind} calls
$y\ \dots$.  
It leaves the dynamic extent of the most recent $x$ and calls without
arguments the corresponding \var{after} procedure.
If the \var{after} procedure returns, the escape procedure proceeds to
the next most recent $x$, and so on.
Once each $x$ has been handled in this manner,
the escape procedure calls without arguments the \var{before} procedure
corresponding to the least recent $y$.
If the \var{before} procedure returns, the escape procedure reenters the
dynamic extent of the least recent $y$ and proceeds with the next least
recent $y$, and so on.
Once each $y$ has been handled in this manner, control is transferred to
the continuation packaged in the escape procedure.

\implresp The implementation must check the restrictions on
\var{thunk} and \var{after} only if they are actually called.

\begin{scheme}
(let ((path '())
      (c \#f))
  (let ((add (lambda (s)
               (set! path (cons s path)))))
    (dynamic-wind
      (lambda () (add 'connect))
      (lambda ()
        (add (call-with-current-continuation
               (lambda (c0)
                 (set! c c0)
                 'talk1))))
      (lambda () (add 'disconnect)))
    (if (< (length path) 4)
        (c 'talk2)
        (reverse path))))
    \lev (connect talk1 disconnect
               connect talk2 disconnect)

(let ((n 0))
  (call-with-current-continuation
    (lambda (k)
      (dynamic-wind
        (lambda ()
          (set! n (+ n 1))
          (k))
        (lambda ()
          (set! n (+ n 2)))
        (lambda ()
          (set! n (+ n 4))))))
  n) \ev 1

(let ((n 0))
  (call-with-current-continuation
    (lambda (k)
      (dynamic-wind
        values
        (lambda ()
          (dynamic-wind
            values
            (lambda ()
              (set! n (+ n 1))
              (k))
            (lambda ()
              (set! n (+ n 2))
              (k))))
        (lambda ()
          (set! n (+ n 4))))))
  n) \ev 7%
\end{scheme}

\begin{note}
  Entering a dynamic extent restores its dynamic environment; see
  section~\ref{dynamicenvironmentsection}.
\end{note}
\end{entry}

\section{Iteration}\unsection

\begin{entry}{%
\rproto{let}{ \hyper{variable} \hyper{bindings} \hyper{body}}{\exprtype}}

\label{namedlet}
``Named {\cf let}'' is a variant on the syntax of \ide{let} that provides
a general looping construct and may also be used to express
recursion.
It has the same syntax and semantics as ordinary {\cf let}
except that \hyper{variable} is bound within \hyper{body} to a procedure
whose parameters are the bound variables and whose body is
\hyper{body}.  Thus the execution of \hyper{body} may be repeated by
invoking the procedure named by \hyper{variable}.

%                                              |  <-- right margin
\begin{scheme}
(let loop ((numbers '(3 -2 1 6 -5))
           (nonneg '())
           (neg '()))
  (cond ((null? numbers) (list nonneg neg))
        ((>= (car numbers) 0)
         (loop (cdr numbers)
               (cons (car numbers) nonneg)
               neg))
        ((< (car numbers) 0)
         (loop (cdr numbers)
               nonneg
               (cons (car numbers) neg))))) %
  \lev  ((6 1 3) (-5 -2))%
\end{scheme}

\end{entry}

\section{Quasiquotation}\unsection
\label{quasiquotesection}

\begin{entry}{%
\proto{quasiquote}{ \hyper{qq template}}{\exprtype}
\litproto{unquote}
\litproto{unquote-splicing}}

``Backquote'' or ``quasiquote''\index{backquote} expressions are useful
for constructing a list or vector structure when some but not all of the
desired structure is known in advance.  

\syntax \hyper{Qq template} should be as specified by the grammar at
the end of this entry.

\semantics If no
{\cf unquote} or {\cf unquote-splicing} forms
appear within the \hyper{qq template}, the result of
evaluating
{\cf (quasiquote \hyper{qq template})} is equivalent to the result of evaluating
{\cf (quote \hyper{qq template})}.

If an {\cf (unquote \hyper{expression} \dotsfoo)} form appears inside a
\hyper{qq template}, however, the \hyper{expression}s are evaluated
(``unquoted'') and their results are inserted into the structure instead
of the {\cf unquote} form.

If an {\cf (unquote-splicing \hyper{expression} \dotsfoo)} form
appears inside a \hyper{qq template}, then the \hyper{expression}s must
evaluate to lists; the opening and closing parentheses of the lists are
then ``stripped away'' and the elements of the lists are inserted in
place of the {\cf unquote-splicing} form.

Any {\cf unquote-splicing} or multi-operand {\cf unquote} form must
appear only within a list or vector \hyper{qq template}.

As noted in section~\ref{abbreviationsection},
{\cf (quasiquote \hyper{qq template})} may be abbreviated
\backquote\hyper{qq template},
{\cf (unquote \hyper{expression})} may be abbreviated
{\cf,}\hyper{expression}, and
{\cf (unquote-splicing \hyper{expression})} may be abbreviated
{\cf,}\atsign\hyper{expression}.

\begin{scheme}
`(list ,(+ 1 2) 4)  \ev  (list 3 4)
(let ((name 'a)) `(list ,name ',name)) %
          \lev  (list a (quote a))
`(a ,(+ 1 2) ,@(map abs '(4 -5 6)) b) %
          \lev  (a 3 4 5 6 b)
`(({\cf foo} ,(- 10 3)) ,@(cdr '(c)) . ,(car '(cons))) %
          \lev  ((foo 7) . cons)
`\#(10 5 ,(sqrt 4) ,@(map sqrt '(16 9)) 8) %
          \lev  \#(10 5 2 4 3 8)
(let ((name 'foo))
  `((unquote name name name)))%
          \lev (foo foo foo)
(let ((name '(foo)))
  `((unquote-splicing name name name)))%
          \lev (foo foo foo)
(let ((q '((append x y) (sqrt 9))))
  ``(foo ,,@q)) \lev `(foo
                 (unquote (append x y) (sqrt 9)))
(let ((x '(2 3))
      (y '(4 5)))
  `(foo (unquote (append x y) (sqrt 9)))) \lev (foo (2 3 4 5) 3)%
\end{scheme}

Quasiquote forms may be nested.  Substitutions are made only for
unquoted components appearing at the same nesting level
as the outermost {\cf quasiquote}.  The nesting level increases by one inside
each successive quasiquotation, and decreases by one inside each
unquotation.

\begin{scheme}
`(a `(b ,(+ 1 2) ,(foo ,(+ 1 3) d) e) f) %
          \lev  (a `(b ,(+ 1 2) ,(foo 4 d) e) f)
(let ((name1 'x)
      (name2 'y))
  `(a `(b ,,name1 ,',name2 d) e)) %
          \lev  (a `(b ,x ,'y d) e)%
\end{scheme}

A {\cf quasiquote} expression may return either fresh, mutable objects
or literal structure for any structure that is constructed at run time
during the evaluation of the expression.  Portions that do not need to
be rebuilt are always literal.  Thus,
%
\begin{scheme}
(let ((a 3)) `((1 2) ,a ,4 ,'five 6))%
\end{scheme}
%
may be equivalent to either of the following expressions:
%
\begin{scheme}
'((1 2) 3 4 five 6)
(let ((a 3)) 
  (cons '(1 2)
        (cons a (cons 4 (cons 'five '(6))))))%
\end{scheme}
%
However, it is not equivalent to this expression:
%
\begin{scheme}
(let ((a 3)) (list (list 1 2) a 4 'five 6))
\end{scheme}
%
It is a syntax violation if any of the identifiers
\ide{quasiquote}, \ide{unquote}, or \ide{unquote-splicing} appear in
positions within a \hyper{qq template} otherwise than as described above.

The following grammar for quasiquote expressions is not context-free.
It is presented as a recipe for generating an infinite number of
production rules.  Imagine a copy of the following rules for $D = 1, 2,
3, \ldots$.  $D$ keeps track of the nesting depth.

\begin{grammar}%
\meta{qq template} \: \meta{qq template 1}
\meta{qq template 0} \: \meta{expression}
\meta{quasiquotation $D$} \: (quasiquote \meta{qq template $D$})
\meta{qq template $D$} \: \meta{lexeme datum}
\>    \| \meta{list qq template $D$}
\>    \| \meta{vector qq template $D$}
\>    \| \meta{unquotation $D$}
\meta{list qq template $D$} \: (\arbno{\meta{qq template or splice $D$}})
\>    \| (\atleastone{\meta{qq template or splice $D$}} .\ \meta{qq template $D$})
\>    \| \meta{quasiquotation $D+1$}
\meta{vector qq template $D$} \: \#(\arbno{\meta{qq template or splice $D$}})
\meta{unquotation $D$} \: (unquote \meta{qq template $D-1$})
\meta{qq template or splice $D$} \: \meta{qq template $D$}
\>    \| \meta{splicing unquotation $D$}
\meta{splicing unquotation $D$} \:
\>\> (unquote-splicing \arbno{\meta{qq template $D-1$}})
\>    \| (unquote \arbno{\meta{qq template $D-1$}}) %
\end{grammar}

In \meta{quasiquotation}s, a \meta{list qq template $D$} can sometimes
be confused with either an \meta{un\-quota\-tion $D$} or a \meta{splicing
un\-quo\-ta\-tion $D$}.  The interpretation as an
\meta{un\-quo\-ta\-tion} or \meta{splicing
un\-quo\-ta\-tion $D$} takes precedence.

\end{entry}

\section{Binding constructs for syntactic keywords}
\label{bindsyntax}

The {\cf let-syntax} and {\cf letrec-syntax} forms 
bind keywords.
Like a {\cf begin} form, a {\cf let-syntax} or {\cf letrec-syntax} form
may appear in a definition context, in which case it is treated as a
definition, and the forms in the body must also be
definitions.
A {\cf let-syntax} or {\cf letrec-syntax} form may also appear in an
expression context, in which case the forms within their bodies must be
expressions.

\begin{entry}{%
\proto{let-syntax}{ \hyper{bindings} \hyper{form} \dotsfoo}{\exprtype}}

\syntax
\hyper{Bindings} must have the form
\begin{scheme}
((\hyper{keyword} \hyper{expression}) \dotsfoo)%
\end{scheme}
Each \hyper{keyword} is an identifier,
and each \hyper{expression} is 
an expression that evaluates, at macro-expansion
time, to a \textit{transformer}\index{transformer}\index{macro transformer}.
Transformers may be created by {\cf syntax-rules} or {\cf identifier-syntax}
(see section~\ref{syntaxrulessection}) or by one of the other mechanisms
described in library chapter~\extref{lib:syntaxcasechapter}{{\cf syntax-case}}.  It is a
syntax violation for \hyper{keyword} to appear more than once in the list of keywords
being bound.

\semantics
The \hyper{form}s are expanded in the syntactic environment
obtained by extending the syntactic environment of the
{\cf let-syntax} form with macros whose keywords are
the \hyper{keyword}s, bound to the specified transformers.
Each binding of a \hyper{keyword} has the \hyper{form}s as its region.

The \hyper{form}s of a {\cf let-syntax}
form are treated, whether in definition or expression context, as if
wrapped in an implicit {\cf begin}; see section~\ref{begin}.
Thus definitions in the result of expanding the \hyper{form}s have
the same region as any definition appearing in place of the {\cf
  let-syntax} form would have.

\implresp The implementation should detect if the value of
\hyper{expression} cannot possibly be a transformer.

\begin{scheme}
(let-syntax ((when (syntax-rules ()
                     ((when test stmt1 stmt2 ...)
                      (if test
                          (begin stmt1
                                 stmt2 ...))))))
  (let ((if \schtrue))
    (when if (set! if 'now))
    if))                           \ev  now

(let ((x 'outer))
  (let-syntax ((m (syntax-rules () ((m) x))))
    (let ((x 'inner))
      (m))))                       \ev  outer%

(let ()
  (let-syntax
    ((def (syntax-rules ()
            ((def stuff ...) (define stuff ...)))))
    (def foo 42))
  foo) \ev 42

(let ()
  (let-syntax ())
  5) \ev 5%
\end{scheme}

\end{entry}

\begin{entry}{%
\proto{letrec-syntax}{ \hyper{bindings} \hyper{form} \dotsfoo}{\exprtype}}

\syntax
Same as for {\cf let-syntax}.

\semantics
The \hyper{form}s are
expanded in the syntactic environment obtained by
extending the syntactic environment of the {\cf letrec-syntax}
form with macros whose keywords are the
\hyper{keyword}s, bound to the specified transformers.
Each binding of a \hyper{keyword} has the \hyper{bindings}
as well as the \hyper{form}s within its region,
so the transformers can
transcribe forms into uses of the macros
introduced by the {\cf letrec-syntax} form.

The \hyper{form}s of a {\cf letrec-syntax}
form are treated, whether in definition or expression context, as if
wrapped in an implicit {\cf begin}; see section~\ref{begin}.
Thus definitions in the result of expanding the \hyper{form}s have
the same region as any definition appearing in place of the {\cf
  letrec-syntax} form would have.

\implresp The implementation should detect if the value of
\hyper{expression} cannot possibly be a transformer.

\begin{scheme}
(letrec-syntax
  ((my-or (syntax-rules ()
            ((my-or) \schfalse)
            ((my-or e) e)
            ((my-or e1 e2 ...)
             (let ((temp e1))
               (if temp
                   temp
                   (my-or e2 ...)))))))
  (let ((x \schfalse)
        (y 7)
        (temp 8)
        (let odd?)
        (if even?))
    (my-or x
           (let temp)
           (if y)
           y)))        \ev  7%
\end{scheme}

The following example highlights how {\cf let-syntax}
and {\cf letrec-syntax} differ.

\begin{scheme}
(let ((f (lambda (x) (+ x 1))))
  (let-syntax ((f (syntax-rules ()
                    ((f x) x)))
               (g (syntax-rules ()
                    ((g x) (f x)))))
    (list (f 1) (g 1)))) \lev (1 2)

(let ((f (lambda (x) (+ x 1))))
  (letrec-syntax ((f (syntax-rules ()
                       ((f x) x)))
                  (g (syntax-rules ()
                       ((g x) (f x)))))
    (list (f 1) (g 1)))) \lev (1 1)%
\end{scheme}

The two expressions are identical except that the {\cf let-syntax} form
in the first expression is a {\cf letrec-syntax} form in the second.
In the first expression, the {\cf f} occurring in {\cf g} refers to
the {\cf let}-bound variable {\cf f}, whereas in the second it refers
to the keyword {\cf f} whose binding is established by the
{\cf letrec-syntax} form.
\end{entry}

\section{Macro transformers}
\label{syntaxrulessection}

\begin{entry}{%
\pproto{(syntax-rules (\hyper{literal} \dotsfoo) \hyper{syntax rule} \dotsfoo)}{\exprtype~({\cf expand})}
\litprotoexpandnoindex{\_}
\litprotoexpand{...}}
\mainschindex{syntax-rules}\schindex{\_}

\syntax Each \hyper{literal} must be an identifier.
Each \hyper{syntax rule} must have the following form:

\begin{scheme}
(\hyper{srpattern} \hyper{template})%
\end{scheme}

An \hyper{srpattern} is a restricted form of \hyper{pattern},
namely, a nonempty \hyper{pattern} in one of four parenthesized forms below
whose first subform is an identifier or an underscore {\cf \_}\schindex{\_}.
A \hyper{pattern} is an identifier, constant, or one of the following.

\begin{schemenoindent}
(\hyper{pattern} \ldots)
(\hyper{pattern} \hyper{pattern} \ldots . \hyper{pattern})
(\hyper{pattern} \ldots \hyper{pattern} \hyper{ellipsis} \hyper{pattern} \ldots)
(\hyper{pattern} \ldots \hyper{pattern} \hyper{ellipsis} \hyper{pattern} \ldots . \hyper{pattern})
\#(\hyper{pattern} \ldots)
\#(\hyper{pattern} \ldots \hyper{pattern} \hyper{ellipsis} \hyper{pattern} \ldots)%
\end{schemenoindent}

An \hyper{ellipsis} is the identifier ``{\cf ...}'' (three periods).\schindex{...}

A \hyper{template} is a pattern variable, an identifier that
is not a pattern
variable, a pattern datum, or one of the following.

\begin{scheme}
(\hyper{subtemplate} \ldots)
(\hyper{subtemplate} \ldots . \hyper{template})
\#(\hyper{subtemplate} \ldots)%
\end{scheme}

A \hyper{subtemplate} is a \hyper{template} followed by zero or more ellipses.

\semantics An instance of {\cf syntax-rules} evaluates, at
macro-expansion time, to a new macro
transformer by specifying a sequence of hygienic rewrite rules.  A use
of a macro whose keyword is associated with a transformer specified by
{\cf syntax-rules} is matched against the patterns contained in the
\hyper{syntax rule}s, beginning with the leftmost \hyper{syntax rule}.
When a match is found, the macro use is transcribed hygienically
according to the template.  It is a syntax violation when no match is found.

An identifier appearing within a \hyper{pattern} may be an underscore
(~{\cf \_}~), a literal identifier listed in the list of literals
{\cf (\hyper{literal} \dotsfoo)}, or an ellipsis (~{\cf ...}~).
All other identifiers appearing within a \hyper{pattern} are
\textit{pattern variables\mainindex{pattern variable}}.
It is a syntax violation if an ellipsis or underscore appears in {\cf (\hyper{literal} \dotsfoo)}.

While the first subform of \hyper{srpattern} may be an identifier, the
identifier is not involved in the matching and
is not considered a pattern variable or literal identifier.

Pattern variables match arbitrary input subforms and
are used to refer to elements of the input.
It is a syntax violation if the same pattern variable appears more than once in a
\hyper{pattern}.

Underscores also match arbitrary input subforms but are not pattern variables
and so cannot be used to refer to those elements.
Multiple underscores may appear in a \hyper{pattern}.

A literal identifier matches an input subform if and only if the input
subform is an identifier and either both its occurrence in the input
expression and its occurrence in the list of literals have the same
lexical binding, or the two identifiers have the same name and both have
no lexical binding.

A subpattern followed by an ellipsis can match zero or more elements of
the input.

More formally, an input form $F$ matches a pattern $P$ if and only if
one of the following holds:

\begin{itemize}
\item $P$ is an underscore (~{\cf \_}~).

\item $P$ is a pattern variable.

\item $P$ is a literal identifier
and $F$ is an identifier such that both $P$ and $F$ would refer to the
same binding if both were to appear in the output of the macro outside
of any bindings inserted into the output of the macro.
(If neither of two like-named identifiers refers to any binding, i.e., both
are undefined, they are considered to refer to the same binding.)

\item $P$ is of the form
{\cf ($P_1$ \dotsfoo{} $P_n$)}
and $F$ is a list of $n$ elements that match $P_1$ through
$P_n$.

\item $P$ is of the form
{\cf ($P_1$ \dotsfoo{} $P_n$ . $P_x$)}
and $F$ is a list or improper list of $n$ or more elements
whose first $n$ elements match $P_1$ through $P_n$
and
whose $n$th cdr matches $P_x$.

\item $P$ is of the form
{\cf ($P_1$ \dotsfoo{} $P_k$ $P_e$ \hyper{ellipsis} $P_{m+1}$ \dotsfoo{} $P_n$)},
where \hyper{ellipsis} is the identifier {\cf ...}
and $F$ is a list of $n$
elements whose first $k$ elements match $P_1$ through $P_k$,
whose next $m-k$ elements each match $P_e$,
and
whose remaining $n-m$ elements match $P_{m+1}$ through $P_n$.

\item $P$ is of the form
{\cf ($P_1$ \dotsfoo{} $P_k$ $P_e$ \hyper{ellipsis} $P_{m+1}$ \dotsfoo{} $P_n$ . $P_x$)},
where \hyper{ellipsis} is the identifier {\cf ...}
and $F$ is a list or improper list of $n$
elements whose first $k$ elements match $P_1$ through $P_k$,
whose next $m-k$ elements each match $P_e$,
whose next $n-m$ elements match $P_{m+1}$ through $P_n$,
and 
whose $n$th and final cdr matches $P_x$.

\item $P$ is of the form
{\cf \#($P_1$ \dotsfoo{} $P_n$)}
and $F$ is a vector of $n$ elements that match $P_1$ through
$P_n$.

\item $P$ is of the form
{\cf \#($P_1$ \dotsfoo{} $P_k$ $P_e$ \hyper{ellipsis} $P_{m+1}$ \dotsfoo{} $P_n$)},
where \hyper{ellipsis} is the identifier {\cf ...}
and $F$ is a vector of $n$ or more elements
whose first $k$ elements match $P_1$ through $P_k$,
whose next $m-k$ elements each match $P_e$,
and
whose remaining $n-m$ elements match $P_{m+1}$ through $P_n$.

\item $P$ is a pattern datum (any nonlist, nonvector, nonsymbol
datum) and $F$ is equal to $P$ in the sense of the
{\cf equal?} procedure.
\end{itemize}

When a macro use is transcribed according to the template of the
matching \hyper{syntax rule}, pattern variables that occur in the
template are replaced by the subforms they match in the input.

Pattern data and identifiers that are not pattern variables
or ellipses are copied into the output.
A subtemplate followed by an ellipsis expands
into zero or more occurrences of the subtemplate.
Pattern variables that occur in subpatterns followed by one or more
ellipses may occur only in subtemplates that are
followed by (at least) as many ellipses.
These pattern variables are replaced in the output by the input
subforms to which they are bound, distributed as specified.
If a pattern variable is followed by more ellipses in the subtemplate
than in the associated subpattern, the input form is replicated as
necessary.
The subtemplate must contain at least one pattern variable from a
subpattern followed by an ellipsis, and for at least one such pattern
variable, the subtemplate must be followed by exactly as many ellipses as
the subpattern in which the pattern variable appears.
(Otherwise, the expander would not be able to determine how many times the
subform should be repeated in the output.)
It is a syntax violation if the constraints of this paragraph are not met.

A template of the form
{\cf (\hyper{ellipsis} \hyper{template})} is identical to \hyper{template}, except that
ellipses within the template have no special meaning.
That is, any ellipses contained within \hyper{template} are
treated as ordinary identifiers.
In particular, the template {\cf (... ...)} produces a single
ellipsis, {\cf ...}.
This allows syntactic abstractions to expand into forms containing
ellipses.

\begin{scheme}
(define-syntax be-like-begin
  (syntax-rules ()
    ((be-like-begin name)
     (define-syntax name
       (syntax-rules ()
         ((name expr (... ...))
          (begin expr (... ...))))))))

(be-like-begin sequence)
(sequence 1 2 3 4) \ev 4%
\end{scheme}

As an example for hygienic use of auxiliary identifier,
if \ide{let} and \ide{cond} are defined as in
section~\ref{let} and appendix~\ref{derivedformsappendix} then they
are hygienic (as required) and the following is not an error.

\begin{scheme}
(let ((=> \schfalse))
  (cond (\schtrue => 'ok)))           \ev ok%
\end{scheme}

The macro transformer for {\cf cond} recognizes {\cf =>}
as a local variable, and hence an expression, and not as the
identifier {\cf =>}, which the macro transformer treats
as a syntactic keyword.  Thus the example expands into

\begin{scheme}
(let ((=> \schfalse))
  (if \schtrue (begin => 'ok)))%
\end{scheme}

instead of

\begin{scheme}
(let ((=> \schfalse))
  (let ((temp \schtrue))
    (if temp ('ok temp))))%
\end{scheme}

which would result in an assertion violation.
\end{entry}

\begin{entry}{%
\proto{identifier-syntax}{ \hyper{template}}{\exprtype~({\cf expand})}
\pproto{(identifier-syntax}{\exprtype~({\cf expand})}}\\
{\tt\obeyspaces
  (\hyperi{id} \hyperi{template})\\
  ((set! \hyperii{id} \hyper{pattern})\\
   \hyperii{template}))\\
\litprotoexpandnoindex{set!}}

\syntax The \hyper{id}s must be identifiers.  The \hyper{template}s
must be as for {\cf syntax-rules}.

\semantics
When a keyword is bound to a transformer produced by the first form of
{\cf identifier-syntax}, references to the keyword within the scope
of the binding are replaced by \hyper{template}.

\begin{scheme}
(define p (cons 4 5))
(define-syntax p.car (identifier-syntax (car p)))
p.car \ev 4
(set! p.car 15) \ev \exception{\&syntax}%
\end{scheme}

The second, more general, form of {\cf identifier-syntax} permits
the transformer to determine what happens when {\cf set!} is used.
In this case, uses of the identifier by itself are replaced by
\hyperi{template}, and uses of {\cf set!} with the identifier are
replaced by \hyperii{template}.

\begin{scheme}
(define p (cons 4 5))
(define-syntax p.car
  (identifier-syntax
    (\_ (car p))
    ((set! \_ e) (set-car! p e))))
(set! p.car 15)
p.car           \ev 15
p               \ev (15 5)%
\end{scheme}

\end{entry}

\section{Tail calls and tail contexts}
\label{basetailcontextsection}

A {\em tail call}\mainindex{tail call} is a procedure call that occurs
in a {\em tail context}.  Tail contexts are defined inductively.  Note
that a tail context is always determined with respect to a particular lambda
expression.

\begin{itemize}
\item The last expression within the body of a lambda expression,
  shown as \hyper{tail expression} below, occurs in a tail context.
%
\begin{scheme}
(l\=ambda \hyper{formals}
  \>\arbno{\hyper{definition}} 
  \>\arbno{\hyper{expression}} \hyper{tail expression})%
\end{scheme}
%
\item If one of the following expressions is in a tail context,
then the subexpressions shown as \hyper{tail expression} are in a tail context.
These were derived from specifications of the syntax of the forms described in
this chapter by replacing some occurrences of \hyper{expression}
with \hyper{tail expression}.  Only those rules that contain tail contexts
are shown here.
%
\begin{scheme}
(if \hyper{expression} \hyper{tail expression} \hyper{tail expression})
(if \hyper{expression} \hyper{tail expression})

(cond \atleastone{\hyper{cond clause}})
(cond \arbno{\hyper{cond clause}} (else \hyper{tail sequence}))

(c\=ase \hyper{expression}
  \>\atleastone{\hyper{case clause}})
(c\=ase \hyper{expression}
  \>\arbno{\hyper{case clause}}
  \>(else \hyper{tail sequence}))

(and \arbno{\hyper{expression}} \hyper{tail expression})
(or \arbno{\hyper{expression}} \hyper{tail expression})

(let \hyper{bindings} \hyper{tail body})
(let \hyper{variable} \hyper{bindings} \hyper{tail body})
(let* \hyper{bindings} \hyper{tail body})
(letrec* \hyper{bindings} \hyper{tail body})
(letrec \hyper{bindings} \hyper{tail body})
(let-values \hyper{mv-bindings} \hyper{tail body})
(let*-values \hyper{mv-bindings} \hyper{tail body})

(let-syntax \hyper{bindings} \hyper{tail body})
(letrec-syntax \hyper{bindings} \hyper{tail body})

(begin \hyper{tail sequence})%
\end{scheme}
%
A \hyper{cond clause} is 
%
\begin{scheme}
(\hyper{test} \hyper{tail sequence})\textrm{,}%
\end{scheme}
a \hyper{case clause} is
%
\begin{scheme}
((\arbno{\hyper{datum}}) \hyper{tail sequence})\textrm{,}%
\end{scheme}
%
a \hyper{tail body} is
\begin{scheme}
\arbno{\hyper{definition}} \hyper{tail sequence}\textrm{,}%
\end{scheme}
%
and a \hyper{tail sequence} is
%
\begin{scheme}
\arbno{\hyper{expression}} \hyper{tail expression}\textrm{.}%
\end{scheme}%

\item
If a {\cf cond} expression is in a tail context, and has a clause of
the form {\cf (\hyperi{expression} => \hyperii{expression})}
then the (implied) call to
the procedure that results from the evaluation of \hyperii{expression} is in a
tail context.  \hyperii{Expression} itself is not in a tail context.

\end{itemize}

Certain built-in procedures must also perform tail calls.
The first argument passed to {\cf apply} and to
{\cf call-\hp{}with-\hp{}current-continuation}, and the second argument passed to
{\cf call-with-values}, must be called via a tail call.

In the following example the only tail call is the call to {\cf f}.
None of the calls to {\cf g} or {\cf h} are tail calls.  The reference to
{\cf x} is in a tail context, but it is not a call and thus is not a
tail call.
\begin{scheme}%
(lambda ()
  (if (g)
      (let ((x (h)))
        x)
      (and (g) (f))))%
\end{scheme}%

\begin{note}
Implementations may
recognize that some non-tail calls, such as the call to {\cf h}
above, can be evaluated as though they were tail calls.
In the example above, the {\cf let} expression could be compiled
as a tail call to {\cf h}. (The possibility of {\cf h} returning
an unexpected number of values can be ignored, because in that
case the effect of the {\cf let} is explicitly unspecified and
implementation-dependent.)
\end{note}

%%% Local Variables: 
%%% mode: latex
%%% TeX-master: "r6rs"
%%% End: 

    \par
\clearchaptergroupstar{Appendices}
\appendix
\chapter{Formal semantics}
\label{formalsemanticschapter}
%!TEX root = r6rs.tex

This appendix presents a non-normative, formal, operational semantics for Scheme, that is based on an earlier semantics~\cite{mf:scheme-op-sem}. It does not cover the entire language. The notable missing features are the macro system, I/O, and the numerical tower. The precise list of features included is given in section~\ref{sec:semantics:grammar}.

The core of the specification is a single-step term rewriting relation that indicates how an (abstract) machine behaves. In general, the report is not a complete specification, giving implementations freedom to behave differently, typically to allow optimizations. This underspecification shows up in two ways in the semantics. 

The first is reduction rules that reduce to special ``\textbf{unknown:} \textit{string}'' states (where the string provides a description of the unknown state). The intention is that rules that reduce to such states can be replaced with arbitrary reduction rules. The precise specification of how to replace those rules is given in section~\ref{sec:semantics:underspecification}.

The other is that the single-step relation relates one program to
multiple different programs, each corresponding to a legal transition
that an abstract machine might take. Accordingly we use the transitive
closure of the single step relation $\rightarrow^*$ to define the
semantics, \calS, as a function from programs (\calP)
to sets of observable results (\calR):
\begin{center}
\begin{tabular}{l}
$\calS : \calP \longrightarrow 2^{\calR}$ \\
$\calS(\calP) = \{ \scrO(\calA) \mid \calP \rightarrow^* \calA \}$
\end{tabular}
\end{center}
where the function $\scrO$ turns an answer ($\calA$) from the semantics into an observable result. Roughly, $\scrO$ is the identity function on simple base values, and returns a special tag for more complex values, like procedure and pairs.

So, an implementation conforms to the semantics if, for every program $\calP$, the implementation produces one of the results in $\calS(\calP)$ or, if the implementation loops forever, then there is an infinite reduction sequence starting at $\calP$, assuming that the reduction relation $\rightarrow$ has been adjusted to replace the \textbf{unknown:} states.

The precise definitions of $\calP$, $\calA$, $\calR$, and $\scrO$ are also given in section~\ref{sec:semantics:grammar}.

To help understand the semantics and how it behaves, we have
implemented it in PLT Redex. The implementation is available at the
report's website: \url{http://www.r6rs.org/}. All of the reduction
rules and the metafunctions shown in the figures in this semantics
were generated automatically from the source code.

\section{Background}

We assume the reader has a basic familiarity with context-sensitive
reduction semantics. Readers unfamiliar with this system may wish to
consult Felleisen and Flatt's monograph~\cite{ff:monograph} or Wright
and Felleisen~\cite{wf:type-soundness} for a thorough introduction,
including the relevant technical background, or an introduction to PLT
Redex~\cite{mfff:plt-redex} for a somewhat lighter one.

As a rough guide, we define the operational semantics of a language
via a relation on program terms, where the relation corresponds to a
single step of an abstract machine. The relation is defined using
evaluation contexts, namely terms with a distinguished place in them,
called \emph{holes}\index{hole}, where the next step of evaluation
occurs. We say that a term $e$ decomposes into an evaluation
context $E$ and another term $e'$ if $e$ is the
same as $E$ but with the hole replaced by $e'$. We write
$E[e']$ to indicate the term obtained by replacing the hole in
$E$ with $e'$.

For example, assuming that we have defined a grammar containing
non-terminals for evaluation contexts ($E$), expressions
($e$), variables ($x$), and values ($v$), we
would write:
%
\begin{displaymath}
  \begin{array}{l}
    E_1[\texttt{((}\sy{lambda}~\texttt{(}x_1 \cdots{}\texttt{)}~e_1\texttt{)}~v_1~\cdots\texttt{)}] \rightarrow
    \\
    E_1[\{ x_1 \cdots \mapsto v_1 \cdots \} e_1] ~~~~~ (\#x_1 = \#v_1)
  \end{array}
\end{displaymath}
%
to define the $\beta_v$ rewriting rule (as a part of the $\rightarrow$
single step relation). We use the names of the non-terminals (possibly
with subscripts) in a rewriting rule to restrict the application of
the rule, so it applies only when some term produced by that grammar
appears in the corresponding position in the term. If the same
non-terminal with an identical subscript appears multiple times, the
rule only applies when the corresponding terms are structurally
identical (nonterminals without subscripts are not constrained to
match each other). Thus, the occurrence of $E_1$ on both the
left-hand and right-hand side of the rule above means that the context
of the application expression does not change when using this rule.
The ellipses are a form of Kleene star, meaning that zero or more
occurrences of terms matching the pattern proceeding the ellipsis may
appear in place of the the ellipsis and the pattern preceding it. We
use the notation $\{ x_1 \cdots \mapsto v_1 \cdots \} e_1$ for
capture-avoiding substitution; in this case it means that each
$x_1$ is replaced with the corresponding $v_1$ in
$e_1$. Finally, we write side-conditions in parentheses beside
a rule; the side-condition in the above rule indicates that the number
of $x_1$s must be the same as the number of $v_1$s.
Sometimes we use equality in the side-conditions; when we do it merely
means simple term equality, i.e., the two terms must have the
same syntactic shape.


\addtocounter{figure}{1} % get the figure counter in sync with the section counter
\subfigurestart{}
\beginfig
\input{r6-fig-grammar-parti.tex}
\caption{Grammar for programs and observables}\label{fig:grammar}
\endfig

Making the evaluation context $E$ explicit in the rule allows
us to define relations that manipulate their context. As a simple
example, we can add another rule that signals a violation when a
procedure is applied to the wrong number of arguments by discarding
the evaluation context on the right-hand side of a rule:
%
\begin{displaymath}
  \begin{array}{l}
    E[\texttt{((}\sy{lambda}~\texttt{(}x_1 \cdots\texttt{)}~e\texttt{)}~v_1~\cdots\texttt{)}] \rightarrow
    \\
    \textrm{\textbf{violation:} wrong argument count} ~~~~~ (\#x_1 \neq \#v_1)
  \end{array}
\end{displaymath}
%
Later we take advantage of the explicit evaluation context in more
sophisticated ways.



\section{Grammar}\label{sec:semantics:grammar}

\beginfig
\subfigureadjust{}
\input{r6-fig-grammar-partii.tex}
\caption{Grammar for evaluation contexts}\label{fig:ec-grammar}
\endfig
\subfigurestop{}

Figure~\ref{fig:grammar} shows the grammar for the subset of the
report this semantics models. Non-terminals are written in
\textit{italics} or in a calligraphic font ($\calP$
$\calA$, $\calR$, and $\calRv$) and literals are 
written in a \texttt{monospaced} font.

The $\calP$ non-terminal represents possible program states. The
first alternative is a program with a store and an expression. 
The second alternative is an uncaught exception, and the third is
used to indicate a place where the model does not completely specify
the behavior of the primitives it models (see section~\ref{sec:semantics:underspecification} for details of those situations). 
The $\calA$ non-terminal
represents a final result of a program. It is just like $\calP$
except that expression has been reduced to some sequence of values.

The $\calR$ and $\calRv$ non-terminals specify the observable results of a program. Each $\calR$ is either a sequence of values that correspond to the values produced by the program that terminates normally, or a tag indicating an uncaught exception was raised, or \sy{unknown} if the program encounters a situation the semantics does not cover. The $\calRv$ non-terminal specifies what the observable results are for a particular value: a pair, the empty list, a symbol, a self-quoting value (\schtrue, \schfalse, and numbers), a condition, or a procedure.

The \nt{sf} non-terminal generates individual elements of the
store. The store holds all of the mutable state of a program. It is
explained in more detail along with the rules that manipulate it.

Expressions ($\mathit{es}$) include quoted data, \sy{begin} expressions, \sy{begin0} expressions%
\footnote{ \sy{begin0} is not part of the standard, but we include it
  to make the rules for \va{dynamic-wind} and \va{letrec} easier to read. Although
  we model it directly, it can be defined in terms of other forms we
  model here that do come from the standard:
\begin{displaymath}
  \begin{array}{rcl}\tt
    \texttt{(}\sy{begin0}~e_1~e_2~\cdots\texttt{)} &=&
    \begin{array}{l}
      \texttt{(}\va{call\mbox{-}with\mbox{-}values}\\
      ~\texttt{(}\sy{lambda}~\texttt{()}~e_1\texttt{)}\\
      ~\texttt{(}\sy{lambda}~x\\
      ~~~e_2~\cdots\\
      ~~~\texttt{(}\va{apply}~\va{values}~x\texttt{)))}
    \end{array}
  \end{array}
\end{displaymath}
}, application expressions, \sy{if} expressions, \sy{set!}
expressions, variables, non-procedure values (\nt{nonproc}), primitive
procedures (\nt{pproc}), lambda expressions, \sy{letrec} and \sy{letrec*} expressions. 

The last few expression forms are only generated for intermediate states (\sy{dw} for \sy{dynamic-wind}, \sy{throw} for continuations, \sy{unspecified} for the result of the assignment operators, \sy{handlers} for exception handlers, and \sy{l!} and \sy{reinit} for \sy{letrec}), and should not appear in an initial program. Their use is described in the relevant sections of this appendix.

The \nt{f} non-terminal describes the formals for \sy{lambda} expressions. (The \sy{dot} is used instead of a period for procedures that accept an arbitrary number of arguments, in order to avoid meta-circular confusion in our PLT Redex model.) 

The \nt{s} non-terminal covers all datums, which can be either non-empty sequences (\nt{seq}), the empty sequence, self-quoting values (\nt{sqv}), or symbols. Non-empty sequences are either just a sequence of datums, or they are terminated with a dot followed by either a symbol or a self-quoting value. Finally the self-quoting values are numbers and the booleans \semtrue{} and \semfalse{}.

The \nt{p} non-terminal represents programs that have no quoted
data. Most of the reduction rules rewrite \nt{p} to \nt{p},
rather than $\calP$ to $\calP$, since quoted data is first
rewritten into calls to the list construction functions before
ordinary evaluation proceeds. In parallel to \nt{es}, \nt{e} represents
expressions that have no quoted expressions.

\beginfig
\begin{center}
\input{r6-fig-Quote.tex}

\input{r6-fig-QtocQtoic.tex}
\end{center}
\caption{Quote}\label{fig:quote}
\endfig

The values ($v$) are divided into four categories:
%
\begin{itemize}
\item Non-procedures (\nt{nonproc}) include pair pointers
  (\va{pp}), the empty list (\va{null}), symbols, self-quoting values
  (\nt{sqv}), and conditions. Conditions represent
  the report's condition values, but here just contain a message and
  are otherwise inert.
\item User procedures (\texttt{(}\sy{lambda} \nt{f} \nt{e} \nt{e} $\cdots$\texttt{)}) include multi-arity lambda expressions and lambda expressions with dotted parameter lists,
\item Primitive procedures (\nt{pproc}) include

\begin{itemize}
\item
 arithmetic procedures
  (\nt{aproc}): \va{+}, \va{-}, \va{/}, and \va{*}, 
\item 
  procedures of one
  argument (\nt{proc1}): \va{null?}, \va{pair?}, \va{car}, \va{cdr},
  \va{call/cc}, \va{procedure?}, \va{condition?}, \va{unspecified?}, \va{raise}, and \va{raise-continuable}, 
  \item
  procedures of
  two arguments (\nt{proc2}): \va{cons}, \va{set-car!}, \va{set-cdr!}, \va{eqv?},
  and \va{call-with-values}, 
  \item as well as \va{list}, \va{dynamic-wind},
  \va{apply}, \va{values}, and \va{with-exception-handler}.
\end{itemize}
\item Finally, continuations are represented as \sy{throw} expressions
  whose body consists of the context where the continuation was
  grabbed.
\end{itemize}
%
The next three set of non-terminals in figure~\ref{fig:grammar} represent pairs (\nt{pp}), which are divided into immutable pairs (\nt{ip}) and mutable pairs (\nt{mp}). The final set of non-terminals in figure~\ref{fig:grammar}, \nt{sym},
\nt{x}, and $n$ represent symbols, variables, and
numbers respectively. The non-terminals \nt{ip}, \nt{mp}, and \nt{sym} are all assumed to all be disjoint. Additionally, the variables $x$ are assumed not to include any keywords or primitive operations, so any program variables whose names coincide with them must be renamed before the semantics can give the meaning of that program.

\beginfig
\begin{center}
\input{r6-fig-Multiple--values--and--call-with-values.tex}
\end{center}
\caption{Multiple values and call-with-values}\label{fig:Multiple--values--and--call-with-values}
\endfig

The set of non-terminals for evaluation contexts is shown in
figure~\ref{fig:ec-grammar}. The \nt{P} non-terminal controls where
evaluation happens in a program that does not contain any quoted data.
The $E$ and $F$ evaluation contexts are for expressions.  They are factored in
that manner so that the \nt{PG}, \nt{G}, and \nt{H} evaluation contexts can
re-use \nt{F} and have fine-grained control over the context to support
exceptions and \va{dynamic-wind}. The starred and circled variants,
\Estar{}, \Eo{}, \Fstar{}, and \Fo{} dictate where a single value is
promoted to multiple values and where multiple values are demoted to a
single value. The \nt{U} context is used to manage the report's underspecification of the results of \sy{set!}, \va{set-car!}, and \va{set-cdr!} (see section~\ref{sec:semantics:underspecification} for details). Finally, the \nt{S} context is where quoted expressions can be simplified. The precise use of the evaluation contexts is explained along with the relevant rules.

To convert the answers ($\calA$)  of the semantics into observable results, we use these two functions:
\input{r6-fig-observable}
\input{r6-fig-observable-value}
They eliminate the store, and replace complex values with simple tags that indicate only the kind of value that was produced or, if no values were produced, indicates that either an uncaught exception was raised, or that the program reached a state that is not specified by the semantics.

\section{Quote}\label{sec:semantics:quote}

The first reduction rules that apply to any program is the 
rules in figure~\ref{fig:quote} that eliminate quoted expressions. 
The first two rules erase the quote for quoted expressions that do not introduce any pairs.
The last two rules lift quoted datums to the top of the expression so
they are evaluated only once, and turn the datums into calls to either \va{cons} or \va{consi}, via the metafunctions $\mathscr{Q}_i$ and $\mathscr{Q}_m$.

Note that the left-hand side of the \rulename{6qcons} and \rulename{6qconsi} rules are identical, meaning that if one rule applies to a term, so does the other rule. 
Accordingly, a quoted expression may be lifted out into a sequence of \va{cons} expressions, which create mutable pairs, or into a sequence of \va{consi} expressions, which create immutable pairs (see section~\ref{sec:semantics:lists} for the rules on how that happens).

These rules apply before any other because of the contexts in which they, and all of the other rules, apply. In particular, these rule applies in the
\nt{S} context. Figure~\ref{fig:ec-grammar} shows that the
\nt{S} context allows this reduction to apply in
any subexpression of an \nt{e}, as long as all of the
subexpressions to the left have no quoted expressions in them,
although expressions to the right may have quoted expressions.
Accordingly, this rule applies once for each quoted expression in the
program, moving out to the beginning of the program.
The rest of the rules apply in contexts that do not contain any quoted
expressions, ensuring that these rules convert all quoted data
into lists before those rules apply.

Although the identifier \nt{qp} does not have a subscript, the semantics of PLT Redex's ``fresh'' declaration takes special care to ensures that the \nt{qp} on the right-hand side of the rule is indeed the same as the one in the side-condition.

\beginfig
\begin{center}
\input{r6-fig-Exceptions}
\end{center}
\caption{Exceptions}\label{fig:Exceptions}
\endfig

\section{Multiple values}

The basic strategy for multiple values is to add a rule that demotes
$(\va{values}~v)$ to $v$ and another rule that promotes
$v$ to $(\va{values}~v)$. If we allowed these rules to apply
in an arbitrary evaluation context, however, we would get infinite
reduction sequences of endless alternation between promotion and
demotion. So, the semantics allows demotion only in a context
expecting a single value and allows promotion only in a context
expecting multiple values. We obtain this behavior with a small
extension to the Felleisen-Hieb framework (also present in the
operational model for R$^5$RS~\cite{mf:op-r5rs}).
We extend the notation so that
holes have names (written with a subscript), and the context-matching
syntax may also demand a hole of a particular name (also written with
a subscript, for instance $E[e]_{\star}$).  The extension
allows us to give different names to the holes in which multiple
values are expected and those in which single values are expected, and
structure the grammar of contexts accordingly.

To exploit this extension, we use three kinds of holes in the
evaluation context grammar in figure~\ref{fig:ec-grammar}. The
ordinary hole \hole{} appears where the usual kinds of
evaluation can occur. The hole \holes{} appears in contexts that
allow multiple values and \holeone{} appears in
contexts that expect a single value. Accordingly, the rule
\rulename{6promote} only applies in \holes{} contexts, and 
\rulename{6demote} only applies in \holeone{} contexts.

To see how the evaluation contexts are organized to ensure that
promotion and demotion occur in the right places, consider the \nt{F},
\Fstar{} and \Fo{} evaluation contexts. The \Fstar{} and \Fo{}
evaluation contexts are just the same as \nt{F}, except that they allow
promotion to multiple values and demotion to a single value,
respectively. So, the \nt{F} evaluation context, rather than being
defined in terms of itself, exploits \Fstar{} and \Fo{} to dictate
where promotion and demotion can occur. For example, \nt{F} can be
$\texttt{(}\sy{if}~\Fo{}~e~e\texttt{)}$ meaning that demotion from
$\texttt{(}\va{values}~v\texttt{)}$ to
$v$ can occur in the test of an \sy{if} expression.
Similarly, $F$ can be $\texttt{(}\sy{begin}~\Fstar{}~e~e~\cdots\texttt{)}$ meaning that
$v$ can be promoted to $\texttt{(}\va{values}~v\texttt{)}$ in the first subexpression of a \sy{begin}.

In general, the promotion and demotion rules simplify the definitions
of the other rules. For instance, the rule for \sy{if} does not
need to consider multiple values in its first subexpression.
Similarly, the rule for \sy{begin} does not need to consider the
case of a single value as its first subexpression.

\beginfig
\begin{center}
\input{r6-fig-Arithmetic.tex}
\input{r6-fig-Basic--syntactic--forms.tex}
\end{center}
\caption{Arithmetic and basic forms}\label{fig:Arithmetic}
\endfig

The other two rules in
figure~\ref{fig:Multiple--values--and--call-with-values} handle
\va{call-\hp{}with-\hp{}values}. The evaluation contexts for
\va{call-with-values} (in the $F$ non-terminal) allow
evaluation in the body of a procedure that has been passed as the first
argument to \va{call-with-values}, as long as the second argument
has been reduced to a value. Once evaluation inside that procedure
completes, it will produce multiple values (since it is an \Fstar{}
position), and the entire \va{call-with-values} expression reduces
to an application of its second argument to those values, via the rule
\rulename{6cwvd}. Finally, in the
case that the first argument to \va{call-with-values} is a value,
but is not of the form $\texttt{(}\sy{lambda}~\texttt{()}~e\texttt{)}$, the rule
\rulename{6cwvw} wraps it in a thunk to trigger evaluation.

\beginfig
\begin{center}
\input{r6-fig-Cons.tex}
\end{center}
\caption{Lists}\label{fig:Cons}
\endfig

\section{Exceptions}

The workhorses for the exception system are $$\texttt{(}\sy{handlers}~\nt{proc}~\cdots{}~\nt{e}\texttt{)}$$ expressions and the \nt{G} and \nt{PG} evaluation contexts (shown in figure~\ref{fig:ec-grammar}). 
The \sy{handlers} expression records the
active exception handlers (\nt{proc} $\cdots$) in some expression (\nt{e}). The
intention is that only the nearest enclosing \sy{handlers} expression
is relevant to raised exceptions, and the $G$ and \nt{PG} evaluation
contexts help achieve that goal. They are just like their counterparts
\nt{E} and \nt{P}, except that \sy{handlers} expressions cannot occur on the
path to the hole, and the exception system rules take advantage of
that context to find the closest enclosing handler.

To see how the contexts work together with \sy{handler}
expressions, consider the left-hand side of the \rulename{6xunee}
rule in figure~\ref{fig:Exceptions}.
It matches expressions that have a call to \va{raise} or
\va{raise-continuable} (the non-terminal \nt{raise*} matches
both exception-raising procedures) in a \nt{PG}
evaluation context. Since the \nt{PG} context does not contain any
\sy{handlers} expressions, this exception cannot be caught, so
this expression reduces to a final state indicating the uncaught
exception. The rule \rulename{6xuneh} also signals an uncaught
exception, but it covers the case where a \sy{handlers} expression
has exhausted all of the handlers available to it. The rule applies to
expressions that have a \sy{handlers} expression (with no
exception handlers) in an arbitrary evaluation context where a call to
one of the exception-raising functions is nested in the
\sy{handlers} expression. The use of the \nt{G} evaluation
context ensures that there are no other \sy{handler} expressions
between this one and the raise.

The next two rules cover call to the procedure \va{with-exception-handler}.
The \rulename{6xwh1} rule applies when there are no \sy{handler}
expressions. It constructs a new one and applies $\nt{v}_2$ as a
thunk in the \sy{handler} body. If there already is a handler
expression, the \rulename{6xwhn} applies. It collects the current
handlers and adds the new one into a new \sy{handlers} expression
and, as with the previous rule, invokes the second argument to
\va{with-exception-handlers}.

The next two rules cover exceptions that are raised in the context of
a \sy{handlers} expression. If a continuable exception is raised,
\rulename{6xrc} applies. It takes the most recently installed
handler from the nearest enclosing \sy{handlers} expression and
applies it to the argument to \va{raise-continuable}, but in a
context where the exception handlers do not include that latest
handler. The \rulename{6xr} rule behaves similarly, except it
raises a new exception if the handler returns. The new exception is
created with the \sy{make-cond} special form.

\beginfig
\begin{center}
\input{r6-fig-Eqv.tex}
\end{center}
\caption{Eqv}\label{fig:Eqv}
\endfig

The \sy{make-cond} special form is a stand-in for the report's
conditions. It does not evaluate its argument (note its absence from
the $E$ grammar in figure~\ref{fig:ec-grammar}). That argument
is just a literal string describing the context in which the exception
was raised. The only operation on conditions is \va{condition?},
whose semantics are given by the two rules \rulename{6ct} and
\rulename{6cf}.

Finally, the rule \rulename{6xdone} drops a \sy{handlers} expression
when its body is fully evaluated, and the rule \rulename{6weherr}
raises an exception when \va{with-exception-handler} is supplied with
incorrect arguments.

\section{Arithmetic and basic forms}

This model does not include the report's arithmetic, but does include
an idealized form in order to make experimentation with other features
and writing test suites for the model simpler.
Figure~\ref{fig:Arithmetic} shows the reduction rules for the
primitive procedures that implement addition, subtraction,
multiplication, and division. They defer to their mathematical
analogues. In addition, when the subtraction or divison operator are
applied to no arguments, or when division receives a zero as a
divisor, or when any of the arithmetic operations receive a
non-number, an exception is raised.

The bottom half of figure~\ref{fig:Arithmetic} shows the rules for
\sy{if}, \sy{begin}, and \sy{begin0}. The relevant
evaluation contexts are given by the $F$ non-terminal.

The evaluation contexts for \sy{if} only allow evaluation in its
test expression. Once that is a value, the rules reduce
an \sy{if} expression to its consequent if the test is not
\semfalse{}, and to its alternative if it is \semfalse{}.

The \sy{begin} evaluation contexts allow evaluation in the first
subexpression of a begin, but only if there are two or more
subexpressions. In that case, once the first expression has been fully
simplified, the reduction rules drop its value. If there is only a
single subexpression, the \sy{begin} itself is dropped.

\subfigurestart{}
\beginfig
\begin{center}
\input{r6-fig-Procedure--application.tex}
\end{center}
\caption{Procedures \& application}\label{fig:Procedure--application}
\endfig

Like the \sy{begin} evaluation contexts, the \sy{begin0}
evaluation contexts allow evaluation of the first subexpression of a
\sy{begin0} expression when there are two or more subexpressions.
The \sy{begin0} evaluation contexts also allow evaluation in the
second subexpression of a \sy{begin0} expression, as long as the first
subexpression has been fully simplified. The \rulename{6begin0n} rule for
\sy{begin0} then drops a fully simplified second subexpression.
Eventually, there is only a single expression in the \sy{begin0},
at which point the \rulename{begin01} rule fires, and removes the
\sy{begin0} expression.

\section{Lists}\label{sec:semantics:lists}

The rules in figure~\ref{fig:Cons} handle lists. The first two rules handle \va{list} by reducing it to a succession of calls to \va{cons}, followed by \va{null}.

The next two rules, \rulename{6cons} and \rulename{6consi}, allocate new \va{cons} cells.
They both move $\texttt{(}\va{cons}~v_1~v_2\texttt{)}$ into the store, bound to a fresh
pair pointer (see also section~\ref{sec:semantics:quote} for a description of ``fresh''). 
The \rulename{6cons} uses a \nt{mp} variable, to indicate the pair is mutable, and the \rulename{6consi} uses a \nt{ip} variable to indicate the pair is immutable.

The rules \rulename{6car} and \rulename{6cdr} extract the components of a pair from the store when presented with a pair pointer (the \nt{pp} can be either \nt{mp} or \nt{ip}, as shown in figure~\ref{fig:grammar}).

The rules \rulename{6setcar} and \rulename{6setcdr} handle assignment of mutable pairs. 
They replace the contents of the appropriate location in the store with the new value, and reduce to \va{unspecified}. See section~\ref{sec:semantics:underspecification} for an explanation of how \va{unspecified} reduces.

\beginfig
\subfigureadjust{}
\begin{center}
\input{r6-fig-Var-set!d_.tex}
\end{center}
\caption{Variable-assignment relation}\label{fig:varsetd}
\endfig

The next four rules handle the \va{null?} predicate and the \va{pair?} predicate, and the final four rules raise exceptions when \va{car}, \va{cdr}, \va{set-car!} or \va{set-cdr!} receive non pairs.

\section{Eqv}

The rules for \va{eqv?} are shown in figure~\ref{fig:Eqv}. The first two rules cover most of the behavior of \va{eqv?}. 
The first says that when the two arguments to \va{eqv?} are syntactically identical, then \va{eqv?} produces \semtrue{} and the second says that when the arguments are not syntactically identical, then \va{eqv?} produces \semfalse{}. 
The structure of \nt{v} has been carefully designed so that simple term equality corresponds closely to \va{eqv?}'s behavior. 
For example, pairs are represented as pointers into the store and \va{eqv?} only compares those pointers.

The side-conditions on those first two rules ensure that they do not apply when simple term equality does not match the behavior of \va{eqv?}. There are two situations where it does not match: comparing two conditions and comparing two procedures. For the first, the report does not specify \va{eqv?}'s behavior, except to say that it must return a boolean, so the remaining two rules (\rulename{6eqct}, and \rulename{6eqcf}) allow such comparisons to return \semtrue{} or \semfalse{}. Comparing two procedures is covered in section~\ref{sec:semantics:underspecification}. 

\section{Procedures and application}

In evaluating a procedure call, the report leaves
unspecified the order in which arguments are evaluated. So, our reduction system allows multiple, different reductions to occur, one for each possible order of evaluation.

To capture unspecified evaluation order but allow only evaluation that
is consistent with some sequential ordering of the evaluation of an
application's subexpressions, we use non-deterministic choice to first pick
a subexpression to reduce only when we have not already committed to
reducing some other subexpression. To achieve that effect, we limit
the evaluation of application expressions to only those that have a
single expression that is not fully reduced, as shown in the
non-terminal $F$, in figure~\ref{fig:ec-grammar}. To evaluate
application expressions that have more than two arguments to evaluate,
the rule \rulename{6mark} picks one of the subexpressions of an
application that is not fully simplified and lifts it out in its own
application, allowing it to be evaluated. Once one of the lifted
expressions is evaluated, the \rulename{6appN} substitutes its value
back into the original application.

The \rulename{6appN} rule also handles other applications whose
arguments are finished by substituting the first argument for
the first formal parameter in the expression. Its side-condition uses
the relation in figure~\ref{fig:varsetd} to ensure that there are no
\sy{set!} expressions with the parameter $x_1$ as a target.
If there is such an assignment, the \rulename{6appN!} rule applies (see also section~\ref{sec:semantics:quote} for a description of ``fresh'').
Instead of directly substituting the actual parameter for the formal
parameter, it creates a new location in the store, initially bound the
actual parameter, and substitutes a variable standing for that
location in place of the formal parameter. The store, then, handles
any eventual assignment to the parameter. Once all of the parameters
have been substituted away, the rule \rulename{6app0} applies and
evaluation of the body of the procedure begins.

At first glance, the rule \rulename{6appN} appears superfluous, since it seems like the rules could just reduce first by \rulename{6appN!} and then look up the variable when it is evaluated. 
There are two reasons why we keep the \rulename{6appN}, however. 
The first is purely conventional: reducing applications via substitution is taught to us at an early age and is commonly used in rewriting systems in the literature.
The second reason is more technical:  the
\rulename{6mark} rule requires that \rulename{6appN} be applied once $\nt{e}_i$ has been reduced to a value. \rulename{6appN!} would
lift the value into the store and put a variable reference into the application, leading to another use of \rulename{6mark}, and another use of \rulename{6appN!}, which continues forever.

The rule \rulename{6$\mu$app} handles a well-formed application of a function with a dotted parameter lists. 
It such an application into an application of an
ordinary procedure by constructing a list of the extra arguments. Similarly, the rule \rulename{6$\mu$app1} handles an application of a procedure that has a single variable as its parameter list.

The rule \rulename{6var} handles variable lookup in the store and \rulename{6set} handles variable assignment.

The next two rules \rulename{6proct} and \rulename{6procf} handle applications of \va{procedure?}, and the remaining rules cover applications of non-procedures and arity violations.

\beginfig
\subfigureadjust{}
\begin{center}
\input{r6-fig-Apply.tex}
\input{r6-fig-circular_.tex}
\end{center}
\caption{Apply}\label{fig:Apply}
\endfig
\subfigurestop{}

The rules in figure~\ref{fig:Apply} 
cover \va{apply}. 
The first rule, \rulename{6applyf}, covers the case where the last argument to
\va{apply} is the empty list, and simply reduces by erasing the
empty list and the \va{apply}. The second rule, \rulename{6applyc}
covers a well-formed application of \va{apply} where \va{apply}'s final argument is a pair. It
reduces by extracting the components of the pair from the store and
putting them into the application of \va{apply}. Repeated
application of this rule thus extracts all of the list elements passed
to \va{apply} out of the store. 

The remaining five rules cover the
various violations that can occur when using \va{apply}. The first one covers the case where \va{apply} is supplied with a cyclic list. The next four cover applying a
non-procedure, passing a non-list as the last argument, and supplying
too few arguments to \va{apply}.

\section{Call/cc and dynamic wind}

\beginfig
\begin{center}
\input{r6-fig-Call-cc--and--dynamic-wind.tex} \\
\input{r6-fig-TrimpRepoSt.tex}
\end{center}
\caption{Call/cc and dynamic wind}\label{fig:Call-cc--and--dynamic-wind}
\endfig

The specification of \va{dynamic-wind} uses 
$\texttt{(}\sy{dw}~x~e~e~e\texttt{)}$
expressions to record which dynamic-wind \var{thunk}s are active at
each point in the computation. Its first argument is an identifier
that is globally unique and serves to identify invocations of
\va{dynamic-wind}, in order to avoid exiting and re-entering the
same dynamic context during a continuation switch. The second, third,
and fourth arguments are calls to some \var{before}, \var{thunk}, and
\var{after} procedures from a call to \va{dynamic-wind}. Evaluation only
occurs in the middle expression; the \sy{dw} expression only
serves to record which \var{before} and \var{after} procedures need to be run during a
continuation switch. Accordingly, the reduction rule for an
application of \va{dynamic-wind} reduces to a call to the
\var{before} procedure, a \sy{dw} expression and a call to the
\var{after} procedure, as
shown in rule \rulename{6wind} in
figure~\ref{fig:Call-cc--and--dynamic-wind}. The next two rules cover
abuses of the \va{dynamic-wind} procedure: calling it with
non-procedures, and calling it with the wrong number of arguments. The
\rulename{6dwdone} rule erases a \sy{dw} expression when its second
argument has finished evaluating.

The next two rules cover \va{call/cc}. The rule
\rulename{6call/cc} creates a new continuation. It takes the context
of the \va{call/cc} expression and packages it up into a
\sy{throw} expression that represents the continuation. The
\sy{throw} expression uses the fresh variable $x$ to record
where the application of \va{call/cc} occurred in the context for
use in the \rulename{6throw} rule when the continuation is applied.
That rule takes the arguments of the continuation, wraps them with a
call to \va{values}, and puts them back into the place where the
original call to \va{call/cc} occurred, replacing the current
context with the context returned by the $\mathscr{T}$ metafunction.

The $\mathscr{T}$ (for ``trim'') metafunction accepts two $D$ contexts and
builds a context that matches its second argument, the destination
context, except that additional calls to the \var{before} and
\var{after} procedures
from \sy{dw} expressions in the context have been added.

The first clause of the $\mathscr{T}$ metafunction exploits the
$H$ context, a context that contains everything except
\sy{dw} expressions. It ensures that shared parts of the
\va{dynamic-wind} context are ignored, recurring deeper into the
two expression contexts as long as the first \sy{dw} expression in
each have matching identifiers ($x_1$). The final rule is a
catchall; it only applies when all the others fail and thus applies
either when there are no \sy{dw}s in the context, or when the
\sy{dw} expressions do not match. It calls the two other
metafunctions defined in figure~\ref{fig:Call-cc--and--dynamic-wind} and
puts their results together into a \sy{begin} expression.

The $\mathscr{R}$ metafunction extracts all of the \var{before}
procedures from its argument and the $\mathscr{S}$ metafunction extracts all of the \var{after} procedures from its argument. They each construct new contexts and exploit
$H$ to work through their arguments, one \sy{dw} at a time.
In each case, the metafunctions are careful to keep the right
\sy{dw} context around each of the procedures in case a continuation
jump occurs during one of their evaluations. 
Since $\mathscr{R}$,
receives the destination context, it keeps the intermediate
parts of the context in its result.
In contrast
$\mathscr{S}$ discards all of the context except the \sy{dw}s,
since that was the context where the call to the
continuation occurred.

\section{Letrec}

\beginfig
\begin{center}
\input{r6-fig-Letrec.tex}
\end{center}
\caption{Letrec and letrec*}
\label{fig:Letrec}
\endfig

Figre~\ref{fig:Letrec} shows the rules that handle \sy{letrec} and \sy{letrec*} and the supplementary expressions that they produce, \sy{l!} and \sy{reinit}. As a first approximation, both \va{letrec} and \va{letrec*} reduce by allocating locations in the store to hold the values of the init expressions, initializing those locations to \sy{bh} (for ``black hole''), evaluating the init expressions, and then using \va{l!} to update the locations in the store with the value of the init expressions. They also use \va{reinit} to detect when an init expression in a letrec is reentered via a continuation.

Before considering how \sy{letrec} and \sy{letrec*} use \sy{l!} and \sy{reinit}, first consider how \sy{l!} and \sy{reinit} behave. The first two rules in figure~\ref{fig:Letrec} cover \sy{l!}. It behaves very much like \sy{set!}, but it initializes both ordinary variables, and variables that are current bound to the black hole (\sy{bh}).

The next two rules cover ordinary \sy{set!} when applied to a variable
that is currently bound to a black hole. This situation can arise when
the program assigns to a variable before letrec initializes it, eg
\verb|(letrec ((x (set! x 5))) x)|. The report specifies that either
an implementation should perform the assignment, as reflected in the
\rulename{6setdt} rule or it raise an exception, as reflected in the \rulename{6setdte} rule.

The \rulename{6dt} rule covers the case where a variable is referred
to before the value of a init expression is filled in, which must
always raise an exception.

A \va{reinit} expression is used to detect a program that captures a continuation in an initialization expression and returns to it, as shown in the three rules \rulename{6init}, \rulename{6reinit}, and \rulename{6reinite}. The \va{reinit} form accepts an identifier that is bound in the store to a boolean as its argument. Those are identifiers are initially \semfalse{}. When \va{reinit} is evaluated, it checks the value of the variable and, if it is still \semfalse{}, it changes it to \semtrue{}. If it is already \semtrue{}, then \va{reinit} either just does nothing, or it raises an exception, in keeping with the two legal behaviors of \va{letrec} and \va{letrec*}. 

The last two rules in figure~\ref{fig:Letrec} put together \sy{l!} and \sy{reinit}. The \rulename{6letrec} rule reduces a \sy{letrec} expression to an application expression, in order to capture the unspecified order of evaluation of the init expressions. Each init expression is wrapped in a \sy{begin0} that records the value of the init and then uses \sy{reinit} to detect continuations that return to the init expression. Once all of the init expressions have been evaluated, the procedure on the right-hand side of the rule is invoked, causing the value of the init expression to be filled in the store, and evaluation continues with the body of the original \sy{letrec} expression.

The \rulename{6letrec*} rule behaves similarly, but uses a \sy{begin} expression rather than an application, since the init expressions are evaluated from left to right. Moreover, each init expression is filled into the store as it is evaluated, so that subsequent init expressions can refer to its value.

\section{Underspecification}\label{sec:semantics:underspecification}

\beginfig
\begin{center}
\input{r6-fig-Underspecification.tex}
\end{center}
\caption{Explicitly unspecified behavior}\label{fig:Underspecification}
\endfig

The rules in figure~\ref{fig:Underspecification} cover aspects of the
semantics that are explicitly unspecified. Implementations can replace
the rules \rulename{6ueqv}, \rulename{6uval} and with different rules that cover the left-hand sides and, as long as they follow the informal specification, any replacement is valid. Those three situations correspond to the case when \va{eqv?} applied to two procedures and when multiple values are used in a single-value context.

The remaining rules in figure~\ref{fig:Underspecification} cover the results from the assignment operations, \sy{set!}, \va{set-car!}, and \va{set-cdr!}. An implementation does not adjust those rules, but instead renders them useless by adjusting the rules that insert \va{unspecified}: \rulename{6setcar}, \rulename{6setcdr}, \rulename{6set}, and \rulename{6setd}. Those rules can be adjusted by replacing \va{unspecified} with any number of values in those rules.

So, the remaining rules just specify the minimal behavior that we know that a value or values must have and otherwise reduce to an \textbf{unknown:} state. The rule \rulename{6udemand} drops \va{unspecified} in the \sy{U} context. See figure~\ref{fig:ec-grammar} for the precise definition of \sy{U}, but intuitively it is a context that is only a single expression layer deep that contains expressions whose value depends on the value of their subexpressions, like the first subexpression of a \sy{if}. Following that are rules that discard \va{unspecified} in expressions that discard the results of some of their subexpressions. The \rulename{6ubegin} shows how \sy{begin} discards its first expression when there are more expressions to evaluate. The next two rules, \rulename{6uhandlers} and \rulename{6udw} propagate \va{unspecified} to their context, since they also return any number of values to their context. Finally, the two \va{begin0} rules preserve \va{unspecified} until the rule \rulename{6begin01} can return it to its context.

%\section*{Acknowledgments}
%Thanks to Michael Sperber for many helpful discussions of specific points in the semantics, for spotting many mistakes and places where the formal semantics diverged from the informal semantics, and for generally making it possible for us to keep up with changes to the informal semantics as it developed. Thanks also to Will Clinger for a careful reading of the semantics and its explanation.

%%% Local Variables: 
%%% mode: latex
%%% TeX-master: "r6rs"
%%% End: 
 \par
\chapter{Sample definitions for derived forms}
\label{derivedformsappendix}

This appendix contains sample definitions for some of the keywords
described in this report in terms of simpler forms:

\subsubsection*{{\tt cond}}
The {\cf cond} keyword (section~\ref{cond}) 
could be defined in terms of {\cf if}, {\cf let} and {\cf
  begin} using {\cf syntax-rules} as follows:

\begin{scheme}
(define-syntax \ide{cond}
  (syntax-rules (else =>)
    ((cond (else result1 result2 ...))
     (begin result1 result2 ...))
    ((cond (test => result))
     (let ((temp test))
       (if temp (result temp))))
    ((cond (test => result) clause1 clause2 ...)
     (let ((temp test))
       (if temp
           (result temp)
           (cond clause1 clause2 ...))))
    ((cond (test)) test)
    ((cond (test) clause1 clause2 ...)
     (let ((temp test))
       (if temp
           temp
           (cond clause1 clause2 ...))))
    ((cond (test result1 result2 ...))
     (if test (begin result1 result2 ...)))
    ((cond (test result1 result2 ...)
           clause1 clause2 ...)
     (if test
         (begin result1 result2 ...)
         (cond clause1 clause2 ...)))))%
\end{scheme}
\subsubsection*{{\tt case}}
The {\cf case} keyword (section~\ref{case}) could be defined in terms of {\cf let}, {\cf cond}, and
{\cf memv} (see library chapter~\extref{lib:listutilities}{List utilities}) using {\cf syntax-rules} as follows:

\begin{scheme}
(define-syntax \ide{case}
  (syntax-rules (else)
    ((case expr0
       ((key ...) res1 res2 ...)
       ...
       (else else-res1 else-res2 ...))
     (let ((tmp expr0))
       (cond
         ((memv tmp '(key ...)) res1 res2 ...)
         ...
         (else else-res1 else-res2 ...))))
    ((case expr0
       ((keya ...) res1a res2a ...)
       ((keyb ...) res1b res2b ...)
       ...)
     (let ((tmp expr0))
       (cond
         ((memv tmp '(keya ...)) res1a res2a ...)
         ((memv tmp '(keyb ...)) res1b res2b ...)
         ...)))))%
\end{scheme}

\subsubsection*{{\tt let*}}

The {\cf let*} keyword (section~\ref{let*})
could be defined in terms of {\cf let}
using {\cf syntax-rules} as follows:

\begin{scheme}
(define-syntax \ide{let*}
  (syntax-rules ()
    ((let* () body1 body2 ...)
     (let () body1 body2 ...))
    ((let* ((name1 expr1) (name2 expr2) ...)
       body1 body2 ...)
     (let ((name1 expr1))
       (let* ((name2 expr2) ...)
         body1 body2 ...)))))%
\end{scheme}

\subsubsection*{{\tt letrec}}
The {\cf letrec} keyword (section~\ref{letrec})
could be defined approximately in terms of {\cf let}
and {\cf set!} using {\cf syntax-rules}, using a helper
to generate the temporary variables
needed to hold the values before the assignments are made,
as follows:

\begin{scheme}
(define-syntax \ide{letrec}
  (syntax-rules ()
    ((letrec () body1 body2 ...)
     (let () body1 body2 ...))
    ((letrec ((var init) ...) body1 body2 ...)
     (letrec-helper
       (var ...)
       ()
       ((var init) ...)
       body1 body2 ...))))

(define-syntax letrec-helper
  (syntax-rules ()
    ((letrec-helper
       ()
       (temp ...)
       ((var init) ...)
       body1 body2 ...)
     (let ((var <undefined>) ...)
       (let ((temp init) ...)
         (set! var temp)
         ...)
       (let () body1 body2 ...)))
    ((letrec-helper
       (x y ...)
       (temp ...)
       ((var init) ...)
       body1 body2 ...)
     (letrec-helper
       (y ...)
       (newtemp temp ...)
       ((var init) ...)
       body1 body2 ...))))%
\end{scheme}

The syntax {\cf <undefined>} represents an expression that
returns something that, when stored in a location, causes an exception
with condition type {\cf\&assertion} to
be raised if an attempt to read from or write to the location occurs before the
assignments generated by the {\cf letrec} transformation take place.
(No such expression is defined in Scheme.)

A simpler definition using {\cf syntax-case} and {\cf
generate-\hp{}temporaries} is given in library
chapter~\extref{lib:syntaxcasechapter}{{\cf syntax-case}}.

\subsubsection*{{\tt letrec*}}

The {\cf letrec*} keyword could be defined approximately in terms of
{\cf let} and {\cf set!}  using {\cf syntax-rules} as follows:

\begin{scheme}
(define-syntax \ide{letrec*}
  (syntax-rules ()
    ((letrec* ((var1 init1) ...) body1 body2 ...)
     (let ((var1 <undefined>) ...)
       (set! var1 init1)
       ...
       (let () body1 body2 ...)))))%
\end{scheme}

The syntax {\cf <undefined>} is as in the definition of {\cf letrec} above.

\subsubsection*{{\tt let-values}}
The following definition of {\cf let-values} (section~\ref{let-values})
using {\cf syntax-rules}
employs a pair of helpers to
create temporary names for the formals.

\begin{scheme}
(define-syntax let-values
  (syntax-rules ()
    ((let-values (binding ...) body1 body2 ...)
     (let-values-helper1
       ()
       (binding ...)
       body1 body2 ...))))

(define-syntax let-values-helper1
  ;; map over the bindings
  (syntax-rules ()
    ((let-values
       ((id temp) ...)
       ()
       body1 body2 ...)
     (let ((id temp) ...) body1 body2 ...))
    ((let-values
       assocs
       ((formals1 expr1) (formals2 expr2) ...)
       body1 body2 ...)
     (let-values-helper2
       formals1
       ()
       expr1
       assocs
       ((formals2 expr2) ...)
       body1 body2 ...))))

(define-syntax let-values-helper2
  ;; create temporaries for the formals
  (syntax-rules ()
    ((let-values-helper2
       ()
       temp-formals
       expr1
       assocs
       bindings
       body1 body2 ...)
     (call-with-values
       (lambda () expr1)
       (lambda temp-formals
         (let-values-helper1
           assocs
           bindings
           body1 body2 ...))))
    ((let-values-helper2
       (first . rest)
       (temp ...)
       expr1
       (assoc ...)
       bindings
       body1 body2 ...)
     (let-values-helper2
       rest
       (temp ... newtemp)
       expr1
       (assoc ... (first newtemp))
       bindings
       body1 body2 ...))
    ((let-values-helper2
       rest-formal
       (temp ...)
       expr1
       (assoc ...)
       bindings
       body1 body2 ...)
     (call-with-values
       (lambda () expr1)
       (lambda (temp ... . newtemp)
         (let-values-helper1
           (assoc ... (rest-formal newtemp))
           bindings
           body1 body2 ...))))))%
\end{scheme}

\subsubsection*{{\tt let*-values}}

The following macro defines {\cf let*-values} in terms of {\cf let}
and {\cf let-values} using {\cf syntax-rules}:

\begin{scheme}
(define-syntax let*-values
  (syntax-rules ()
    ((let*-values () body1 body2 ...)
     (let () body1 body2 ...))
    ((let*-values (binding1 binding2 ...)
       body1 body2 ...)
     (let-values (binding1)
       (let*-values (binding2 ...)
         body1 body2 ...)))))%
\end{scheme}


\subsubsection*{{\tt let}}

The {\cf let} keyword could be defined in terms of {\cf lambda} and {\cf letrec}
using {\cf syntax-rules} as
follows:

\begin{scheme}
(define-syntax \ide{let}
  (syntax-rules ()
    ((let ((name val) ...) body1 body2 ...)
     ((lambda (name ...) body1 body2 ...)
      val ...))
    ((let tag ((name val) ...) body1 body2 ...)
     ((letrec ((tag (lambda (name ...)
                      body1 body2 ...)))
        tag)
      val ...))))%
\end{scheme}


%%% Local Variables: 
%%% mode: latex
%%% TeX-master: "r6rs"
%%% End: 
 \par
\chapter{Additional material}
\label{additionalmaterialappendix}

This report itself, as well as more material related to this report
such as reference implementations of some parts of Scheme and archives of
mailing lists discussing this report is at
\begin{center}
\url{http://www.r6rs.org/}
\end{center}

The Schemers web site at
\begin{center}
\url{http://www.schemers.org/}
\end{center}
as well as the Readscheme site at
\begin{center}
\url{http://library.readscheme.org/}
\end{center}
contain extensive Scheme bibliographies, as well as papers,
programs, implementations, and other material related to Scheme.

%%% Local Variables: 
%%% mode: latex
%%% TeX-master: "r6rs"
%%% End: 
 \par
\input{example} \par
\chapter{Language changes}
\label{languagechangesappendix}

This chapter describes most of the changes that have been made to
Scheme since the ``Revised$^5$ Report''~\cite{R5RS} was published:

\begin{itemize}
\item Scheme source code now uses the Unicode character set.
  Specifically, the character set that can be used for identifiers has
  been greatly expanded.
\item Identifiers can now start with the characters {\cf ->}.
\item Identifiers and symbol literals are now case-sensitive.
\item Identifiers and representations of characters, booleans,
  number objects, and {\cf .} must be explicitly delimited.
\item {\cf \sharpsign} is now a delimiter.
\item Bytevector literal syntax has been added.
\item Matched square brackets can be used synonymously with parentheses.
\item The read-syntax abbreviations {\cf \sharpsign{}'} (for {\cf
    syntax}), {\cf \sharpsign\backquote} (for {\cf quasisyntax}), {\cf
    \sharpsign{},} (for {\cf unsyntax}), and {\cf \sharpsign{},@}
  (for {\cf unsyntax-splicing} have been added; see section~\ref{abbreviationsection}.)
\item {\cf \sharpsign} can no longer be used in place of digits in number
  representations.
\item The external representation of number objects can now include a
  mantissa width.
\item Literals for NaNs and infinities were added.
\item String and character literals can now use a variety of escape
  sequences.
\item Block and datum comments have been added.
\item The {\cf \sharpsign{}!r6rs} comment for marking report-compliant
  lexical syntax has been added.
\item Characters are now specified to correspond to Unicode scalar
  values.
\item Many of the procedures and syntactic forms of the language are
  now part of the \rsixlibrary{base} library.  Some procedures and
  syntactic forms have been moved to other libraries; see figure~\ref{r5rsmovedfigure}.

  \begin{figure*}[tb]
    \centering
    \small
    \begin{tabular}[t]{ll}
      identifier & moved to \\\hline
      {\cf assoc} & \rsixlibrary{lists} \\
      {\cf assv} & \rsixlibrary{lists} \\
      {\cf assq} & \rsixlibrary{lists} \\
      {\cf call-with-input-file} & \rsixlibrary{io simple} \\
      {\cf call-with-output-file} & \rsixlibrary{io simple} \\
      {\cf char-upcase} & \rsixlibrary{unicode} \\
      {\cf char-downcase} & \rsixlibrary{unicode} \\
      {\cf char-ci=?} & \rsixlibrary{unicode} \\
      {\cf char-ci<?} & \rsixlibrary{unicode} \\
      {\cf char-ci>?} & \rsixlibrary{unicode} \\
      {\cf char-ci<=?} & \rsixlibrary{unicode} \\
      {\cf char-ci>=?} & \rsixlibrary{unicode} \\
      {\cf char-alphabetic?} & \rsixlibrary{unicode} \\
      {\cf char-numeric?} & \rsixlibrary{unicode} \\
      {\cf char-whitespace?} & \rsixlibrary{unicode} \\
      {\cf char-upper-case?} & \rsixlibrary{unicode} \\
      {\cf char-lower-case?} & \rsixlibrary{unicode} \\
      {\cf close-input-port} & \rsixlibrary{io simple} \\
      {\cf close-output-port} & \rsixlibrary{io simple} \\
      {\cf current-input-port} & \rsixlibrary{io simple} \\
      {\cf current-output-port} & \rsixlibrary{io simple} \\
      {\cf display} & \rsixlibrary{io simple} \\
      {\cf do} & \rsixlibrary{control} \\
      {\cf eof-object?} & \rsixlibrary{io simple} \\
      {\cf eval} & \rsixlibrary{eval} \\
      {\cf delay} & \rsixlibrary{r5rs}\\
      {\cf exact->inexact} & \rsixlibrary{r5rs}\\
      {\cf force} & \rsixlibrary{r5rs}
\htmlonly \\ \endhtmlonly
\texonly
    \end{tabular}
    \qquad
    \begin{tabular}[t]{ll}
      identifier & moved to \\\hline
\endtexonly
      {\cf inexact->exact} & \rsixlibrary{r5rs}\\
      {\cf member} & \rsixlibrary{lists} \\
      {\cf memv} & \rsixlibrary{lists} \\
      {\cf memq} & \rsixlibrary{lists} \\
      {\cf modulo} & \rsixlibrary{r5rs} \\
      {\cf newline} & \rsixlibrary{io simple} \\
      {\cf null-environment} & \rsixlibrary{r5rs} \\
      {\cf open-input-file} & \rsixlibrary{io simple} \\
      {\cf open-output-file} & \rsixlibrary{io simple} \\
      {\cf peek-char} & \rsixlibrary{io simple} \\
      {\cf quotient} & \rsixlibrary{r5rs} \\
      {\cf read} & \rsixlibrary{io simple} \\
      {\cf read-char} & \rsixlibrary{io simple} \\
      {\cf remainder} & \rsixlibrary{r5rs} \\
      {\cf scheme-report-environment} & \rsixlibrary{r5rs} \\
      {\cf set-car!} & \rsixlibrary{mutable-pairs} \\
      {\cf set-cdr!} & \rsixlibrary{mutable-pairs} \\
      {\cf string-ci=?} & \rsixlibrary{unicode} \\
      {\cf string-ci<?} & \rsixlibrary{unicode} \\
      {\cf string-ci>?} & \rsixlibrary{unicode} \\
      {\cf string-ci<=?} & \rsixlibrary{unicode} \\
      {\cf string-ci>=?} & \rsixlibrary{unicode} \\
      {\cf string-set!} & \rsixlibrary{mutable-strings} \\
      {\cf string-fill!} & \rsixlibrary{mutable-strings} \\
      {\cf with-input-from-file} & \rsixlibrary{io simple} \\
      {\cf with-output-to-file} & \rsixlibrary{io simple} \\
      {\cf write} & \rsixlibrary{io simple} \\
      {\cf write-char} & \rsixlibrary{io simple}
    \end{tabular}
    \caption{Identifiers moved to libraries}
    \label{r5rsmovedfigure}
  \end{figure*}
\item The base language has the following new procedures and syntactic
  forms: {\cf letrec*}, {\cf let-values}, {\cf let*-\hp{}values}, {\cf
    real-valued?}, {\cf rational-valued?}, {\cf integer-valued?}, {\cf
    exact}, {\cf inexact}, {\cf finite?}, {\cf infinite?}, {\cf nan?},
  {\cf div}, {\cf mod}, {\cf
    div-and-mod}, {\cf div0}, {\cf mod0}, {\cf div0-and-mod0}, {\cf
    exact-integer-sqrt}, {\cf boolean=?}, {\cf symbol=?}, {\cf
    string-for-each}, {\cf vector-map}, {\cf vector-\hp{}for-\hp{}each}, {\cf
    error}, {\cf assertion-violation}, {\cf assert}, {\cf call/cc},
  {\cf identifier-syntax}.
\item The following procedures have been removed: {\cf
    char-\hp{}ready?}, {\cf transcript-on}, {\cf transcript-off},
  {\cf load}.
\item The case-insensitive string comparisons ({\cf string-\hp{}ci=?}, {\cf
    string-\hp{}ci<?}, {\cf string-ci>?}, {\cf string-ci<=?}, {\cf
    string-ci>=?}) operate on the case-folded versions of the strings
  rather than as the simple lexicographic ordering induced by the
  corresponding character comparison procedures.
\item Libraries have been added to the language.
\item A number of standard libraries are described in a separate
  report~\cite{R6RS-libraries}.
\item Many situations that ``were an error'' now have defined or
  constrained behavior.  In particular, many are now specified in
  terms of the exception system.
\item The full numerical tower is now required.
\item The semantics for the transcendental functions has been
  specified more fully.
\item The semantics of {\cf expt} for zero bases has been refined.
\item In {\cf syntax-rules} forms, a {\cf\_} may be used in place of
  the keyword.
\item The {\cf let-syntax} and {\cf letrec-syntax} no longer introduce a
  new environment for their bodies.
\item For implementations that support NaNs or infinities,
  many arithmetic operations have been specified on
  these values consistently with IEEE~754.
\item For implementations that support a distinct -0.0, the semantics
  of many arithmetic operations with regard to -0.0 has been specified
  consistently with IEEE~754.
\item Scheme's real number objects now have an exact zero as their imaginary part.
\item The specification of {\cf quasiquote} has been extended.  Nested
  quasiquotations work correctly now, and {\cf unquote} and {\cf
    unquote-splicing} have been extended to several operands.
\item Procedures now may or may not refer to
  locations.  Consequently, {\cf eqv?} is now unspecified in a few
  cases where it was specified before.
\item The mutability of the values of {\cf quasiquote} structures has
  been specified to some degree.
\item The dynamic environment of the \var{before} and \var{after}
  procedures of {\cf dynamic-wind} is now specified.
\item Various expressions that have only side effects are now allowed
  to return an arbitrary number of values.
\item The order and semantics for macro expansion has been more fully
  specified.
\item Internal definitions are now defined in terms of {\cf letrec*}.
\item The old notion of program structure and Scheme's top-level
  environment has been replaced by top-level programs and libraries.
\item The denotational semantics has been replaced by an operational
  semantics based on an earlier semantics for the language of the
  ``Revised$^5$ Report''~\cite{R5RS,mf:scheme-op-sem}.
\end{itemize}

%%% Local Variables: 
%%% mode: latex
%%% TeX-master: "r6rs"
%%% End: 
 \par
\newpage
\renewcommand{\bibname}{References}

\bibliographystyle{plain}
\bibliography{abbrevs,rrs}

\vfill\eject


\newcommand{\indexheading}{Alphabetic index of definitions of
  concepts, keywords, and procedures}
\texonly
\newcommand{\indexintro}{The index includes entries from the library
  document; the entries are marked with ``(library)''.}
\endtexonly

\printindex

\clearpage\end{CJK*}                              % if you are typesetting your resume in Chinese using CJK; the \clearpage is required for fancyhdr to work correctly with CJK, though it kills the page numbering by making \lastpage undefined
\end{document}
